\begin{savequote}[75mm]
What I cannot create, I do not understand.
\qauthor{Richard Feynman}
\end{savequote}

% pending plagiarism check
\begin{flushleft}
\chapter{Image-based Profiling of genetically engineered models of early colorectal cancer development}

\section{Motivation}

In the previous chapter I demonstrated that (1) interpretable multi-view representations from patient derived organoids can be learnt and (2) treatment-induced morphological changes can be interpreted using these multi-view representations to guide their mechanistic validation. Building on these observations, I wondered whether this approach could be extended to further understand early colorectal cancer pathogensis and identify small molecule treatments that interfere with disease-associated phenotypes. To this end I created a set of four genetically engineered mouse colon organoid lines that model the initiation of colorectal cancer by harboring tumorigenic Apc and Kras mutations in isolation and combination. The organoid models were then characterised using transcript expression, proteomics and lipidomics measurements. A high-throughput image-based profiling experiment covering c.a 1700 well-annotated FDA-approved, natural and targeted small molecules was performed. I then explored wether a multi-view representation model could be used to interpret treatment-induced morphologies and identify small molecules that acted specifically on phenotype changes caused by tumorigenic Apc and Kras mutations. 

\begin{figure}[h]
\centering
\includegraphics[width=250,
                height=\textheight,
                keepaspectratio]{figures/adenomaprofiling/pdf/fig_0_2.pdf}
\caption[Visual abstract of adenoma model profiling project]{\textbf{Visual abstract of adenoma model profiling project.} Mouse colon organoid models were developed, characterized and subjected to image-based small molecule profiling. The resulting data was used to learn a representation space of organoid states, identify state dependent changes in organoid lipid composition and generate hypotheses about small molcules that potentially have the ability to move organoid states.}
\label{fig_a02}
\end{figure}
\bigbreak



\section{Generation and characterization of organoid models}
\subsection{Generation of organoid colon adenoma models}
The emergence of colorectal cancer via the chromosomal instability process is a well understood sequence of genetic events that start with hyperactivation of canonical Wnt signaling, i.e. through truncating mutations of APC, followed by the hyperactivation of RAS-MAPK signaling, i.e. via oncogenic mutations of KRAS. I genetically engineered mouse colon organoid models carrying Apc truncating mutations and/or a Kras G12D allele, thereby modelling the first set of genetic events within the chromosomal instability process leading to colorectal cancer. 

\begin{figure}[H]
\centering
\includegraphics[width=\textwidth,
                height=\textheight,
                keepaspectratio]{figures/adenomaprofiling/pdf/fig_1_0.pdf}
\caption[Establishing organoid models of colon adenoma]{\textbf{Establishing organoid models of colon adenoma, a} Overview of organoid model establishment. Mouse colon organoids were isolated from a transgenic donor animal carrying an inactive conditional oncogenic KrasG12D allele. Homozygous truncation of Apc via CRISPR and activation of the heterozygous KrasG12D allele lead to four different genetically defined organoid models.
\textbf{b} In vitro growth factor dependency of adenoma models. Organoids were cultured in complete or modified medium containing combinations of Wnt3A, R-Spondin1-Fc and EGF for 120h and subsequently imaged. Scalebar = 200µm.
\textbf{c}	Oncogenic KrasG12D increases resistance to Egfr inhibition. Organoid ATP levels were measured 4 days after Gefitinib treatment and adjusted for organoid growth rate. Points represent mean of n=2 independent experiments. Error bars represent standard error of mean.
\textbf{d} Erk phosphorylation is increased by oncogenic KrasG12D. Organoid models were cultured with or without Wnt3A and R-Spondin1-Fc for 72h and analyzed for protein levels. p, phospho.   
\textbf{e}	Loss of Apc induces transcription of canonical Wnt-signaling target genes. qRT–PCR for Axin2 and Ccnd in the presence or absence of Wnt 3a and R-spondin1-Fc after 120h of culture. Expression levels are normalized to Sdha and Hprt transcript abundance. Bar graphs represent the mean of n=4 independent experiments. Wilcoxon rank sum test 
}
\label{fig_a10}
\end{figure}
\bigbreak

To model the formation of colon adenomas in vitro, I used a transgenic mouse to derive organoid cultures. The transgenic animal carried a conditional tamoxifen inducible KrasG12D/+ allele \citep{Jackson2001-wv} (Figure \ref{fig_a10} a). After isolation, I confirmed that extracted colon organoids did not express an activated form of KrasG12D (Figure \ref{fig_a11}a) and defined these organoids as wildtype (WT). To model loss-of-function mutations of the tumor suppressor Apc, the ortholog of the frequently mutated mutation-cluster-region on the APC gene was targeted by CRISPR (Figure \ref{fig_a10} a). Generated organoids harbored biallelic loss-of-function mutations in Apc (Figure \ref{fig_a11}a). Subsequent activation of oncogenic KrasG12D by treatment with 4-Hydroxytamoxifen led to four distinct organoid adenoma models (Figure \ref{fig_a10} a and Figure \ref{fig_a11}a-b); wildtype (WT), Apc-/- (A), KrasG12D/+ (K), and Apc-/- / KrasG12D/+ (AK).


\begin{figure}[h]
\centering
\includegraphics[width=\textwidth,
                height=\textheight,
                keepaspectratio]{figures/adenomaprofiling/pdf/fig_1_1.pdf}
\caption[Structural validation of organoid colon adenoma models]{\textbf{Structural validation of organoid colon adenoma models, a} Allele-specific PCR products of colon organoid models isolated from a transgenic mouse with a conditional tamoxifen inducible KrasG12D/+ allele.
\textbf{b} Amplicon sequencing result of the murine mutation cluster region ortholog for organoids transfected with an Apc targeting sgRNA and Cas9 carrying plasmid. The sequencing results show the presence of 3 different insertion/deletions within the pool of sgRNA treated organoid models. Wildtype sequences are absent within the CRISPR targeted pool, while mutant sequences are absent in the untreated organoid pool.}
\label{fig_a11}
\end{figure}
\bigbreak

Similar to genetically modified human colon organoids (\citet{Drost2015-ph}; \citet{Matano2015-zw}), murine colon organoids showed characteristic niche requirements. Both Apc mutant organoid lines grew independent of the Wnt-signaling activating factors Wnt 3a and R-Spondin1 (Figure \ref{fig_a10} b) and at an accelerated proliferation rate. In fact, Apc mutant lines showed an increased growth in a Wnt3a and R-Spondin1 free environment when compared to the complete medium. Organoid models with an activated KrasG12D allele were less sensitive to removal of EGF from the media. However, as observed before \citep{Drost2015-ph}, the mutant KrasG12D allele was insufficient to compensate completely for the loss of EGF from the medium. Nevertheless, KrasG12D mutant organoid lines were more resistant to pharmacological inhibition of Egfr signaling (Figure \ref{fig_a10} c). In conclusion, organoid model genotypes were reflected in characteristic growth factor dependencies in experimental conditions.

\bigbreak
Next, I investigated the effects of mutations in Apc and Kras on both canonical Wnt- and Erk dependent signaling. While the presence of the KrasG12D/+ allele led to an increase in Erk-phosphorylation across models, Apc-/- / KrasG12D/+ organoids showed no marked additional increase in Erk-phosphorylation when compared to KrasG12D/+ organoids (Figure \ref{fig_a10} d). Moreover, Apc-/- / KrasG12D/+ adenoma models showed no significant differences in expression of the Wnt target genes Axin2 and Ccnd when compared to Apc-/- single-mutant models (A) (p > 0.34 for all conditions, Wilcoxon rank sum test) (Figure \ref{fig_a10} e). These results indicate that organoid adenoma models show genotype-dependent activity of characteristic signaling pathways, while there is no extensive crosstalk between the Apc-/-  and KrasG12D/+ allele in mouse colon organoids that that is directly reflected in canonical Wnt- and Erk dependent signaling.  

\bigbreak
\subsection{Molecular characterisation of organoid models}
To comprehensively characterise the four organoid models, transcriptome, proteome and lipidome profiling were performed using mRNA microarrays and mass spectrometry, respectively (Figure \ref{fig_161}). For these measurements organoids were kept in the same culture condition and duration that were used during the subsequent image-based profiling experiment: After passaging, all organoids were kept in Wnt 3a, R-Spondin1 rich medium (WENRAS) to model conditions within the niche and stimulate outgrowth before the medium was changed to a Wnt 3a-free medium (ENR) to model conditions outside the niche. The medium was supplemented with EGF both before and after the medium change. Transcriptome profiling of organoid models showed an increased expression of the stem-cell marker Lgr5 and negative Wnt-signaling regulators such as Nkd1, Notum, Wif1 and Znrf3 in Apc mutant organoid lines (Figure \ref{fig_160} b). In contrast, Apc wildtype organoid lines showed an increased expression of epithelial differentiation markers, such as Krt20, Alpp and Abcb1 (P-glycoprotein). Overall, the number of genes with significant expression changes after Apc loss was 2.5 times greater compared to isolated KrasG12D activation (FDR = 0.1, Apc-/-: 44.5\%, KrasG12D/+: 18.3\% of assessed genes). A related observation was made during the analysis of protein abundance. Again, Wnt signaling regulators (Axin2, Notum) were enriched in Apc mutant organoid lines and the number of significantly regulated proteins after Apc loss was 2.5 times greater compared to an isolated KrasG12D activation (FDR = 0.1, Apc-/-: 260, KrasG12D/+: 105 assessed proteins). Principal component analysis of both transcriptome, proteome and lipidome data showed related axes of variation across measurements. In all observed modalities, the first principal component captured differences between Apc wildtype and Apc mutant organoid models, while the second (in case of proteomics measurements the third) principal component captured differences between wildtype and KrasG12D/+ single-mutant models (Figure \ref{fig_161}b, \ref{fig_161}c and \ref{fig_161}d). In every modality, a high degree of similarity was observed among Apc-/- and Apc-/- / KrasG12D/+ organoid lines. While activation of oncogenic KrasG12D in wildtype organoids led to global changes in transcript, protein and lipid expression, these changes were not as pronounced in organoids without functional Apc. In fact, only the mRNA expression of 91 genes was significantly altered between Apc-/- and Apc-/- / KrasG12D/+ organoids (FDR = 0.1). 

\begin{figure}[H]
\centering
\includegraphics[width=\textwidth,
                height=\textheight,
                keepaspectratio]{figures/adenomaprofiling/pdf/fig_1_6_1_2.pdf}
\caption[Molecular characterisation of organoid adenoma models]{\textbf{Molecular characterisation of organoid adenoma models. a} Differential gene expression of adenoma models. Shown are scaled expression values for the top 125 differentially expressed genes for every organoid line. Selected genes are highlighted. All organoids were cultured for 3 days in WENRAS before exposure to ENR for 4 days. Cell number was controlled between experiments. Whole organoid lysates were analyzed. 
\textbf{b} Transcript abundance data. Shown are the first two principal components of scaled gene expression data. The proportion of variance of each principal component is listed in parenthesis. 
\textbf{c} Protein abundance data. Shown are the first and third principal component of scaled protein expression data. The proportion of variance of each principal component is listed in parenthesis. 
\textbf{d} Lipid species abundance data. Shown are the first two principal components of scaled lipid abundance data. The proportion of variance of each principal component is listed in parenthesis. 
\textbf{e} Loss of Apc leads to increased expression of proliferation and intestinal stem cell associated genes. Shown is a gene set enrichment analysis of differentially expressed genes between Apc mutant and WT organoids. Intestinal gene expression signatures were used according to Merlos-Suarez et al. NES, normalized enrichment score. 
\textbf{f} Overview of cellular processes in organoid adenoma models. Shown are selected enriched differential gene expression signatures from Reactome and Merlos-Suarez et al. NES, normalized enrichment score. NES > 0 suggests an enriched/ activated biological process. FDR < 0.1.}
\label{fig_161}
\end{figure}
\bigbreak

To explore active biological processes, gene set enrichment analysis on organoid transcript expression data was performed. The strongest changes in expression after loss of Apc were linked to an increased proliferation rate (Figure \ref{fig_160} e). Gene set enrichment analysis of published intestinal cell-proliferation and stem cell signatures showed an enrichment of both signatures in Apc-/- organoids (Figure \ref{fig_160} e) \citep{Merlos-Suarez2011-gd}. In contrast, a signature for differentiating, transit-amplifying cells was depleted. Gene set enrichment analysis of Apc-/- / KrasG12D/+ double-mutant organoids showed the same results. Next to these published signatures, I explored the enrichment of curated gene sets from the Reactome database \citep{Griss2020-qi}. Here, both Apc-/- and Apc-/- / KrasG12D/+ double-mutant lines showed a positive enrichment of cell cycle and DNA repair related genes when compared to wildtype organoids (Figure \ref{fig_162}a). Unique to the KrasG12D/+ organoid line was a decreased expression of citric acid cycle and respiratory chain related genes (Figure \ref{fig_162}b). This effect, was not observed in Apc-/- / KrasG12D/+ double mutant organoids (Figure \ref{fig_161}f). In addition, organoid models with an KrasG12D/+ genotype showed a downregulation of the EGFR receptor, in line with a potential negative feedback response to hyperactivated RAS-MAPK signaling (Figure \ref{fig_162}b). Both Apc-/- and KrasG12D/+ organoid models showed a strong reduction of lipid metabolism and beta-oxidation (Figure \ref{fig_162}a,b). In summary, (1) loss of Apc leads to a global shift in transcript, protein, and lipid composition in colon organoids, including a strong increase in cell proliferation associated transcripts; (2) Activation of isolated oncogenic KrasG12D leads to pronounced reduction in citric acid cycle related transcripts while this phenotype was not seen in organoid models with an additional loss of Apc; (3) Both Apc loss and activation of oncogenic KrasG12D lead to a reduction of lipid beta-oxidation related transcripts.

\begin{figure}[h]
\centering
\includegraphics[width=\textwidth,
                height=\textheight,
                keepaspectratio]{figures/adenomaprofiling/pdf/fig_1_6_2.pdf}
\caption[Representative up and down-regulated transcriptional processes after loss of Apc and activation of oncogenic Kras G12D]{
\textbf{a} Representative up and down-regulated transcriptional processes after loss of Apc. Expression signatures were sourced from Reactome and average log2 fold changes for included transcripts are illustrated. FDR < 0.1.
\textbf{b} Representative up and down-regulated transcriptional processes after activation of oncogenic Kras G12D. Expression signatures were sourced from Reactome and average log2 fold changes for included transcripts are shown. FDR < 0.1.
}
\label{fig_162}
\end{figure}
\bigbreak

To further understand the pronounced changes in fatty acid metabolism observed in both Apc-/- and KrasG12D/+ organoid models, differences in lipid composition were measured using untargeted lipid extraction and Mass Spectrometry. In total, more than 350 lipids from 15 species were identified across all samples (Figure \ref{fig_168} a). The majority of identified lipids had fatty acid chain lengths of 20 to 40 carbon atoms, while Triglycerides (TAG) had an increased length of 40 to 60 carbon atoms (Figure \ref{fig_168} b). Major differences between organoid lines were seen especially for storage lipids - Triglycerides (TAG) and Cholesterol Esters (CE) (Figure \ref{fig_168} c and d, respectively). Both lipid species were more abundant in the two Apc-/- organoid lines (p < 0.05, TAG: t > 4.8, CE: t > 3.7, ANOVA). In single-mutant KrasG12D/+ organoids, Triglycerides were also more abundant compared to wildtype organoids (p < 0.05, t = 5.9, ANOVA), while Cholesterol esters were depleted (p < 0.05, t = -3.7, ANOVA). The increase in the abundance of storage lipids as a result of Apc loss of function and oncogenic KrasG12D mutation were aligned with the transcriptional changes that indicated a reduced rate of beta-oxidation in these models (Figure \ref{fig_162}a,b). 

\begin{figure}[h]
\centering
\includegraphics[width=350,
                height=\textheight,
                keepaspectratio]{figures/adenomaprofiling/pdf/fig_1_6_8.pdf}
\caption[Lipid composition changes across organoid adenoma models]{\textbf{Lipid composition changes across organoid adenoma models. a} Lipid species abundance of adenoma models. Shown are scaled abundance values for major lipid species for every organoid line. Selected lipid species are highlighted (Lipid Maps Abbreviations: HexCer - Hexosylceramide; PA - Phosphatidate; Cer - Ceramide; SM - Sphingomyelin; LPC - Lysophosphatidylcholine; PS - Phosphatidylserine; PE - Phosphatidylethanolamine; PI - Phosphatidylinositol; PG - Phosphatidylglycerol; DAG - Diacylglycerol; PC - Phosphatidylcholine; TAG - Triacylglycerol; Hex2Cer - Dihexosylceramide; CE - Cholersterolester, Chol - Cholesterol). All organoids were cultured for 3 days in WENRAS before exposure to ENR for 4 days. Cell number was controlled between experiments. Whole organoid lysates were analyzed. 
\textbf{b} Fatty acid chain lengths by lipid species. Shown are the distribution of fatty acid chain lengths.
\textbf{c} Distribution of Triacylglycerol (TAG) abundance across organoid adenoma models. ANOVA was performed to model average lipid abundance as a function of organoid line across replicates.
\textbf{d} Distribution of Cholesterolester (CE) abundance across organoid adenoma models. ANOVA was performed to model average lipid abundance as a function of organoid line across replicates.
}
\label{fig_168}
\end{figure}
\bigbreak


\newpage
\section{Image-based profiling of organoid models}

\begin{figure}[h!]
\centering
\includegraphics[width=\textwidth,
                height=\textheight,
                keepaspectratio]{figures/adenomaprofiling/pdf/fig_1_2.pdf}
\caption[Image-based profiling of organoid adenoma models]{\textbf{Image-based profiling of organoid adenoma models. a} Overview of experiments. Organoids were isolated from a transgenic mouse model and genetically edited. Organoids were dissociated and evenly seeded in 384-well plates before perturbation with an experimental small molecule library. After treatment, high-throughput fluorescence microscopy was used to capture the morphology of organoids in 16 selected z-layers and 3 channels. 3D imaging data were projected on a 2D plane using a maximum contrast projection. Here, only pixel areas with the largest contrast among the z-axis were retained. Morphological features were computed based on the projection. Untreated organoid morphology, organoid size and treatment activity scores were integrated with transcript expression, protein abundance, lipid abundance and genogtype data in a Multi-Omics Factor Analysis (MOFA) model. Figure created with support from Johannes Betge (graphical presentation). 
\textbf{b} Uniform Manifold Approximation and Projection (UMAP) of organoid-level features for a random 5\% sample out of imaged organoids. The identical sample is used for visualizations throughout the figure. Organoid genotype is colorcoded and representative images are displayed (magenta = DNA, cyan = actin, cell permeability = yellow, scale-bar: 200µm). \textbf{c} Graph-based clustering of organoids by morphology with 8 resulting clusters. \textbf{d} Organoid size distribution. Color corresponds to the log-scaled organoid area (dark blue: minimum size, yellow: maximum size).}
\label{fig_120}
\end{figure}
\bigbreak

\subsection{Single-Organoid level phenotypes are organised by model genotypes}
Once models were characterised on a molecular level, the previously developed image-based profiling approach was applied. Organoid models of the four different genotypes were perturbed with a library of ca. 1700 compounds and morphological profiles were systematically observed (Figure \ref{fig_120} a). As previously described, classic morphological features were collected for single organoids, normalised, and principal components were calculated, of which 25 components (accounting for >80\% of the variance) were retained and used throughout the further analysis. A UMAP projection of single-organoid morphology showed distinct genotype-dependent morphological states for identified organoids (Figure \ref{fig_120} b). Graph based clustering of organoid morphology profiles resulted in 8 clusters (Figure \ref{fig_120} c). Organoids within cluster 4 and 3 were enriched for Apc+/+ organoid models, cluster 2 and 1 were populated by Apc-/- models. Analogous to gene expression, lipidomics and proteomics representation space, the two Apc mutant organoid models were less distinct from each other than organoids with a WT and isolated KrasG12D/+ genotype (Figure \ref{fig_140} b). While developed organoids that present with a larger organoid area showed distinct genotype-specific morphologies, small organoids and non-viable organoid fragments clustered together across genotypes within cluster 5 (Figure \ref{fig_120} c and d). The distribution of DMSO-treated organoids and small molecule perturbed organoids in morphological space overlapped considerably (Figure \ref{fig_140} a), most likely because many treatments were inactive and did not alter organoid morphology. 

\bigbreak
\begin{figure}[h!]
\centering
\includegraphics[width=\textwidth,
                height=\textheight,
                keepaspectratio]{figures/adenomaprofiling/pdf/fig_1_4.pdf}
\caption[Treatment and genotype dependent effects within the organoid morphology distribution]{\textbf{Treatment and genotype dependent effects within the organoid morphology distribution. a} UMAP representation of DMSO treated (vehicle) and small molecule treated organoids. \textbf{b}, UMAP embeddings of four organoid genotypes (baseline state = 0.1\% DMSO control-treated organoids), grey background consists of randomly sampled organoids.}
\label{fig_140}
\end{figure}

When comparing the morphologies of different organoid lines in detail, characteristic differences were identifiable (Figure \ref{fig_130} a). DMSO-treated Apc+/+ organoids showed a strong, regular apical actin cytoskeleton (high average actin intensity) that organized the multicellular formation into a regular-patterned spherical morphology (low average eccentricity). In contrast, Apc-/- organoids showed a relative lack of a regular actin cytoskeleton (low average actin intensity) and a irregular, non-spherical morphology (high average eccentricity). In summary, organoid models showed genotype-dependent differences in morphology. Analogous to differences in molecular state, a primary source of variation was caused by loss of the tumor suppressor gene Apc. Organoids with truncated Apc presented with a higher proliferation rate, increased overall DNA staining intensity and loss of the regular spherical apical actin cytoskeleton that was observed in Apc +/+ organoid models.

\begin{figure}[h!]
\centering
\includegraphics[width=200,
                height=\textheight,
                keepaspectratio]{figures/adenomaprofiling/pdf/fig_1_3.pdf}
\caption[Genotype dependent effects on organoid morphology]{\textbf{Genotype dependent effects on organoid morphology. a}  Morphological organoid profiles from vehicle-treated adenoma models were aggregated. Shown are representative individual organoids with selected features. Points show the mean phenotype for each independent biological replicate. Representative, interpretable features and their z-scores relative to all single organoid profiles are shown (magenta = DNA, cyan = actin, cell permeability = yellow, scale-bar: 25µm)}
\label{fig_130}
\end{figure}
\bigbreak



\subsection{Scoring small molecule induced phenotypes across organoid models}

\begin{figure}[h!]
\centering
\includegraphics[width=\textwidth,
                height=\textheight,
                keepaspectratio]{figures/adenomaprofiling/pdf/fig_1_5_2.pdf}
\caption[Treatment activity scoring]{\textbf{Treatment activity scoring. a} A logistic regression classifier is trained to distinguish morphology profiles of individual treated and untreated organoids across all available replicates. Afterwards, the classifier is applied to a validation set of organoids and the classification performance is estimated using the area under the receiver operating characteristic curve (AUROC) metric. Method implemented by Jan Sauer.
\textbf{b} Distribution of treatment activity scores for all organoid lines, replicates and perturbations. 
\textbf{c} Identifying related treatment induced phenotypes. Normal vectors of treatment specific classifiers were compared by calculating the angular distance (related to cosine similarity, ranging from 0-180 degrees). Small angular distance between vectors correspond to a high similarity between the treatment-induced organoid phenotypes. Method implemented by Jan Sauer.
\textbf{d} A clustered heatmap of compound induced phenotypes for Apc mutant and Apc wildtype organoids. Highlighted are clusters of compound induced phenotypes with related targets. Normal vectors for Apc mutant and Apc wildtype organoids were concatenated before angular distance calculation. Method implemented by Jan Sauer.
}
\label{fig_150}
\end{figure}
\bigbreak

After identifying genotype-dependent morphological differences, the next step was to explore effects of small molecule treatment on different organoid models. To describe the activity of a treatment, the classification-based approach developed during the study of human cancer organoid phenotypes in the previous chapter was used. Briefly, for every treatment and genotype, a logistic regression classifier was trained to distinguish DMSO-treated organoids from treated organoids. The classification performance, expressed as the AUROC, was used to determine the activity of a treatment. A high AUROC score (approaching 1) is observed for compounds that lead to a treatment-induced organoid morphology that is very distinct from DMSO treated organoids. In contrast a low AUROC (centered around 0.5) is observed for compounds where the model's classification performance approaches random guessing (Figure \ref{fig_150} a). The distribution of activity scores across organoid lines showed that most compounds did not lead to an identifyable morphological change (Figure \ref{fig_150} b) and showed AUROC values centered around 0.5. Next to identifying differences around the number of active treatments, I was interested what small molecules were active in a given organoid genotype and their treatment-induced morphology change. Based on the observation that the primary source of variation for treatment activity was the state of the Apc allele, I aggregated organoid lines by their Apc allele for further analysis. In line with the approach taken in the previous chapter, normal vectors of the logistic regression classifiers were compared using the cosine distance (Figure \ref{fig_150} d). The resulting clustering of treatments showed an enrichment for small molecules with related mechanism of action (Figure \ref{fig_150} e and f). For example, EGFR inhibitors were significantly enriched in Apc-mutant organoid lines, while GSK3B-inhibitors, which lead to a stimulation of canonical Wnt signaling, were only enriched in Apc-wildtype organoid models. To summarise the findings above, organoid models showed genotype-specific treatment-induced phenotypes. For example, GSK3B-inhibitors were active in Apc +/+ organoids and showed a characteristic treatment-induced phenotype in these models.

\section{Multi-omics factor analysis identifies shared factors linking functional and structural biological views}

\subsection{Learning a multi-view representation with MOFA}

\begin{figure}[h!]
\centering
\includegraphics[width=400,
                height=\textheight,
                keepaspectratio]{figures/adenomaprofiling/pdf/fig_1_7.pdf}
\caption[Multi-omics factor analysis (MOFA) to identify shared factors linking morphology, size, gene expression, lipidomics, proteomics, genotype and treatment activity]{\textbf{Multi-omics factor analysis (MOFA) to identify shared factors linking morphology, size, gene expression, lipidomics, proteomics, genotype and treatment activity. a} Percent variance explained by the MOFA model for each factor. Untreated organoid morphology, organoid size and treatment activity scores were integrated with genotype, proteomics, lipidomics and mRNA expression data. \textbf{b} Cumulative proportion of total variance explained by each experimental data modality within the MOFA model. \textbf{c}, Visualization of samples in factor space showing factors 1 and 2 as well as factor 1 and 3. Shown are independent replicates for each organoid line. 
}
\label{fig_170}
\end{figure}
\bigbreak

To jointly model the biological and morphological state of the four organoid lines, I performed multi-omics factor analysis (MOFA). Analogous to the process described in the previous chapter, molecular and morphological features from untreated organoids were factorized using k=4 factors (Figure \ref{fig_170} a and \ref{fig_180} a). The learned model was based on both morphological (e.g. morphology, size, small molecule activity) and molecular (e.g. genotype, proteomics, lipidomics and transcript expression) information. To reduce the dimensionality of data modalities with a high number of features, only high variance features from gene expression and proteomics analysis were used. The resulting factorization explained the data ranging from ca. 90\% (gene expression) to 50\% (morphology) of explained variance ($R^{2}$) across the analyzed views. The first three factors captured the majority of variance, >40\%, ca. 10\%, and <10\%, respectively (Figure \ref{fig_170} a). The learned model explained most variance within the mRNA expression and genotype data, while measurements within the organoid morphology data had the lowest explained variance (Figure \ref{fig_170} b). When inspecting factor weights for the morphology domain, principal component 1, which accounted for ca. 30\% of variance within all observed (treated and untreated) morphology data during the image processing had small relative assigned feature weights across all learnt factors (from high to low absolute feature weight: rank 8/25, 11/25, 24/25 and 25/25 for factors 1 through 4, respectively). In contrast, principal components 2, 3, and 4 had the highest assigned feature weights across all four factors. In summary, MOFA was used to learn k=4 factors across a set of distinct data modalities. In the imaging domain, the learnt representation captured only a subspace of the observed morphological space, possibly because of high noise levels in morphological data and the fact that only observations from untreated organoids were contained in the modeled support set. 
\par

Visual inspection of factors as well as exploration of factor weights within the genotype view showed that factor 1 explained differences caused by Apc loss of function, while factor 2 explained differences caused by the activation of KrasG12D in an Apc+/+ genotype (Figure \ref{fig_170} c and \ref{fig_180} b). In contrast to factor 2, factor 3 captured differences between Kras+/+ and KrasG12D/+ organoids with Apc loss of function. While the number of factors is a user-defined hyperparameter within MOFA, the method automatically drops excess factors if they are not considered effective based on an applied automatic relevance determination (ARD) prior \citep{Argelaguet2018-yi}. Increasing the number of factors above k=4 in this analysis, did not lead to an increased number of interpretable factors. In fact, factor 4 already did not capture differences between organoid genoypes and was not interpretable from a biological point of view by the author (Figure \ref{fig_180} b). The determined number of explanatory factors corresponded to the hypothesized intrinsic dimensionality of the data: effects attributed to the Apc-/- allele, the Kras G12D allele, and their interaction.

\begin{figure}[h!]
\centering
\includegraphics[width=400,
                height=\textheight,
                keepaspectratio]{figures/adenomaprofiling/pdf/fig_1_8.pdf}
\caption[Multi-omics factor analysis input data and loadings]{\textbf{Multi-omics factor analysis input data and loadings. a} measurement modalities, dimensionality and number of measurements. A third replicate of measurements were available for proteomics and lipidomics only. \textbf{b} Factor loadings for genotype information.} 
\label{fig_180}
\end{figure}
\bigbreak

\subsection{A canonical Wnt signaling associated program caused by Apc loss}

To understand the molecular changes associated with factor 1, factor loadings for mRNA expression data were analyzed using Reactome gene-set enrichment analysis (Figure \ref{fig_190} a). Three clusters of biological processes were significantly associated with a negative factor loading, caused by Apc loss-of-function: 1) Mitotic Anaphase related processes, including spindle checkpoints; 2) Mitotic S-phase, including DNA replication and 3) DNA repair mechanisms, including homology directed repair. In line with the enrichment of processes associated with cell proliferation, factor 1 loadings showed an enrichment of the previously described intestinal proliferation signature (Figure \ref{fig_190} c) and an LGR5+ instestinal stem cell identity signature (Figure \ref{fig_190} b). These findings are in line with the long-standing evidence that loss of Apc leads to a hyperactivation of canonical Wnt signaling, which in turn leads to increased intestinal cell proliferation and Myc-dependent changes towards a stem-like cell state \citep{Sansom2007-wm, Satoh2017-nd}. When focusing on compound activity, a low factor 1 score was significantly linked to increased activity of microtubuli and focal adhesion kinase (FAK) targeting small molecules (Figure \ref{fig_190} d). This morphological sensitivity presented itself primarily as reduced organoid size and number relative to the DMSO vehicle control (Figure \ref{fig_190} e). In contrast, the average treatment activity scores of small molecules targeting Wnt signaling were associated with high factor 1 scores (Figure \ref{fig_190} d).  

%\begin{figure}[h]
%\centering
%\includegraphics[width=\textwidth,
%                height=\textheight,
%                keepaspectratio]{figures/adenomaprofiling/pdf/fig_1_9.pdf}
%\caption{\textbf{Factor loadings for treatment activity. a} Factor 1 and 2 loadings, and \textbf{b} Factor 1 and 3 loadings. Average treatment activity score (AUROC) is color coded.}
%\label{fig_180}
%\end{figure}
%\bigbreak

\begin{figure}[h!]
\centering
\includegraphics[width=\textwidth,
                height=\textheight,
                keepaspectratio]{figures/adenomaprofiling/pdf/fig_2_1.pdf}
\caption[Factor 1, canonical Wnt signaling]{\textbf{Factor 1, canonical Wnt signaling. a} Gene-set enrichment network of factor 1 gene expression loadings. An edge connects Reactome pathways with more than 20\% overlap. Central enriched processes include mitosis, DNA replication and DNA damage repair. \textbf{b and c} Gene set enrichment results of the "Lgr5 intestinal stem cell" and "proliferation" signature by Merlos-Suarez et al \citep{Merlos-Suarez2011-gd}. over ranked factor 1 gene expression loadings (ranking from high factor 1 loading to low factor 1 loading, NES = normalized enrichment score). \textbf{d} Distributions of treatment activity loadings grouped by drug target for factor 1. \textbf{e} Example images of compound treated organoids with WT or Apc-/- genotype. Representaধve images are displayed (magenta = DNA, cyan = actin, yellow = cell permeability, scale-bar: 200μm).}
\label{fig_190}
\end{figure}
\bigbreak

Further exploration of the association between the treatment activity score and Apc genotype showed that small molecule inhibitors of the canonical Wnt secretion pathway protein Porcupine (Porcn), IWP-L6 and LGK-974, were more active in Apc+/+ organoids relative to their Apc-/- counterparts \citep{Liu2013-dh} (Figure \ref{fig_199}a). In contrast, this effect was not observable for PRI-724, a small molecule inhibitor targeting the interaction of beta-catenin and CREB-binding-protein in the canonical Wnt signaling pathway \citep{Okazaki2019-gy} (Figure \ref{fig_199}a). The observed differences in treatment activity scores among small molecule inhibitors are most likely related to their targets' relative location to Apc in the canonical Wnt signaling cascade. While Porcn-dependent Wnt secretion is generally upstream of the Apc-scaffolded destruction complex, the interaction of beta-catenin and the transcriptional coactivator CREB-binding-protein is located downstream of it. As a consequence, inhibition of destruction complex function by loss of Apc is expected to render cells less sensitive to perturbations of the Wnt secretion cascade than direct perturbations of transcription factor binding properties (Figure \ref{fig_199}b).

\begin{figure}[h!]
\centering
\includegraphics[scale=0.75,keepaspectratio]{figures/adenomaprofiling/pdf/fig_2_2_1.pdf}
\caption[Activity of small molecule Wnt signaling inhibitors]{\textbf{Activity of small molecule Wnt signaling inhibitors. a} AUROC activity score for three small molecule inhibitors of canonical Wnt signaling. \textbf{b} Target proteins for small molecules within the canonical Wnt signaling cascade with their relative position to the destruction complex (highlighted in blue).}
\label{fig_199}
\end{figure}
\bigbreak

\newpage

\subsection{A program caused by isolated KrasG12D activation shares signs of oncogene-induced senescence}
While the Apc-/- genotype contributed primarily to factor 1 (Figure \ref{fig_180} b), the KrasG12D allele showed a strong loading for both factor 2 and factor 3. This observation corresponded with the fact that only organoid models without Apc loss of function were separated by factor 2 (Figure \ref{fig_170} c). Factor 2 described a KrasG12D dependent change in cell state in the presence of intact Apc function.


\begin{figure}[h!]
\centering
\includegraphics[scale=0.75,
                keepaspectratio]{figures/adenomaprofiling/pdf/fig_3_1_2.pdf}
\caption[Factor 2, isolated KrasG12D activity and oncogene induced senescence]{\textbf{Factor 2, isolated KrasG12D activity and oncogene induced senescence. a} Distributions of treatment activity loadings grouped by drug target for factor 2. \textbf{b} Relationship of representative ERK inhibitor activity with factor 2 score. Shown are compounds from highlighted groups in panel (a). \textbf{c} Relationship of representative EGFR inhibitor activity with factor 2 score. Shown are compounds from highlighted groups in panel (a). \textbf{d} Gene set enrichment results of a senescence signature \citep{Fridman2008-ky} over ranked factor 2 gene expression loadings (ranking from high factor 2 loading to low factor 2 loading, NES = normalized enrichment score).}
\label{fig_200}
\end{figure}

As above, to understand the molecular mechanisms represented by factor 2, features with large absolute loadings were identified. ERK and MEK inhibitors were more active in factor 2 low models (KrasG12D+/-) while EGFR/HER2 inhibitors were more active in factor 2 high organoids (WT, figure \ref{fig_200} a and b). This juxtaposition in treatment activity against RAS-MAPK pathway members was reminiscent of the previous observations made for canonical Wnt signaling inhibitors (Figure \ref{fig_199}). With oncogenic Kras localized between the receptor-layer (including Egfr and Her2) and downstream mediating kinases (for example Erk), hyperactive Kras signaling likely leads to a cell state with relative resistance to EGFR inhibitors and increased dependency on Erk signaling. The previously observed transcriptional process of Egfr-downregulation as a response to KrasG12D+/- is in line with these observations (Figure \ref{fig_162}b). 

\bigbreak

Oncogene induced senescence is a cell state marked by an arrest of the cell cycle and expression of pro-inflammatory mediators as a response to an oncogenic perturbation. An activated KrasG12D+/- genotype leads to oncogene induced senescence of colon epithelial cells in vivo \citepp{Bennecke2010-zf}. Prompted by previous reports on the effect of an isolated oncogenic Kras allele, I identified an enrichment of a senescence related gene expression signature by Fridman et al. within the loadings of factor 2 \citep{Fridman2008-ky} (Figure \ref{fig_200} c). Transcripts linked to cell senescence, including Igfbp3 (factor 2 loading ca. -0.999) and Hmga2 (factor 2 loading ca. -1.050) ranked among the strongest contributors to the factor. 
\bigbreak


\newpage
\subsection{A program describing the effect of oncogenic Kras in the context of Apc loss of function is marked by increased mTOR signaling }

While factor 2 captured the effect of oncogenic Kras in an Apc wildtype state, factor 3 scores separated organoid models by Kras genotype in the context of an Apc loss of function (Figure \ref{fig_170} c). On average, factor 3 only accounted for less than 10\% of variance across modalities within the MOFA model, supporting the overall similarity of Apc-/- and Apc-/- KrasG12D+/- organoids previously observed separately on the transcriptome, proteome, lipidome and morphology level (Figure \ref{fig_161}a-d and \ref{fig_140} b).

\begin{figure}[h]
\centering
\includegraphics[scale=0.75,
                keepaspectratio]{figures/adenomaprofiling/pdf/fig_4_1.pdf}
\caption[Factor 3, KrasG12D effects in the context of Apc loss of function]{\textbf{Factor 3, KrasG12D effects in the context of Apc loss of function. a} Distributions of treatment activity loadings grouped by drug target for factor 3. \textbf{b} Relationship of representative mTOR inhibitor activity with factor 3 score. \textbf{c} Gene set enrichment results of a Reactome mTORC1 activation signature over ranked factor 3 gene expression loadings (ranking from high factor 3 loading to low factor 3 loading, NES = normalized enrichment score). \textbf{d} Visual summary of Myc gene set enrichment results for organoid state transitions. Myc signatures were significantly enriched in Apc-/- models and depleted in models with isolated KrasG12D+/- mutation.}
\label{fig_300}
\end{figure}
\bigbreak

Exploration of factor 3 loadings identified mTOR and FAK inhibitor activity to contribute to a negative factor score. Double-mutant organoid models (Apc-/- KrasG12D+/-), which had a low factor 3 score, showed a greater treatment activity for these small molecule inhibitors than their single-mutant Apc-/- counterparts (Figure \ref{fig_300} a). This difference in treatment activity was observerable for both ATP-competitive (e.g. INK-128, Sapanisertib) and non-ATP-competitive (e.g. Rapamycin) inhibitors (Figure \ref{fig_300} b). In line with the increased sensitvity to mTOR inhibitors, differential transcript expression analysis comparing single (Apc -/-, abbreviated A) and double mutant (Apc-/- KrasG12D+/-, abbreviated AK) organoids identified a significant increase in mTORC1 activation according to a Reactome signature (Figure \ref{fig_300} c, FDR=0.0095, NES=2.5019). In summary, factor 3 was aligned with the effect of oncogenic Kras in Apc mutant colon organoids and was characterized by increased transcriptional activity and sensitivity to mTOR signaling.


\newpage
\section{Interpreting treatment-induced organoid morphologies with Multi-omics factor analysis to identify small molecules with factor-specific effects}


% perform a scenario in which new observations are added over time, only morphology data is available
% goal is to retrieve something unexpected - concretely a small molecule treatment that is not (1) inactive, (2) unspecifically toxic, but (3) is shifting the organoids along the previously identified factors. Of special interest are compounds that can move Apc or Kras mutatnt organoids (disease precursor state) towards a WT phenotype (healthy state)





To project the observed organoid morphologies (1699 treatments x 25 morphology features) into the previously learnt MOFA representation, the pseudoinverse of the morphology factor-weight matrix was approximated and factor scores were estimated (1699 treatments x 3 factors). Projection of treated organoid morphologies recovered the previously identified distribution of untreated organoid genotypes (Figure \ref{fig_500} c and d). As expected, treatments which showed a strong dispersion away from the centers formed by their respective genotype showed a high AUROC score (Figure \ref{fig_500} e and f).
\bigbreak

\begin{figure}[h]
\centering
\includegraphics[scale=0.75,
                keepaspectratio]{figures/adenomaprofiling/pdf/fig_5_0.pdf}
\caption[Visualization of treated samples projected into factor space]{\textbf{Visualization of treated samples projected into factor space. a and b} Visualization of projected samples along factor 1, factor 2 and factor 3, respectively. Samples are colored by genotype. \textbf{c and d} Visualization of projected samples along factor 1, factor 2 and factor 3, respectively. Samples are colored by AUROC score distance from vehicle control (DMSO).}
\label{fig_500}
\end{figure}
\bigbreak


\begin{figure}[h]
\centering
\includegraphics[scale=0.75,
                keepaspectratio]{figures/adenomaprofiling/pdf/fig_5_3_1.pdf}
\caption[Treatment-induced changes on factor 1 and factor 2 scores]{\textbf{Treatment-induced changes on factor 1 and factor 2 scores. a} Highlighted are treatments with a significant effect on projected scores for factor 1 and factor 2 across all organoid lines (ANOVA, FDR = 0.05). Treatments with a significant effect in only one of the three evaluated factors are highlighted in black. \textbf{b} Example for treatments with unspecific effects. Shown are Proteasome inhibitors. \textbf{c} Example for treatment with specific effects along one factor. Treatments are colored by their literature-derived mechanism of action.)}
\label{fig_180}
\end{figure}
\bigbreak

\subsection{Interpreting treatment-induced organoid morphologies using a learnt multi-omics factor model}

Next, I was interested in identifying treatments that changed organoid morphology along one of the two primary factors (Factors 1 and 2) that were previously identified. To this end an ANOVA modeling the factor scores as a function of the treatment and organoid line (without an interaction term) was performed for each factor (Figure \ref{fig_185} a). Of the x evaluated treatment, y ( \%) showed a significant effect on either factor 1 or factor 2, z showed an effect on both factors, and q showed no effect - thereby creating three groups: (1) treatments with a specific effect, (2) an unspecific effect, and (3) no observed effect. 
\par

As treatments that cause low-viability phenotypes, which were not modeled during training of the MOFA model, can cause spurious and unspecific effects in the learnt factor space, treatments with unspecific effects were inspected first. Among these treatments were multiple compounds which were tested at toxic concentrations including Proteasome inhibitors (i.e. Bortezomib; t =, Delanzomib; t= , Carfilzomib; t= ; Figure \ref{fig_185} b), conventional DNA-intercalating agents (i.e. Epirubicin; t=), and compounds with PAINS-like properties (Ellagic Acid; t=). 
\par

Among factor specific treatments, small molecules that displayed a negative effect on factor 1 (shifting from a state associated with an Apc wildtype allele to an Apc-/- associated state) were enriched for GSK3 beta inhibitors (i.e. CHIR-98014, CHIR-99021, LY2090314). Along factor 2, small molecules targeting downstream mediators of RAS-MAPK signaling (i.e.  PD184352, Ulixertinib) and PI3K-AKT-mTOR signaling (i.e. Buparlisib, KU−0063794) had a positive effect (shifting from a state associated with KrasG12D +/- genotype to a Kras wildtype associated state). No treatment with positive effect on factor 1, indicating a shift from an Apc-/- associated state towards an Apc wildtype state was identified. 

\subsection{GSK3 beta inhibitors move Apc wildtype organoids specifically along a canonical Wnt signaling program}

GSK3 beta is a kinase with central function within the canonical Wnt signaling destruction complex \citep{stamos} and inhibition of GSK3 beta has been shown to lead to hyperactivation of canonical Wnt signaling \citep{Stambolic}. Based on the fact that Apc -/- organoids are "already" in a state of high canonical Wnt signaling, I hypothesized that GSK3 beta inhibition should lead to a larger observed effect along factor 1 in Apc+/+ organoids than their Apc-/- counterparts, a case of treatment-genotype interaction. 

\begin{figure}[h]
\centering
\includegraphics[scale=0.75,
                keepaspectratio]{figures/adenomaprofiling/pdf/fig_2_4_1.pdf}
\caption[GSK3 beta inhibition dependent morphology in colon organoid models]{\textbf{GSK3 beta inhibition dependent morphology in colon organoid models. a} Small molecule inhibition of GSK3 beta (CHIR98014) leads to phenocopying of Apc-/- genotype organoid models. \textbf{b} Shift of morphological features of wildtype and KrasG12D +/- organoid models treated with CHIR98014. Shown is an increase in organoid size (Area) and DNA intensity. \textbf{c} Excerpt of clustering from figure \ref{fig_150} d, labeled with known binding activity of listed small molecules. Rucaparib is not member of the cluster and shown for comparison.}
\label{fig_185}
\end{figure}
\bigbreak

Indeed, the effect on factor 1 was stronger for WT and KrasG12D+/- organoid lines (Figure \ref{fig_185}). Treatment with the GSK3 beta inhibitor CHIR-98014 led to treatment-induced phenotypes in WT and KrasG12D+/- organoids that phenocopied the unperturbed morphology of Apc-/- and double mutant organoid models (Figure \ref{fig_185} b). On the individual morphology feature level, treatment led to an increase in organoid size and DNA intensity (Figure \ref{fig_185} c) in Apc+/+ models. This change in morphology was likely due to an increased proliferation rate of mutant cells, leading to rising organoid size and a higher density of nuclei per analyzed object. Guided by the identification of an interpretable GSK3 beta inhibition induced phenotype, I analyzed small molecules that clustered with known inhibitors of this kinase based on the similarity of their drug effect vectors (Figure \ref{fig_185} d). In line with the observations in this experiment, all small molecules clustering with well-described inhibitors of GSK3 beta had previously described off-target binding activity against this kinase within the LINCS KINOMEScan database \citep{Duan2014-ku}. 
\par 

In summary, projecting treatment-induced morphologies into a factor space that was previously learnt from annotated organoid samples helped interpret small molecule effects and recovered previously known annotated on-target and off-target activity against GSK3 beta. In the future, I expect such an approach to assist in a iterative model-based exploration of complex in-vitro models by introducing structure-based treatment descriptors into the model and updating itl with new collected data at every experimental cycle.
\bigbreak

\end{flushleft}