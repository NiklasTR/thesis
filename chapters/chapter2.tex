\begin{savequote}[75mm]
Nulla facilisi. In vel sem. Morbi id urna in diam dignissim feugiat. Proin molestie tortor eu velit. Aliquam erat volutpat. Nullam ultrices, diam tempus vulputate egestas, eros pede varius leo.
\qauthor{Quoteauthor Lastname}
\end{savequote}

% not yet modified

\chapter{Automated morphological profiling of patient derived organoids}

\section{A screening workflow for 3D organoid models}
To systematically measure morphological phenotypes of patient derived organoids and their morphological changes after treatment with a large number of compounds, I established a platform for high-throughput image-based drug profiling experiments (Figure 2a). Organoids were digested with a modified trypsin derivate, mechanically dissociated and filtered through a cell strainer to seed small fragments evenly onto 384-well imaging plates which were coated with a small volume of basement membrane extract. I standardized the amount and size of seeded organoid fragments in order to reliably measure organoid phenotypes after perturbation.
Also, the vertical distribution of organoid fragments within the BME layer was controlled in order to allow imaging of PDOs within the few captured confocal layers. In order to control the vertival distribution, I 
PDO fragments were incubated for three days to allow organoid formation before drug treatment. 
Subsequently, PDOs were treated with two compound libraries with a total of 527 small molecule inhibitors. Among them were 63 clinically used compounds that were added in 5 different concentrations (Clinical cancer library, N = 15 PDO lines) and a large experimental library (Ki-Stem library, N = 13 PDO lines) of 464 compounds (842 treatments in total; Figure S1 and Supplemental tables S3 and S4). 
Compounds were selected to target diverse developmental and signaling pathways, as well as clinically relevant targets for compound profiling. 
After four days of compound treatment, organoids were fixed and stained for actin (Phalloidin/TRITC), DNA (DAPI), and cell permeability (DeadGreen/FITC). Subsequently, plates were imaged at multiple z-positions by automated confocal microscopy. The procedure was repeated to generate two independent biological replicates of every PDO line.


\section{Selective 3D imaging of organoids during high-content imaging}
To rapidly analyze 3D imaging data of organoids, we developed a software framework called SCOPE (Selective 3D imaging for Contrast based Organoid Projection and feature Extraction). 
First, we projected the 3D image data onto a plane by applying a maximum contrast projection. 
This is an algorithm that uses the contrast surrounding a given pixel to determine the focal plane, allowing a precise structure detection and 2D representation of 3D objects. 
Next, individual organoids were segmented using a deep convolutional neural network. We developed a two-step procedure to establish segmentation: First, organoids were segmented based on fluorescence intensity of all channels. 
Then we used the intensity segmentation data to train a deep convolutional neural network for object identification. 
This improved the segmentation results by far, as observed by visual inspection (Figure 2b).
Subsequently, morphological profiles were calculated for each individual organoid, yielding 486 phenotypic features. 
These features include shape features, such as area and eccentricity, features describing the intensity distributions of each color channel, and texture features. 
Median and median absolute deviation of all features grouped on a well-wise level (973 features in total, including number of organoids per well as additional feature) showed a robust correlation between biological replicates (Figure 2d).
