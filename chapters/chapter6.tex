\begin{savequote}[75mm]
Nulla facilisi. In vel sem. Morbi id urna in diam dignissim feugiat. Proin molestie tortor eu velit. Aliquam erat volutpat. Nullam ultrices, diam tempus vulputate egestas, eros pede varius leo.
\qauthor{Quoteauthor Lastname}
\end{savequote}

\chapter{A landscape of interconnected organoid phenotypes}

\section{Unperturbed and perturbed organoids reside on a shared embedding of organoid morphology}

To further elucidate the heterogeneity of drug induced phenotypes across and within PDO lines, we embedded features on a single organoid level and assessed drug induced changes in organoid morphology (Fig. 6). Unsupervised embedding of 6.6 million organoids showed a continuum of organoid phenotypes that was organized by size (Fig. 6a), phenotypic subsets (Fig. 6b) and was shared between DMSO and drug-treated organoids (Supplementary Fig. 7a). Increasing doses of active compounds led to a step-wise reduction of morphological heterogeneity and a convergence on final compound induced phenotypes (Fig. 6c). Analog to pseudotime in single-cell gene expression analysis, organoids showed dose-dependent, compound-specific trajectories shared across organoid lines (Fig. 6d). To link regions within the landscape to molecular mechanisms, we performed model based clustering (Supplementary Fig. 7b) on the embedding and identified compounds that shifted organoids towards specific regions within the landscape. For example, region R5, which predominantly contained organoids of the disorganized baseline morphological subset, was also enriched for organoids treated with mTOR and PI3K inhibitors (p<10E-15) (Fig. 6e and Supplementary Fig. 7c). We next analyzed the transcriptomes of organoid lines with strong presence in region R5 and identified an upregulation of an IGF1 receptor signaling signature (NES=1.83, FDR=0.051) (Fig. 6f-g). The IGF1R cascade is activated in around 20% of colorectal cancer cases and leads to downstream mitotic stimuli via PI3K and mTOR (Fig. 6h){Zhong:vs}. Active mTOR signaling in turn leads to transcriptional inhibition of IRS-1 in a negative feedback loop{OReilly:2006hk}. In line with the observed drug induced shift in morphology, a reactive induction of IGF1R signaling has been described as a resistance mechanism to small molecule mTOR inhibitors in cancer{Sharma:2010ge}. In addition, further analysis showed that R5 enriched organoid lines showed signs of 11p15.5 loss of imprinting, marked by overexpression of miR-483 and IGF2 (Fig. 6i){Liu:2013cy}, oncogenes implicated in colorectal cancer{Li:2014ee}. In summary, by integrating gene-expression data and single-organoid morphology information, we were able to identify IGF1R signaling to be associated with mTOR inhibitor response, thereby describing a putative resistance mechanism to this class of therapeutics.
These analyses demonstrated that three-dimensional, multicellular PDOs showed characteristic and complex morphological responses to defined molecular perturbations. Furthermore, we explained morphological responses on a molecular level with changes in RNA expression programs of specific pathways. 

\section{Targeted perturbation shifts phenotypes in a mechanism-of-action dependent way}


