\chapter{Introduction}
\label{introduction}

\begin{flushleft}
\setlength{\parindent}{7ex}
\section{Colorectal Cancer}
Colorectal cancer is the third most common cancer worldwide and is associated with a Western lifestyle. Similar to other solid tumors, colorectal adenocarcinoma progression is classified based on the UICC staging into four stages. These range from a \textit{carcinoma-in-situ}, a malignant patch of cells that has not yet breached the basal lamina of the intestinal mucosa (stage 0), to metastatic disease (stage 4).\par

According to the adenoma-carcinoma sequence model, the majority of all colorectal adenocarcinomas arise from previously formed adenomas, benign neoplasms of the intestinal epithelium \cite{Cho1992}. Thus, today, one of the most effective medical interventions to reduce death from colorectal cancer is the preventative removal of visible adenomas during lower endoscopy, such as colonoscopy \cite{Nishihara2013Long-TermEndoscopy}.\par

The most frequent genetic alterations found in colorectal adenocarcinoma are mutations in the \par
% point to following section.

The treatment of colorectal adenocarcinoma depends on disease stage. While surgical removal of the tumor is at the center of the treatment strategy, neoadjuvant and adjuvant chemotherapy are part of the recommended therapy from UICC stage 2 and 4 on, respectively. Today, for metastatic colorectal cancer the first line treatment includes double chemotherapy (FOLFOX or FOLFIRI) paired with Cetuximab (anti-EGFR) for KRAS wildtype disease, double chemotherapy with Bevacizumab (anti-VEGFR) for KRAS mutant disease or triple chemotherapy (FOLFOXIRI) in combination with Bevacizumab for BRAF mutant disease \cite{Cutsem}. The following lines of therapy include different combinations of the aforementioned agents with the exception of Regorafenib and Triflouridin/Tipiracil as preferred third line agents for non-KRAS wildtype disease \cite{Cutsem}. Consequentially, the only genetic tests currently recommended during therapy are the determination of KRAS and BRAF status \cite{Cutsem}. Other genetic tests or targeted inhibitors have so far not found their way into clinical practice, leaving ample room to identify more genetically informed and granular treatment concepts for this disease.\par

Both the important role of adenocarcinoma development and the limitations of personalized therapy in advanced stages of the disease motivated the research presented in this dissertation.\par

\section{Cellular Signaling during Colorectal Cancer Development}

On a molecular level, colorectal adenocarcinoma can be organized into tumors with a chromosomal instability or a DNA-mismatch repair deficiency phenotype \cite{Markowitz2009}. These two forms have been associated with characteristic gene expression signatures, referred to as consensus molecular subtypes (CMS) \cite{Guinney2015TheCancer.}. Here, CMS 1 is tightly linked to a DNA-mismatch repair deficiency phenotype, associated with microsatellite instability and hypermethlyation of CpG islands \cite{Markowitz2009}. In contrast, CMS 2-4 are associated with increased chromosomal instability, making the ladder phenotype the most prevalent, canonical, form. 

The cascade of genetic events leading to this canonical form of colorectal cancer cause hyperactivation of a range of signaling pathways. Briefly, this cascade, known as the Vogelstein sequence \cite{Cho1992}, starts with the loss of the tumor supressor APC in the intestinal epithelium. Loss of APC, which triggers adenoma formation, is followed by the activation of KRAS, PIK3CA, loss of SMAD4 and TP53. Other forms of the disease, especially microsatellite-instable forms of colorectal cancer also harbor mutations of APC but show an almost exclusive prevalence of BRAF mutations instead of KRAS activations \cite{Guinney2015TheCancer.}.

\subsection{Wnt Signaling} % - cite own review and use figures
\subsection{MAPK Signaling}
\subsection{Other Pathways}
\section{Treatment of Colorectal Cancer}
\section{Intestinal Organoids}
Intestinal organoids are three-dimensional cell culture models from primary tissue. Organoids develop from Lgr5+ adult intestinal stem cells (Sato et al. 2011). Culturing these cells with specific growth factors in 3D leads to continuous organoid development in-vitro. Of note, organoids mimic their tissue of origin, including colorectal cancer (Pauli et al. 2017). Hence, organoids offer new possibilities for both research in cancer therapy and development. Due to a high isolation efficiency and preserved tumor biology, patient derived organoids are promising personalized cancer models (Van De Wetering et al. 2015). Moreover, organoids from healthy tissue can be cultured in-vitro as well and are amenable to genetic editing (Matano et al. 2015 Drost et al. (2015)). Therefore, these models also enable studying tumor development at a single mutation resolution.

A challenge for organoid research is the need for information-rich drug-testing methods. In the past, automated microscopy of 2D-cells has been used to measure the biological activity of compounds (Breinig et al. 2015). Prompted by this, we build a platform for high-throughput drug activity profiling in organoids. The platform uses confocal microscopy to collect fluorescent images of treated organoids in 3D. We devel- oped SCOPE, a software package, to process these images and measure organoid phenotypes. We used this platform to test compounds on both patient derived organoids and genetically engineered organoid models.

First, we isolated and characterized patient derived colorectal cancer organoids. Next, we performed high- throughput drug profiling of these organoids. Here, we observed a variety of recurring treatment-induced phenotypes. These were linked to specific cellular processes. Of note, the treatment response of organoids in-vitro matched the response of donating patients.

Second, we isolated colon organoids from healthy mouse tissue. By using gene-editing, we generated models of colon adenoma - a precursor lesion of colorectal cancer. These models carried different combinations of mutations in the Apc and Kras gene. Both are frequent and co-occurring mutations in colorectal cancer (Schell et al. 2016). To better understand the mechanisms of tumor development, we performed a multi- omics characterization in these adenoma models. Moreover, we profiled genotype specific drug effects. Here we found a reorganization of organoid phenotype after loss of Apc, which masks the effects of an isolated Kras mutation. Loss of Apc leads to genotype specific drug-induced phenotypes and vulnerabilities. Oncogenic Kras buffers a subset of these vulnerabilities, offering a new perspective on the relationship of Apc and Kras during tumor development.

\section{Image-based Phenotyping}
\section{Aims of Thesis}
\subsection{Image-based profiling of patient derived organoids identifies compound-induced phenotypes}
\subsection{Multi-omics profiling of intestinal organoids identifies an epistatic relationship of Apc loss and Kras activation during colorectal cancer development}
\end{flushleft}
