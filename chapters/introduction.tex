\chapter{Introduction}
\label{introduction}

\section{Colorectal Cancer}
Colorectal cancer is the third most common cancer worldwide and occurs at comparable rates in both men and women. Similar to other solid tumors, colorectal cancer progression is classified using the TNM 
% stages
% prevention 
% treatment
% need to study 
\botdevelopment h section{Cellular Signaling during Colorectal Cancer Development}
\subsection{Wnt Signaling} % - cite own review and use figures
\subsection{MAPK Signaling}
\subsection{Other Pathways}
\section{Intestinal Organoids}
Intestinal organoids are three-dimensional cell culture models from primary tissue. Organoids develop from Lgr5+ adult intestinal stem cells (Sato et al. 2011). Culturing these cells with specific growth factors in 3D leads to continuous organoid development in-vitro. Of note, organoids mimic their tissue of origin, including colorectal cancer (Pauli et al. 2017). Hence, organoids offer new possibilities for both research in cancer therapy and development. Due to a high isolation efficiency and preserved tumor biology, patient derived organoids are promising personalized cancer models (Van De Wetering et al. 2015). Moreover, organoids from healthy tissue can be cultured in-vitro as well and are amenable to genetic editing (Matano et al. 2015 Drost et al. (2015)). Therefore, these models also enable studying tumor development at a single mutation resolution.

A challenge for organoid research is the need for information-rich drug-testing methods. In the past, automated microscopy of 2D-cells has been used to measure the biological activity of compounds (Breinig et al. 2015). Prompted by this, we build a platform for high-throughput drug activity profiling in organoids. The platform uses confocal microscopy to collect fluorescent images of treated organoids in 3D. We devel- oped SCOPE, a software package, to process these images and measure organoid phenotypes. We used this platform to test compounds on both patient derived organoids and genetically engineered organoid models.

First, we isolated and characterized patient derived colorectal cancer organoids. Next, we performed high- throughput drug profiling of these organoids. Here, we observed a variety of recurring treatment-induced phenotypes. These were linked to specific cellular processes. Of note, the treatment response of organoids in-vitro matched the response of donating patients.

Second, we isolated colon organoids from healthy mouse tissue. By using gene-editing, we generated models of colon adenoma - a precursor lesion of colorectal cancer. These models carried different combinations of mutations in the Apc and Kras gene. Both are frequent and co-occurring mutations in colorectal cancer (Schell et al. 2016). To better understand the mechanisms of tumor development, we performed a multi- omics characterization in these adenoma models. Moreover, we profiled genotype specific drug effects. Here we found a reorganization of organoid phenotype after loss of Apc, which masks the effects of an isolated Kras mutation. Loss of Apc leads to genotype specific drug-induced phenotypes and vulnerabilities. Oncogenic Kras buffers a subset of these vulnerabilities, offering a new perspective on the relationship of Apc and Kras during tumor development.

\section{Image-based Phenotyping}
\section{Aims of Thesis}
\subsection{Image-based profiling of patient derived organoids identifies compound-induced phenotypes}
\subsection{Multi-omics profiling of intestinal organoids identifies an epistatic relationship of Apc loss and Kras activation during colorectal cancer development}

