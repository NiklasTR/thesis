\begin{savequote}[75mm]
Details matter, it’s worth waiting to get it right.
\qauthor{Steve Jobs}
\end{savequote}

\begin{flushleft}
\chapter{Materials and Methods}

\section{Disclosure}
Significant parts of this chapter have been adapted from own manuscripts, including \textit{The drug-induced phenotypic landscape of colorectal cancer organoids} \cite{Betge2022-kr}. The maximum contrast projection method, organoid segmentation method, feature extraction procedure and organoid viability classification (LDC) were previously developed by Jan Sauer as part of his dissertation \cite{noauthor_undated-ij}. Patient identification and Image-based profiling experiments were supported by Johannes Betge. 

\section{Patients}
All patients were identified at the University Hospital Mannheim, Mannheim, Germany. We included untreated patients with a new diagnosis of colon or rectal cancer in this study and obtained biopsies from their primary tumors and adjacent normal tissue via forceps based endoscopy. Exclusion criteria were active HIV, HBV or HCV infections. Biopsies were transported in phosphate buffered saline (PBS) on ice for subsequent organoid extraction. Clinical data, tumor characteristics and molecular tumor data were pseudonymized. The study was approved by the Medical Ethics Committee II of the Medical Faculty Mannheim, Heidelberg University (Reference no. 2014-633N-MA and 2016-607N-MA). All patients gave written informed consent before tumor biopsy was performed. In this study, we extracted PDOs from 25 patients with colorectal cancer, 10 of them female, 15 male, with a mean age of 66 years (median 65). 16 patients had a rectum carcinoma, 9 a colon carcinoma.

\section{Organoid Culture}

\subsection{Patient Derived Organoid Culture}
Patient derived organiod (PDO) cultures were extracted from biopsies as reported by Sato et al. \cite{Sato2011-lh} with slight modifications. Tissue fragments were washed in DPBS (Life technologies) and digested with Liberase TH (Roche) before embedding into BME R1 (Trevigen). The medium, termed ENA, contained Advanced DMEM/F12 (Life technologies) medium with 1\% v/v penicillin/streptomycin (Life Technologies), Glutamax and HEPES (basal medium) supplemented with 100 ng/ml Noggin (Peprotech), B27 (Life technologies), 1,25 mM n-Acetyl Cysteine (Sigma), 10 mM Nicotinamide (Sigma), 50 ng/ml human EGF (Peprotech), 10 nM Gastrin (Peprotech), 500 nM A83-01 (Biocat), 10 nM Prostaglandin E2 (Santa Cruz Biotechnology), 10 μM Y-27632 (Selleck chemicals) and 100 mg/ml Primocin (Invivogen). After isolation, cells were kept in 2 conditions including medium as described (ENA), or supplemented with additional 3 uM SB202190 (Biomol) (ENAS) as described by Fujii et al. \cite{Fujii2016ATumorigenesi  }. 
The tumor niche was determined after 14 days and organoids were subsequently cultured in the condition with best visible growth. 
Organoids were passaged every 7 days and medium was changed every 2-3 days.

\subsection{Mouse Organoid Culture}
A heterozygous LSL-Kras G12D (B6.129S4-Krastm4Tyj/J) female mouse was crossed with a homozygous Rosa26-CreERT2 (B6.129-Gt(ROSA)26Sortm1(cre/ERT2)Tyj/J) male to generate offspring with a Tamoxifen activatable KRAS G12D allele. A single healthy LSL-Kras G12D CreERT2 mouse (male, 8 weeks) was sacrificed for organoid generation. 

Mouse colon organoids were isolated based on work by Sato et al \cite{Sato2009-jw}. After cervical dislocation of the sacrificed mouse the colon was prepared and excised between caecum and rectum. The tissue was stored on ice in cooled DPBS (Life technologies), cut open lengthwise and washed three times with DPBS. After thorough washing, colon fragments were cut into 2mm pieces and incubated in a 5mM EDTA/DPBS (Sigma) solution for 60 minutes on a rocking table. Digested fragments were allowed to settle and resuspended in DMEM/F12 (Life technologies) by repeated up- and down-pipetting with a serological pipette. The resulting crypt suspension was filtered with a 70ul filter (Falcon), crypts were counted and centrifuged at 150g, 10min, 4C. The resulting pellet was resuspended in 10mg/ml Matrigel (Corning) and plated on prewarmed 6-well suspension plates (Greiner). After 30-60 minutes of solidification, droplets were overlaid with complete organoid growth medium and incubated at 37C, 5\% CO2 in atmospheric air.

Complete colon organoid medium, termed WENRAS, contained 30\% advanced DMEM/F12 (Life Technologies) supplemented with 1\% v/v penicillin/streptomycin solution (Life Technologies), 1\% v/v HEPES buffer (Life Technologies) and 1\% v/v Glutamax (Life Technologies), 50\% Wnt3A conditioned medium, and 20\% R-spondin1-FC conditioned medium. 
The medium was further supplemented with recombinant Noggin (100 ng/ml), 1x B27 (1x), n-Acetyl-cysteine (1.25 mM), Nicotinamide (10 mM), EGF (50 ng/ml), 500 nM A83-01 (Tocris), SB202190 (3 μM), Y-27632 (10 µM) and Primocin (100 µg/ml). All small molecule inhibitors were dissolved in DMSO. 

After isolation, colon organoids were cultured in solidified BME R1 (10mg/ml) droplets and overlaid with genotype and experiment dependent growth medium. The medium was exchanged every 48-72 h. 
APC mutant colon organoid lines were cultured without Wnt and R-spondin conditioned medium, which was replaced by basal medium instead.
Organoids were passaged weekly by digestion with TrypLE (Gibco) and resuspension in BME R1 (10mg/ml). 
Organoids were regularly tested for Mycoplasma contaminations.  

\subsection{Genetic editing of organoids}
An sgRNA targeting the murine ortholog of the APC mutation cluster region (MCR) was designed using E-CRISP(Heigwer et al., 2014). The Apc targeting sgRNA was cloned into the one-vector plasmid pSpCas9(BB)-2A-Puro (PX459) V2.0 according to Ran et al. \cite{Ran2013}. Briefly, the vector was digested with Bbs1-HF (Thermo Fischer Scientific) and the phosphorylated and annealed oligonucleotide sgApc1 (sgAPC1 F and -R) was ligated using T4-Ligase (Thermo Fischer Scientific). The construct was transformed into chemically competent bacterial cells (Stellar, Clontech) and plated on Carbenicillin agar. Individual colonies were isolated and sequencing of plasmid DNA from cultured colonies confirmed successful molecular cloning.   
Extracted organoids (termed “wildtype”, “WT”) were cultured for multiple passages before transfection of the plasmid with Lipofectamine 2000 (Thermo Fischer Scientific). Here, grown organoids were digested with TrypLE (Gibco) and treated with Lipofectamine and plasmid DNA according to the manufacturer’s protocol. Transfected organoids were seeded in BME R1 (10mg/ml) and Wnt3A/R-Spondin1-Fc withdrawal was started 7 days after transfection. Surviving organoids were cultured continuously without Wnt3A and R-Spondin1-Fc conditioned medium (termed “A”).
To activate oncogenic Kras, Wildtype and APC mutant organoid lines were treated for 7 days with 0.5uM 4-Hydroxytamoxifen (Sigma) without EGF in the medium. 4-Hydroxytamoxifen was dissolved in Ethanol. After treatment, organoids were cultured with EGF containing media thereafter (termed “K” or “AK”, respectively).

\section{Biochemical assays}

%% stopped citing

\subsection{Amplicon Sequencing of Patient derived Organoids}
DNA was isolated from 19 organoid cultures with the DNA blood and tissue kit (Qiagen). Sequencing libraries were prepared with a custom panel (Tru-Seq custom library kit, Illumina) according to the manufacturers protocol and sequenced on a MiSeq (Illumina). Targeted regions included the most commonly mutated hot spots in colorectal cancer in 46 genes captured with 157 amplicons of approximately 250bp length. After mapping of the reads to GRC38 reference genome using Burrows-Wheeler Aligner (BWA), data were analyzed using the Genome Analysis Toolkit (GATK) \cite{McKenna2010-cq}. Base recalibration was performed and variants were called using MuTect2 pipeline. Variants with a variant frequency below 10\%, with less than 10 reads, or with a high strand bias (FS<60) were filtered out. Variants were annotated with Ensemble variant effect predictor \cite{McLaren2016-dp} and manually checked and curated using integrative genomics viewer, if necessary \cite{Robinson2011-pc}. Only non-synonymous variants present in COSMIC \cite{Forbes2008-tk} were considered true somatic cancer mutations.
Also, all variants annotated “benign” according to PolyPhen database and “tolerated” in SIFT database were excluded, as well as variants with a high frequency in the general population as determined by a GnomAD \cite{Lek2016-dk} frequency of >0.001.

\subsection{Amplicon Sequencing of Mouse Organoids}
Amplicon sequencing was performed to validate the genetic perturbation of Apc. DNA from Apc targeted and untargeted organoid lines was prepared using the DNA Blood and Tissue Kit (Qiagen), according to the manufacturer’s tissue protocol including an RNAse digestion. The targeted region was PCR amplified using primers F1 to R2, and sequencing libraries were prepared according to the manufacturer’s protocol. Libraries were sequenced on a MySeq (Illumina) using 100bp single end reads. 

\subsection{Genomic PCR of the KRAS G12D allele in Mouse Organoids}
To confirm activation of oncogenic Kras in 4-Hydroxytamoxifen treated lines, genomic DNA was isolated from all 4 organoid lines as described above. Presence or absence of the Lox-STOP-Lox cassette was evaluated by PCR according to the Kras G12D conditional PCR protocol by Tyler Jacks’ group \cite{Jackson2001-wv}. Briefly, primers #2 and #3 were used for genotyping on genomic DNA using the Q5 PCR protocol (NEB).

\subsection{Western Blot of Patient derived Organoids}
Organoids seeded in 6-well plates were cultured in Matrigel (Corning). After 3-days incubation with WYE-132, organoids were collected, and cells were isolated using Matrisperse (Corning) for 40 minutes on a rocking table. Cells were subsequently lysed in RIPA buffer (Thermo Fisher Scientific) supplemented with protease inhibitors (Complete Mini, Roche) and phosphatase inhibitors (Phosphatase Inhibitor 1 and 2, Sigma), followed by sonication (Branson Sonifier, Heinemann). Protein concentrations of supernatants were measured using a BCA assay kit (Thermo Fisher Scientific). Lysates were mixed with an SDS-loading buffer and heated to 99 °C for 5 min. Proteins were separated by SDS–PAGE in MOPS running buffer and transferred to a nitrocellulose membrane. Membranes were blocked with 5\% (w/v) skim milk in PBS containing 0.1\% (v/v) Triton X-100 (PBS-T). Antibodies against IRS1 (06-248, Sigma–Aldrich) and HSP-90 (sc-13119, Santa Cruz) as loading control were used in 1:1000 dilution in 5\% milk in PBS-T, secondary antibodies (Mouse IgG HRP ECL, Sigma–Aldrich) were used in 1:10000. ECL Western Blotting W1001 (Promega) was used for visualization of bands.

\subsection{Western Blot of Mouse Organoids}
Organoids were cultured in Matrigel (Corning). Organoids were collected, and cells were isolated using Matrisperse (Corning) for 40 minutes on a rocking table. Isolated organoids were lysed in RIPA buffer (Sigma) with Protease inhibitor (Sigma) and Phosphatase inhibitor 3 (Sigma). Protein concentration was measured using the Pierce BCA kit (Thermo Fischer Scientific) according to the manufacturers protocol and samples were loaded onto NuPage gels (Thermo Fischer Scientific). Western Blotting was performed with following antibodies: anti-p(hospho)-Erk (1:2000), anti-Erk (1:1000) and anti-beta-actin-HRP secondary antibody (1:150,000).

\subsection{Mouse Organoid Growth Patterns}
Organoids were passaged and seeded in 4 different growth media with medium changes every 48h. Images were taken 120h after seeding.  

\subsection{RT-qPCR of Patient derived Organoids}
After 120h of growth, total RNA was isolated from organoids with the RNeasy Mini kit (Qiagen). cDNA synthesis was done with Verso cDNA kit (Thermo Fisher Scientific), and RT-PCR was performed using the SYBR Green Mix (Roche) on LightCycler480 system (Roche). The following primers for LGR5 were used: 5´-TTC CCA GGG AGT GGA TTC TAT-3′ (forward) and 5′-ACC AGA CTA TGC CTT TGG AAA C-3′ (reverse). Results were normalized to UBC mRNA using 5´-CTG ATC AGC AGA GGT TGA TCT TT-3´ forward and 5′-TCT GGA TGT TGT AGT CAG ACA GG-3′ reverse primers.

\subsection{RT-qPCR of Mouse Organoids}
Organoids were passaged and seeded in 4 different growth media with medium changes every 48h. After 120h, organoid RNA was isolated using the RNAEasy Kit (Qiagen) with beta-Mercaptoethanol (Invitrogen) and a DNAse digestion step. cDNA was synthesized using Oligo-dT primers (Thermo Fisher Scientific), RiboLock Ribonuclease inhibitor (Thermo Fisher Scientific) and Revert Aid H Minus reverse transcriptase (Thermo Fisher Scientific). RT-qPCR was performed using the ROCHE UPL kit (Roche), Sdha and Hprt expression levels were used as controls and averaged. Relative transcript abundance was measured using the ddCT method. RT-qPCR primers targeting Axin2, Ccnd, as well as the mouse reference genes Sdha and Hprt were used. 
%TODO qPCR primer

\subsection{Proteomics Profiling of Mouse Organoids}
Organoids were cultured according to the screening protocol (above). Organoids were isolated with Matrisperse (Corning) as described above. Isolated organoids were lysed in Ammonium Bicarbonate lysis buffer (50mM, pH 8.2) with 2.5\% w/v SDC and 25U/ml Benzonase. 
Samples were applied to 1D-SDS-PAGE and fractionated. Gel pieces were extracted, cysteins residues reduced by DTT and carbamidomethylated using iodoacetamide. The samples were digested with Trypsin overnight.
Resulting peptides were loaded on a cartridge trap column, packed with Acclaim PepMap300 C18, 5µm, 300Å wide pore (Thermo Fischer Scientific) and segregated in a 60 min gradient from 3\% to 40\% ACN on a nanoEase MZ Peptide analytical column (300Å, 1.7 µm, 75 µm x 200 mm, Waters). Eluted peptides were analyzed by an online coupled Q-Exactive-HF-X mass spectrometer.

\subsection{Lipidomics Profiling of Mouse Organoids}
%TODO missing

\subsection{Expression Profiling of Patient derived Organoids}
Organoid RNA was isolated from 19 PDO lines with the RNeasy mini kit after snap freezing organoids on dry ice. Samples were hybridized on Affymetrix U133 plus 2.0 arrays. Raw microarray data were normalized using the robust multi-array average (RMA) method \cite{Irizarry2003-vi} followed by quantile normalization as implemented in the ‘affy’ \cite{Gautier2004-jj} R/Bioconductor \cite{Huber2015-kc} package. In order to exclude the presence of batch effects in the data, principal component analysis and hierarchical clustering were applied. Consensus molecular subtypes were determined as described previously using the single sample CMS classification algorithm with default parameters as implemented in the R package ‘CMSclassifier’. In all cases, differential gene expression analyses were performed using a moderated t-test as implemented in the R/Bioconductor package ‘limma’ \cite{Ritchie2015-vf}. Gene set enrichment analyses were performed using ConsensusPathDB \cite{Kamburov2013-co} for discrete gene sets or GSEA as implemented in the ‘fgsea’ \cite{Korotkevich2021-xu} R/Bioconductor package for ranked gene lists.

\subsection{Expression Profiling of Mouse Organoids}
Mouse organoids were cultured according to a standardized screening protocol. Briefly, organoid models were seeded and cultured for 72h in WENRAS+Y and additional 96h in ENR-Y. Samples were harvested after 7 days and RNA was isolated using the RNAEasy Kit (Qiagen) as described above. Transcript expression levels were measured using MoGene-2\_0-st chips (Affymetrix).
Differential gene expression analyses were performed analogous to Patient derived Organoids.

\section{Image-based profiling}

\subsection{Patient derived organoid seeding during compound testing}
Patient dervied organoid drug profiling followed a standardized protocol with comprehensive documentation of all procedures. Organoids were collected and digested in TrypLE Express (Life technologies). Fragments were collected in basal medium with 300 U/ml DNAse and strained through a 40μm filter to achieve a homogeneous cell suspension with single cells and small clusters of cells, but without large organoid fragments. 384 well μclear assay plates (Greiner) were coated with 10μL BME V2 (Trevigen) at a concentration of 6.3 mg/ml in basal medium, centrifuged and incubated for >20 min at 300G and 37° C to allow solidification of the gel. Organoid cell clusters together with culture medium (ENA) and 0,8 mg/ml BME V2 were added in a volume of 50μl per well using a Multidrop dispenser (Thermo Fisher Scientific). Plates were sealed with a plate-loc (Agilent) and centrifuged for additional 20 min allowing cells to settle on the pre-dispensed gel. Cell number was normalized before seeding by measuring ATP levels in a 1:2 dilution series of digested organoids with CellTiter-Glo (Promega). The number of cells matching 10,000 photons was seeded in each well. After seeding of organoid fragments, plates were incubated for three days at 37°C to allow organoid formation before addition of compounds. Two biological replicates of each organoid line from different passages were profiled at different time points.

\subsection{Mouse Organoid seeding during compound testing}
Mouse organoid screening was performed as described above with slight modifications. Clotting of organoid fragments was avoided by adding 10 U/ml of bovine DNAse1 to the medium during filtration. The cell viability of digested fragment suspensions was estimated using Cell-Titer-Glo (Promega). 40ul of cell suspension was mixed with 40ul of undiluted reagent and measured after 30 minutes on a Mithras plate reader (Berthold). Cell fragments with a viability corresponding to 5000 photons were seeded per well on pre-coated 384 well plates using a Multidrop peristaltic pump robot (Thermo Fischer Scientific). 

\subsection{Compound Libraries for Patient derived Organoids}
Two compound libraries were used for profiling: A library containing 63 clinically relevant drugs (clinical cancer library) and a large library of 464 compounds targeting kinases and stem cell or developmental pathways associated genes (KiStem library). The clinical cancer library was manually curated by relevance for current (colorectal) cancer therapy, mechanism of action and potential clinical applicability. Compounds of this library are in clinical use or at least in phase I/II clinical trials. Five concentrations per compound were screened (five-fold dilutions). The concentrations were determined by analysis of literature data from previous 3D and 2D drug screens and own experiments. The KiStem library includes 464 compounds targeting a diverse set of kinases and stem cell relevant pathways. All compounds in this library were used in a concentration of 7.5μM. All compounds were obtained from Selleck chemicals. Compounds of both libraries were arranged in an optimized random layout. We stored compound libraries in DMSO at -80 C.

\subsection{Compound Libraries for Mouse Organoids}
Two compound libraries were used for profiling: The KiStem library (see above) as well as a library of FDA-approved small molecules. For both the KiStem library and the FDA-approved library, one single concentration was used. All compounds in this library were used in a concentration of 7.5μM. All compounds were obtained from Selleck chemicals. Compounds of both libraries were arranged in an optimized random layout. We stored compound libraries in DMSO at -80 C.

\subsection{Compound Treatment of Patient derived Organoids}
After 72 hours, medium was aspirated from all screening plates and replaced with fresh ENA medium devoid of Y-27632, resulting in 45μl volume per well. Drug libraries were diluted in basal medium and subsequently 5μl of each compound was distributed to screening plates. 
All liquid handling steps were performed using a Biomek FX robotic system (Beckmann Coulter). Plates were sealed and incubated with the compounds for four days. All PDO lines underwent profiling with the clinical cancer library, while the KiStem library was used with 13 PDO lines.

\subsection{Compound Treatment of Mouse Organoids}
Mouse Organoids were treated similar to Patient derived Organoids with slight modifications: After 72 hours of organoid expansion in WENRAS+Y, the medium was changed to ENR-Y and compound libraries were added using a BiomekFX (Beckmann Coulter). All Mouse Organoids were profiled with the KiStem library and FDA-approved compound library.

\subsection{Luminescence Viability Read Out of Patient Derived Organoids}
Plates undergoing viability screening were treated with 30μl CellTiter-Glo reagent after medium aspiration with a Biomek FX. After incubation for 30 minutes, luminescence levels were measured with a Mithras reader (Berthold technologies).

Raw luminescence data of each plate were first normalized using the Loess-fit method in order to correct for edge effects where increased luminescence intensity was observed along the edges of each plate. Subsequently, each plate was normalized by division with the median luminescence intensity of the DMSO controls. Drug response Hill curves (DRC) were fitted and area under the curve values were calculated for each DRC using the ‘PharmacoGx’ \cite{Smirnov2016-ah} R/Bioconductor package.

\subsection{Luminescence Viability Read Out of Patient Derived Organoids of Mouse Organoids}
For selected compounds, before compound addition, organoid viability was measured using Cell-Titer-Glo (Promega) and WENR-Y was added to the remaining plates. Cell viability was measured after compound exposure as described above. The pre-treatment viability of organoids was used to estimate growth-rate controlled dose-response curves according to \cite{Hafner2016-yr} Measurements of GR metrics for was not robust for slow proliferating WT lines. Therefore, these lines were omitted in the analysis.

\subsection{Microscopy}
Image-IT DeadGreen (Thermo Fisher) was added to the cultures with a Multidrop dispenser (Thermo Fisher) in 100nM final concentration and incubated for four hours. Afterwards, medium was removed and organoid cultures were fixed with 3\% PFA in PBS with 1\% BSA. Fixed plates were stored at 4° C for up to three days before permeabilization and staining. On the day of imaging, organoids were permeabilized with 0.3\% Triton-X-100 and 0.05\% Tween in PBS with 1\% BSA and stained with 0.1μg/ml TRITC-Phalloidin (Sigma) and 2μg/ml DAPI (Sigma). All liquid handling steps were performed with a BiomekFX. Screening plates were imaged with an Incell Analyzer 6000 (GE Healthcare) line-scanning confocal fluorescent microscope. We acquired 4 fields per well with z-stacks of 16 slices at 10x magnification. The z-steps between the 16 slices had a distance of 5μm, the depth of field of each slice was 3.9μm.

\section{Image analysis}

\subsection{Image Processing}
Microscopic image z-stacks were compressed to HDF5 format for archival and underwent maximum contrast projection using the R/Bioconductor package MaxContrastProjection developed by Jan Sauer for further processing of the images. Standard image features, including shape, moment, intensity, and Haralick texture features on multiple scales, were extracted using the R/Bioconductor package EBImage \cite{Pau2010-gg}. Of note, the strong diversity of unperturbed organoid phenotypes between organoid lines did not allow the definition of a core set of individual reproducible descriptive features across all screened organoids. Therefore, no correlation-based filtering of features was done, allowing comparisons between different lines. Out-of-focus objects were programmatically and manually removed from the dataset.

\subsection{Drug-Induced Phenotypes}
A principal component analysis (PCA) was performed on the entire datasets (patient derived and mouse organoids seperately) to reduce the dimensionality. 25 principal components were selected, explaining approx. 81\% of the total variance within the dataset. A logistic regression classifier was trained per line and treatment (and per concentration where applicable) to differentiate treated organoids from negative controls based on the PCA-transformed features. To allow comparison between various organoid lines and drug perturbations, the distributions of features describing organoids from different batches were adjusted. Drugs were categorized as either active or inactive based on the accuracy of the model. The direction of the vector perpendicular to the decision plane was interpreted as the treatment effect vector. Drugs were clustered with regard to the angles of their corresponding effect vectors in PCA-feature space.

\subsection{Live-Dead Classification}
A random forest classifier was trained by Jan Sauer on the original single organoid features to differentiate living from dead organoids. Organoids treated with DMSO were used as negative (i.e. living) controls while organoids treated with Bortezomib and SN-38 at the two highest concentrations were used as positive (i.e. dead) controls. Visual inspection of the projected images confirmed the choice of positive controls. Binary classification results were averaged within wells to obtain viability scores ranging from 0 to 1, indicating how lethal a treatment was. A separate classifier was trained for each individual line to ensure inter-line independence.


\section{Multi-omics factor analysis}
\subsection{Model training of Patient Derived Organoids}
A multi-omics factor analysis model \cite{Argelaguet2018-mz} was trained based on a set of five modalities describing unperturbed organoid lines:

* organoid size estimated based on log-normal model fit of all DMSO treated organoids [22 replicates, 1 feature]
* organoid somatic mutations as determined by amplicon sequencing [20 replicates, 12 features]
* organoid gene expression including the top 10\% genes with the highest coefficient of variance after robust multi-array average normalization [22 replicates, 3222 features]
* organoid morphology as determined by averaging DMSO treated morphological profiles [22 replicates, 25 features]
* organoid drug activity as determined by AUROC score of logistic regression models for drugs that were defined as active in at least one observation [22 replicates, 252 features]

Input data was scaled and the MOFA model was trained with default MOFA2 training parameters and a number of 3 factors. The number of factors was chosen given the limited number of observations in the training data. The further analysis focused on the first two factors, which correlated with prominent visible organoid phenotypes. Gene set enrichment analysis and Reactome pathway enrichment of factor loadings was performed using the clusterprofiler R package. Enrichment of drug targets within factor loadings was tested using ANOVA by fitting a linear model, lm(factor loading ~ target). Drug targets that were represented with at least three small molecule inhibitors were included in this analysis. The analysis was run using the MOFA docker container available from https://hub.docker.com/r/gtca/mofa2.

\subsection{Model training of Mouse Organoids}
A multi-omics factor analysis model with k=4 factors was trained and results were analysed as above, with a different set of input data: 
* organoid size all DMSO treated organoids (one replicate of Apc-/- organoids was removed from the analysis) [7 replicates, 1 feature]
* organoid transcript expression including the top 10\% genes with the highest coefficient of variance after robust multi-array average normalization [8 replicates, 2727 features]
* organoid protein expression [12 replicates, 3906 features]
* organoid lipid abundance [12 replicates, 397 features]
* organoid genotype for the Apc and KrasG12 allele [12 replicates, 2 features]
* organoid morphology as determined by averaging DMSO treated morphological profiles (one replicate of Apc-/- organoids was removed from the analysis) [7 replicates, 25 features]
* organoid drug activity as determined by AUROC score of logistic regression models for drugs that were defined as active in at least one observation [7 replicates, 1699 features]

\subsection{Model projection}
To estimate the factor scores for drug-induced organoid morphologies, the morphology profiles of organoids treated with the same drug were averaged. The resulting average profile matrix was multipled with the pseudoinverse of the previously learnt model loading matrix for organoid morphology data. 

\smallbreak
The resulting projected factor score matrix was used to estimate the drug-induced biological changes in both patient derived and mouse organoids. Associations between drug targets and projected factor scores of drug treated organoids were identified via ANOVA by fitting a linear model, lm(projected factor score ~ target). Drug targets that were represented with at least three small molecule inhibitors were included in this analysis.

\end{flushleft}