\begin{savequote}[75mm]
Nulla facilisi. In vel sem. Morbi id urna in diam dignissim feugiat. Proin molestie tortor eu velit. Aliquam erat volutpat. Nullam ultrices, diam tempus vulputate egestas, eros pede varius leo.
\qauthor{Quoteauthor Lastname}
\end{savequote}

\chapter{Supervised learning of patient derived organoid phenotypes}

\section{Supervised identification of lethal and non-lethal patient derived organoid phenotypes}

We demonstrated that morphological profiles of PDOs can be used to identify phenotypic subsets. To analyze compound induced morphological phenotypes, we first aimed to identify viability effects and the corresponding morphological profiles using a supervised approach. We trained random forest live-dead classifiers (LDC) using organoid profiles from negative (DMSO) and positive controls (high-dose bortezomib and sn-38), based on single-organoid features generated with the SCOPE framework for every PDO line (Figure 3a). We analyzed receiver-operating characteristics (ROC) for all classifiers, revealing robust classification performance when applied to independent validation sets of positive and negative controls from the same PDO lines they were initially trained on (area under receiver operating characteristic curve (AUROC): 0.97 – 0.99, Figure 3b). Applying the classifiers to different PDO lines (which they were not initially trained on) also led to high classification performance in most PDO-line-classifier pairs (Figure 3c).
Further analyses showed reproducibility of LDC between biological replicates (Pearson correlation = 0.85 for the whole dataset and 0.67 - 0.93 for individual PDO lines, Supplementary Figure S3a-b) and a high fraction (median = 0.98) of PDOs correctly classified as “viable” in DMSO controls (Supplemental Figure S3c). By testing if a classifier relying on less information (i.e. fewer channels and fewer dyes) would result in similar accuracy, we found that classifiers relying on a combination of actin/TRITC and DNA/DAPI staining alone (mean accuracy 0.958) led to accuracies almost as high as the ones including cell permeability/FITC (mean accuracy 0.968, Supplemental Figure S3d). We also compared the viability prediction of our high-throughput imaging and LDC pipeline with a luminescence based metabolic (ATP-based) viability assay. Similar assays have previously been used{vandeWetering:2015is, Schutte:2017kp}. We used our clinical cancer library for these analyses and performed viability profiling with CellTiter-Glo (CTG) and high-throughput-imaging with LDC in parallel (Supplemental Figure S4a-d and S5a). The correlation between the results obtained with both viability read-outs was high (Pearson correlation = 0.87, Supplemental Figure S5b), proving the validity of our workflow. Nevertheless, we also identified several examples with divergent results, such as the anti-folate drug methotrexate (Supplemental Figure S5c-g).
In conclusion, we used morphological profiling to robustly identify compounds which induced a phenotype of interest (dead organoids vs. live organoids) in high-throughput imaging data.

\section{Drug susceptibilities of PDOs are associated with molecular characteristics}

We next used our live-dead classifier on the datasets of PDO lines profiled with the large experimental compound library (KiStem library, 464 compounds, Figure 4a) and the focused clinical library (63 compounds in five concentrations, Figure 4b and Supplemental Figure S6a-b). In both datasets, we found that several compounds induced heterogeneous viability responses, killing subsets of PDO lines (e.g. EGFR-inhibitors, MEK-inhibitors or PLK-inhibitors in the KiStem dataset, Figure 4a). In the PDOs treated with the clinical compound library, more than half of the compounds led to strong and/or heterogeneous responses, as calculated by comparing areas under the dose response curve (AUCs). While some lines were generally resistant to the majority of compounds (e.g. D007T), others were susceptible to several cytotoxic drugs including docetaxel or vinblastine and targeted therapies like erlotinib or afatinib (e.g. D021T, D020T, D027T). The majority of PDO lines responded to at least one of the tested drugs (Figure 4b, Supplemental Figure S6a,b). 
Next, we analyzed associations of drug response or resistance towards targeted therapies with molecular characteristics of PDOs (Figure 4c-f, Supplemental Figure S6d-f). In the KiStem dataset, PDO lines that showed decreased viability upon EGFR inhibition had wild-type RAS alleles, while resistant organoid lines carried RAS mutations (including KRAS or NRAS, Figure 4c), which is in accordance with the clinical association of EGFR-antibody resistance and RAS mutations in colorectal cancer. Of note, only a subset of RAS mutant PDO lines were strong responders to MEK inhibition. Another example (from our clinical dataset) is response of PDO lines to MDM2 inhibitor nutlin3a, (Figure 4d, Supplemental Figure S6d), which was significantly associated with TP53 mutational status (Figure 4d, Supplemental Figure S6e) and relative enrichment of gene expression associated with p53 signaling (Reactome R-HSA-5633008; q = 0.004; Figure 7e, Supplemental Figure S6f){Croft:2014iq}.
Several other targeted therapies and conventional chemotherapeutics not currently used for colorectal cancer therapy showed differential response profiles among PDO lines that could not be explained by the assessed genomic alterations or expression subtypes (Figure 4f-h, Supplemental Figure S6g-i). These findings suggest that functional drug testing with PDOs might help to develop novel therapeutic strategies and identify predictive markers ex vivo.

\section{Individual PDO Drug Susceptibility Correlates with Clinical Treatment Response}

We have shown that our drug-profiling platform can identify individual susceptibilities of PDOs, but an important question remains whether the drug effects observed ex vivo also correlate with similar drug effects in the clinical setting. A patient with rectal cancer and diffuse hepatic metastases was planned for palliative combination therapy and organoid cultures were established prior to treatment. After six cycles of treatment with 5-FU (400 mg/m2 bolus and 2400 mg/m2 infusion over 46 h), leucovorin (400 mg/m2), irinotecan (180 mg/m2) (FOLFIRI) and the anti EGFR-antibody panitumumab (6 mg/kg) every second week, CT imaging (week 10) showed a strong partial response with significantly smaller and partly necrotic hepatic metastases (Figure 7A, B). In the same week, we had completed two biological replicates of drug screening with PDOs from this patient (Figure 6A), which showed exceptionally strong response to EGFR inhibitors (Figure 7C, gefitinib and erlotinib shown). The remission of the patients disease was durable as seen in CT scans after 23 weeks when the patient had completed 12 cycles of treatment with anti-EGFR therapy (Figure 7B) and after 32 weeks. The patient is to date (week 35) undergoing maintenance treatment with 5-FU and panitumumab with no evidence of progression and good quality of life. This data demonstrates that the susceptibility of this metastasized primary cancer to EGFR targeted therapy can be precisely modeled ex vivo by the corresponding PDO line.
As another example, six other patients from our cohort with rectal cancer without distant metastases underwent neoadjuvant chemoradiation before surgical resection of their primary tumors. All of those patients received the oral 5-FU prodrug capecitabine (1.650 mg/m2 every day) together with radiation (28 x 1.8 Gy) over a period of five weeks. We had established PDO cultures of those tumors before any anti-cancer treatment (Figure 7D) and again, performed drug screenings in parallel to the patients’ treatment. After tumor resection, two patients showed only slight tumor-regression in their surgical explants (Dworak grade I,
15
Figure 7F), while the four other cases had moderate to strong regression score (Dworak grade II-IV, Figure 7F). This trend corresponded to differences in 5-FU susceptibility of PDO cultures belonging to the same patients (Figure 7E).
Together, these data illustrate that drug effects measured by ex vivo drug screening using PDOs can recapitulate drug responses seen in human cancer from diverse clinical stages and that the workflow of PDO culture establishment plus drug screening can be performed in a clinically meaningful timeframe

%Anecdotal in nature