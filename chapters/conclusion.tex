\begin{savequote}[75mm]
I learned that if you work hard and creatively, you can have just about anything you want, but not everything you want. Maturity is the ability to reject good alternatives in order to pursue even better ones.
\qauthor{Ray Dalio}

\end{savequote}

\chapter{Discussion}

factor 1
Wnt-STOP






DMSO
PDO

CTG
LDC

Oncogene induced senescence 
TP53 dependent 



kras oncogene induced senescence
next to mice p16 or TP53 deletion avoid cell state
in zebrafish myc signaling avoids cell state
Myc is a master regulator after loss of Apc 
% cite japenese Myc paper
Myc went up after loss of Apc 
even further up after combination of Apc Kras 
but down after Kras

switch from cholesterol esters to TAG 

		* in fact downregulation of [[beta oxidation]] seen in all 3 mutant organoid lines
		* in [[KRAS]] also reduction of [[citric acid cycle]] and [[respiratory electron transport]] -> [[warburg effect]]
		* accumulation of [[triacylglycerol TG]]
			* [[long chain fatty acids]] - also desaturated as in lizardo
			* [[cholesterol ester CE]] and [[phosphatidyl choline PC]] depletion - previously described in adherent cell lines [[lizardo2017.pdf]]
			* similar bias toward [[triacylglycerol TG]] has also been observed in [[zebrafish]] models with Kras solo mutation


	* bias away from [[glycerophospholipid]] also observed in [[zebrafish]]
% cite Serrano et al 



Myc 
just like reprogramming
high canonical Wnt signaling can rejuvenate organoid models
interferon response as mediator of oncogene induced senescence
%TODO cite https://pubmed.ncbi.nlm.nih.gov/27052162/

oncogene induced senescence
%TODO https://www.nature.com/articles/1210950

 why no KRAS alone? - [[oncogene induced senescence]]
 % * the already come with better colony formation
* why no APC alone? - proliferative advantage? cell cycle? 
	* weakness of this method is presence of [[EGF]] ligand

Sakai 2017
small intestinal organoids
more polyps in AK vs. A! (tumorigenicity in-vitro - other papers)and in-vivo)
no invasion in AK/AKF
K necessary but not sufficient for invasion/metastasis

no differences in proliferation A vs. AK 
they found them to be similar - confirmed with our data

AKT program
AK is the most aggresive combo - mouse in-vivo studies - it is the mostactive form

outlook 
mouse genetics - coarse phenotypes  - Sakai
organoid - high genetic control - Matano, PI3KCA + APC paper
image based profiling coupled with organoids - Rindtorff
identify phase space of cancer development in vitro and ways to therapeutically target it

ellagic acid

## limitations
media condition overemphasize effect of Apc and potentially mask effect of oncogenic Kras
tradeoff between maximizing the phenotype state space while maintaining viable cells 





Modern Drug Discovery is a sequential decision making problem. To motivate this statement, specifically the focus on *decision making* and its *sequential* nature, let's introduce a theoretical scenario, which we can call "the average treatment effect oracle". In its idealised form the oracle is a matrix that has a row for every synthesisable pharmacologically active molecule and a column for every human disease. Each entry of our matrix represents the clinical treatment effect (i.e. in QUALYs) experienced by patients with disease n that have been treated with molecule t. If we restricted this matrix to small druglike molecules only, the conservative estimate for this table's dimensions would still be very large: 10E60 rows t x 10E3 columns n. Unfortunately, most of this table is not known to us. Given the large size of chemical space and constrained resources, an agent within a Drug Discovery project has to make decisions about which molecules (rows) to run experiments with and on which human disease (columns) to focus on at any given time in order to identify a positive treatment effect. Staying within our theoretical scenario, one simple but problematic approach to tackle the treatment effect oracle problem would be a single-point decision making process in which an agent, equipped with all currently available knowledge, selected promising small molecules and directly performed clinical experiments to observe the treatment effect on patients. For obvious ethical and financial reasons, such a single-point decision process is not a viable solution. What follows is that Drug Discovery must be a sequential decision making process. In order to identify a new positive treatment effect, agents need to choose surrogate models, "placeholders", to perform experiments on. Ideally such surrogate models (1) approximate the human disease of interest as close as possible, (2) can be observed in a way that is predictive of the desired clinical outcome, and (3) create low ethical and financial burden. The iterative process of making decisions on which disease to focus on, which correspopnding surrogate models to use, and which molecules to test on the selected models in a given experiment are at the core of modern Drug Discovery. \par