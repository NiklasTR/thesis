\begin{savequote}[75mm]
I learned that if you work hard and creatively, you can have just about anything you want, but not everything you want. Maturity is the ability to reject good alternatives in order to pursue even better ones.
\qauthor{Ray Dalio}

\end{savequote}

\chapter{Discussion}

factor 1
Wnt-STOP

\label{conclusion}

DMSO
PDO

CTG
LDC

Oncogene induced senescence 
TP53 dependent 



kras oncogene induced senescence
next to mice p16 or TP53 deletion avoid cell state
in zebrafish myc signaling avoids cell state
Myc is a master regulator after loss of Apc 
% cite japenese Myc paper
Myc went up after loss of Apc 
even further up after combination of Apc Kras 
but down after Kras

switch from cholesterol esters to TAG 

		* in fact downregulation of [[beta oxidation]] seen in all 3 mutant organoid lines
		* in [[KRAS]] also reduction of [[citric acid cycle]] and [[respiratory electron transport]] -> [[warburg effect]]
		* accumulation of [[triacylglycerol TG]]
			* [[long chain fatty acids]] - also desaturated as in lizardo
			* [[cholesterol ester CE]] and [[phosphatidyl choline PC]] depletion - previously described in adherent cell lines [[lizardo2017.pdf]]
			* similar bias toward [[triacylglycerol TG]] has also been observed in [[zebrafish]] models with Kras solo mutation


	* bias away from [[glycerophospholipid]] also observed in [[zebrafish]]
% cite Serrano et al 



Myc 
just like reprogramming
high canonical Wnt signaling can rejuvenate organoid models
interferon response as mediator of oncogene induced senescence
%TODO cite https://pubmed.ncbi.nlm.nih.gov/27052162/

oncogene induced senescence
%TODO https://www.nature.com/articles/1210950

 why no KRAS alone? - [[oncogene induced senescence]]
 % * the already come with better colony formation
* why no APC alone? - proliferative advantage? cell cycle? 
	* weakness of this method is presence of [[EGF]] ligand

Sakai 2017
small intestinal organoids
more polyps in AK vs. A! (tumorigenicity in-vitro - other papers)and in-vivo)
no invasion in AK/AKF
K necessary but not sufficient for invasion/metastasis

no differences in proliferation A vs. AK 
they found them to be similar - confirmed with our data

AKT program
AK is the most aggresive combo - mouse in-vivo studies - it is the mostactive form

outlook 
mouse genetics - coarse phenotypes  - Sakai
organoid - high genetic control - Matano, PI3KCA + APC paper
image based profiling coupled with organoids - Rindtorff
identify phase space of cancer development in vitro and ways to therapeutically target it

ellagic acid

## limitations
media condition overemphasize effect of Apc and potentially mask effect of oncogenic Kras
tradeoff between maximizing the phenotype state space while maintaining viable cells 