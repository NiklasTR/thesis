\begin{savequote}[75mm]
I learned that if you work hard and creatively, you can have just about anything you want, but not everything you want. Maturity is the ability to reject good alternatives in order to pursue even better ones.
\qauthor{Ray Dalio}

\end{savequote}

\chapter{Discussion}


\section{image-based profiling of organoids for therapeutic discovery}

\subsection{Organoids as high validity in-vitro models for therapeutic discovery}
Organoids are representative in vitro models for diverse human tissues and can be used for image-based small molecule profiling. While the prospective use of organoids as a diagnostic is currently limited by high sample dropout and [[tunraround time]], a high overall predictive validity for multiple therapeutic regimens has been reported. With the use of organoids in a diagnostic context is limited by these factors, the use of organoid models in early stage drug disocovery is not. In the context of therapeutic discovery, previous studies have success- fully used organoids to perform medium-scale small molecule treatment assays. The most commonly used method is screening with an ATP based cell viability readouts 7,9–15,47–49. Additionally, imaging studies with organoids have been used to characterize developmental processes such as the self-organization of intestinal cells25,50 or the morphological response to individual drugs24,51. While image-based profiling of in vitro models has become an important tool for the analysis of biological processes, particularly in drug discovery and functional genomics17–19, performing such high-content experiments in disease models that cannot be cul- tured and perturbed in 2D, has been a technological challenge. 
Primary challenges: high heterogeneity of morphology, technical challenge, interest in including multiple organoid lines into the experiment
How they were solved: 
sparse 3D imaging and projection, mulit view representation learning to interpret morphological differences
Introducing these changes enabled the use of small molecule image-based profiling of patient-derived and genetically engineered organoid disease models. 
\par

High validity in-vitro models, such as organoids, are an essential component of the therapeutic discovery process. Therapeutic discvory is a sequential decision making process in which a selected small moelcule can not, for ethical and practical reasons, be tested directly in a clinical context to observe its treatment effect. Instead, therapeutic candidates have to be evaluated using one or more models to approximate the treatment effect that could be observed in large clinical trials. In the context of optimisation theory, these models can be considered "surrogate models", while the large clinical trial is an "oracle". In practice, multiple in-silico, in-vitro and in-vivo surrogate models are combined and used sequentially to guide decision making during a discovery program. The higher the predictive validity and the lower the (ethical and financial) cost of using a [[surrogate model]], the more value does it provide to the therapeutic discovery process. While determining the validity of an in-vitro model is an empirical process, a set of axioms to prioritise models have been formulated: 

(1) the model must have a clear link to the disease of interest (i.e. matched tissue of origin, representative culture conditions) and, if engineered, model the disease state based on the best understanding of the disease pathophysiology
(2) the phenotype observed during the experiment should represent the desired clinical endpoint (i.e. overall tumor regression) and ideally capture a high degree of information
(3) the treatment schedule and overall experimental design should be aligned closely with assays that have been successfully in clinical a diagnostic context

Designing or "fitting" in-vitro biological models to a disease that align with these axioms is a cost intensive process, but already minor changes in predictive validity of an experiment can offset these cost in a drug discovery program. In fact, economic research into the overall cost of therapeutic discovery by Scannel et al. concludes with the statement:

"The rate of creation of valid screening and disease models may be the major constraint on R&D productivity".  

\subsection{Learning meaningful representations for image-based profiling experiments}

\subsection{}
