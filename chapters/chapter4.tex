\begin{savequote}[75mm]
Nulla facilisi. In vel sem. Morbi id urna in diam dignissim feugiat. Proin molestie tortor eu velit. Aliquam erat volutpat. Nullam ultrices, diam tempus vulputate egestas, eros pede varius leo.
\qauthor{Quoteauthor Lastname}
\end{savequote}

% not yet modified

\chapter{Unsupervised learning of patient derived organoid phenotypes}

\section{Morphological diversity of unperturbed organoids}

PDOs showed significant morphological diversity between lines derived from different donors. Therefore we first aimed to systematically assess the diversity of phenotypes between different PDO lines in their untreated state. We computed an automated morphological classification of PDO lines using the SCOPE pipeline. To this end, we aggregated features across all PDO-lines and reduced their dimensionality by principal component analysis (PCA, Figure 2d, Supplemental Figure S2). After hierarchical clustering of morphological profiles by organoid line, we identified six distinguishable phenotypic subsets of colorectal cancer organoids (Figure 2d). By visual inspection, we observed that these phenotypic subsets were characterized by a different degree of of intra-organoid organization ranging from disorganized (irregular shaped, compact or solid looking organoids) to more “organized” (round, regular shaped organoids with a smooth wall and a defined large lumen / cystic morphology). “Intermediate” PDOs had features of both categories. (Figure 2e). Based on this categorization, we investigated molecular differences between the identified subgroups. Morphological organoid subtypes did not match an unsupervised clustering of expression profiles. Also, we did not observe associations with cancer mutations analyzed or clinical variables. However, gene set enrichment analysis (GSEA) for stem cell signatures{MerlosSuarez:2011cv} revealed upregulation of Lgr5+ stem cell signature-related genes within the group of organized (cystic) PDOs (blue) when compared to intermediate and disorganized (compact) PDOs (FDR=8.8E-3, NES=1.67). In contrast, the disorganized and intermediate PDO groups were enriched in signatures related to cell proliferation (FDR=6.7E-4, NES=1.93, Supplemental Figure S1b). We conclude that heterogeneous PDO phenotypes can be grouped into a limited set of morphological classes that are associated with defined molecular differences. 

\section{Image-based phenotyping of organoids identifies compound mode-of-action}
% semi-supervised

Beyond viability classification, we evaluated compound response of PDOs using an unsupervised learning  approach. Diverse morphological responses to compound treatments have previously been reported for 2D monolayer cell culture models{Carpenter:2007is, Boutros:2015ch, Perlman:2004kr}. Therefore, we hypothesized that complex phenotypic responses to small molecule treatments can be measured in heterogenous and multicellular patient derived organoids..

We applied a support vector machine (SVM) based approach on single organoid level data to describe compound induced morphological changes. Briefly, organoid profiles were measured after treatment with the Ki-Stem library of 464 compounds targeting developmental and signaling pathways (Figure 5a). We trained SVM classifiers to separate perturbed from unperturbed organoid profiles for every PDO-line and compound using PCA-transformed single organoid features. Next, we selected active compound treatments in which robust morphological changes were observed in at least one line. We defined a compound treatment as “active” when treated and untreated organoids in a holdout dataset could be correctly identified by their corresponding SVM leading to an area under the receiver operating characteristic curve (AUROC) of 0.85 or greater (Figure 5b). Using this approach, we identified between 92-192 active compounds across tested PDO lines (Supplemental Figure S7a).
After filtering for active treatments across all PDO lines, we grouped induced phenotypes by determining the anglecosine distance between unit vectors of their respective SVM-hyperplanes, so that treatments leading to similar drug induced phenotypes (i.e. with a small angle between SVM vectors) were grouped together. We found that PDO phenotypes induced by compound treatments with shared mechanisms of action had a high degree of similarity. Specifically, aggregating compound induced phenotypic profiles across all PDO lines and applying contingency testing{Freudenberg:2009jw} showed a strong enrichment of specific mode-of-actions of many different compound classes (Figure 5c, Supplemental Figure S7b-i). Furthermore, we observed clustering of compounds with related targets, for instance belonging to the same signaling pathway. To this end, the MEK inhibitors clustered with a RAF inhibitor (RAF265) and an ERK inhibitor (ulixertinib), both targeting downstream components of MAPK signaling. The mTOR inhibitor cluster included a PI3K inhibitor (VS5584) and was part of a larger meta-cluster of AKT/mTOR/PI3K inhibitors (Supplemental Figure S7). The aggregation also suggested additional mode-of-actions or off-target effects of several other drugs. Specifically, protein kinase c (PKC) inhibitor enzastaurin was found to be related to glycogen synthase kinase 3 (GSK3) inhibitors. This is in accordance with previous data, showing that besides its primary target PKC, enzastaurin can target the alpha and beta subunits of GSK {Kotliarova:2008ec}. As further examples, VEGF inhibitor ZM 306416 and two BCR-ABL/SRC tyrosine kinase inhibitors (imatinib and saracatinib) co-localized with EGFR inhibitors. While ZM 306416 has previously been shown to inhibit EGFR {Antczak:2012fr}, the effect of both BCR-ABL TKIs on EGFR has not been studied in detail. Finally, the cluster enriched for cyclin dependent kinase (CDK) inhibitors also featured two focal adhesion kinase (FAK) inhibitors and one p12 activated kinase (PAK) inhibitor. All three of these targets are closely related to the cell cycle and apoptosis{Aleem:2015ka, Hao:2009fz}. Accordingly, we hypothesized that the identified phenotypic cluster would be related to cell cycle arrest or apoptosis.
To analyze the relationship of multiparametric compound induced morphological profiles with PDO viability, we applied our LDC to the dataset. We arranged the viability measurement in the same order as the clustering of multiparametric PDO phenotypes generated with the SVM approach (Figure 5d) to visualize associations between enriched clusters and viability. As expected, this revealed the CDKi cluster to be related to cell death. Alongside, a few other drug induced phenotypic clusters, including those enriched for ATM inhibition, JAK inhibition or PLK inhibition represented dead PDOs. Consequently, these compounds aggregated in a meta-cluster defined by reduced viability in all or most PDO lines. Importantly, however, many other compound induced phenotypes, including those caused by PKC inhibitors, PI3K inhibitors, AKT inhibitors, mTOR inhibitors, WNT inhibitors, SRC inhibitors, VEGFR inhibitors, TGF-beta inhibitors, SRC inhibitors or GSK3 inhibitors were non-lethal or at most sub-lethal. Interestingly, some phenotypes (including those induced by MEK-inhibition or EGFR inhibition) showed viability effects only in a subset of PDO lines.
These analyses demonstrate that phenotypes induced by compounds in multicellular PDOs can be detected by unsupervised, automated image analysis. We found that compound-induced PDO phenotypes clustered by compound mode-of-action. This allows identification of novel modes-of-action or off-target effects of known compounds and can be used to classify compounds with unknown targets. Furthermore, we demonstrated that apart from viability defects, non- or sub-lethal morphological phenotypes are often responsible for compound clustering. Therefore, high-throughput imaging provides an in-depth phenotypic view complementary to viability drug profiling.

\section{Compound-induced organoid phenotypes reflect biological mechanisms}

Next we aimed to characterize the compound induced phenotypes in more detail. Compounds with similar mode-of-action clustered together across different PDO lines, suggesting links to molecular targets and pathways (Figure 6a-c). The strongly enriched organoid phenotype caused by inhibition of MEK signaling for instance led to a cystic reorganization, i.e. development of a detectable lumen or enlargement of an existing lumen, towards round and thin-walled organoids, resembling the organized baseline phenotype, but coming from diverse basal phenotypic subsets (Figure 6a). On the feature level, this phenotype was represented by a greater degree of eccentricity, while actin and DNA intensity were reduced. The RAF (upstream of MEK) inhibitor, as well as the ERK inhibitor (downstream of MEK in the MAPK pathway) found in the MEKi-enriched cluster (compare Figure 5c) exhibited the same organized/cystic phenotype (Supplemental Figure S8a). EGFR inhibition caused a less striking morphology (Supplemental Figure S8b), thereby suggesting the cystic organization phenotype to be related to MAPK signaling downstream of RAF kinase level. The organized/cystic phenotype occurred in PDOs not dying of MEK inhibition (e.g. D007T, D018T, compare Figure 5d) and PDOs with mild viability defects due to the treatment (D019T, D027T). Interestingly, we observed the same phenotype also in PDO lines with strong viability response (D004T, D046T, next to cloudy dead organoid in Figure 6a), suggesting that this phenotype is independent of PDO viability.

While GSK inhibitors did not affect PDO viability in our phenotypic profiling assay, the induced phenotype was striking, showing a disintegration of PDO architecture towards irregular shaped organoids with small, dismembered structures, evocative of grape-like shapes (Figure 6b). In contrast, inhibition of CDKs prove to be related to cell death as shown by LDC. PDOs exhibited markedly reduced organoid size and disappearance of a lumen, sometimes with speckled satellite cells (Figure 6c). Intensity of actin and DNA staining were reduced in this phenotype, while eccentricity and permeability staining were increased. Other targeted inhibitors, such as mTORi led to reduced organoid size and a more compact PDO organization (reduced eccentricity, Supplemental Figure S8c). 	
To elucidate the molecular mechanisms behind compound induced phenotypes, we performed expression analyses of PDOs treated with MEK inhibitor trametinib and GSK inhibitor CHIR-09014. We selected PDO lines D004T, D007T and D019T that all showed the re-organized cystic phenotype but had different viability responses upon MEK inhibition. After treatment of PDO lines with trametinib for 72h, we observed a downregulation of MAPK related genes and transcriptional targets. In line with prior work, we also observed upregulation of genes related to the cell cycle, alongside with LGR5/EphB2 stem cell signature genes. We had also observed these stem cell signatures to be enriched in organized (large cystic) PDOs in unperturbed state. Hence, sublethal inhibition of MEK kinase appeared to induce a paradoxical, stem cell rich phenotype in our colorectal cancer PDO models. For analysis of the GSK inhibitor induced phenotype, we chose PDO line D019T and the compound CHIR-09014 as it strongly exhibited the grape-like phenotype. Next to increases in RNA processing, indicating strong transcriptional activity, we specifically observed a strong downregulation of the Focal Adhesion-PI3K-AKT-mTOR-signaling pathway (Wikipathways; q < 0.04), IRS mediated signaling (Reactome; q < 0.04) and integrin cell surface signaling (PID; q < 0.04). Thus inhibition of GSK leads to changes in Focal Adhesion-PI3K-AKT-mTOR-signaling which likely contribute to the disassembly of organoid structures. Notably, we found no regulation of genes related to apoptosis or cell cycle arrest, substantiating LDC results.

Next, we asked whether compound induced phenotypes were uniformly induced in all PDOs, or whether some were specific to subsets of PDOs. To this end, we clustered compounds by phenotypic profiles for every individual profiled PDO line. By calculating the odds ratio of specific mechanisms-of-action to be enriched, we identified differences in number and type of enriched phenotypes (Figure 6f). Some compound classes (including MEK and CDK inhibitors) induced significantly enriched phenotypes in almost all PDO lines, while several others (including AKT inhibitors, PKC inhibitors or GSK3 inhibitors) were specific to subsets of PDO lines. Accordingly, PDO lines D020T or D021T treated with GSK inhibitors only rudimentarily exhibited the typical GSK phenotype observed in other PDO lines (Supplemental Figure S8d). We then wondered whether occurrence of specific compound induced phenotypes could be dependent on molecular alterations observed in PDOs. (Supplemental Figure S9a-b). As a measure of compound activity, we again used the AUROC of the SVM to differentiate active compound treatments from inactive treatments. We then proceeded to check for compounds in which activity significantly depended on the mutation status of the treated PDOs. Here we focused on genes mutated at sufficient frequency in our organoid panel (n ≥ 3; PIK3CA, RAS, TP53 and APC). While we could not detect any genotype-dependent differences in drug induced phenotypes for most compounds tested, we found that (amongst others, Supplemental Figure S9a) the mutation status of PIK3CA influences the activity of several compounds targeting the PI3K/AKT pathway, including HER2 (tyrphostin AG 879), AKT (AT7867) or mTOR (WYE-354). Accordingly, PDOs treated with AT7867 showed morphological differences when comparing PIK3CA mutated and PIK3CA wild-type lines (Supplemental Figure 9b).

These analyses demonstrated that three-dimensional, multicellular PDOs showed characteristic and complex morphological responses to defined molecular perturbations. We detected compound specific phenotypes that were reproducible between PDOs from different donors. Furthermore, we explained morphological responses on a molecular level with changes in RNA expression programs of specific pathways. Also, we found phenotypes present only in subsets of PDO lines. Cancer mutations could influence the overall morphologic response of PDOs to targeted compounds even when the treatment might not affect organoid viability.


