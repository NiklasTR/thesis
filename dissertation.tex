%!TEX TS-program = xelatex 
%!TEX encoding = UTF-8 Unicode

% Modify the following line to match your school
% Available options include `Harvard`, `Princeton`, and `NYU`.
\documentclass[School=Harvard]{Dissertate}
\usepackage{textgreek}
\usepackage{nomencl}
\usepackage{etoolbox}
\makenomenclature

% define function
\renewcommand\nomgroup[1]{%
  \ifstrequal{#1}{A}{%
    \item[\bfseries\large Abbreviations]\vspace{10pt} % Adjust the space as needed
  }{%
    \ifstrequal{#1}{N}{%
      \item[\bfseries\large Notation]\vspace{10pt} % Adjust the space as needed
    }{}%
  }%
}

\begin{document}

% the front matter
% Some details about the dissertation.
\title{Multiparametric phenotyping of intestinal organoids to model disease initiation and treatment reponse in colorectal cancer}
\author{Niklas Timon Rindtorff}
\advisor{Prof. Dr. Michael Boutros}

% ... about the degree.
\degree{} % not filled out in modified version
\field{} % not filled out in modified version
\degreeyear{2022}
\degreemonth{Oktober}
%\department{Medizinische Fakult{\"a}t Heidelberg}

% ... about the candidate's previous degrees.

\maketitle
\copyrightpage





% nomenclature
\renewcommand{\nomname}{List of Abbreviations and Notation}
% abbreviation section
\nomenclature[A]{AUC}{Area under the curve}
\nomenclature[A]{CIMP}{CpG island methylator phenotype}
\nomenclature[A]{CIN}{Chromosomal instability}
\nomenclature[A]{CRC}{Colorectal cancer}
\nomenclature[A]{PDO}{Patient Derived Organoid}
\nomenclature[A]{CRISPR}{Clustered Regularly Interspaced Short Palindromic Repeats}
\nomenclature[A]{DMEM}{Dulbecco’s Modified Eagle Medium}
\nomenclature[A]{DMSO}{Dimethyl sulfoxide}
\nomenclature[A]{FDR}{False discovery rate}
\nomenclature[A]{MSI}{Microsatellite instability}
\nomenclature[A]{MSS}{Microsatellite stability}
\nomenclature[A]{AUROC}{Area under the receiver operating characteristic}
\nomenclature[A]{UICC}{Union for International Cancer Control}
\nomenclature[A]{sgRNA}{Single guide RNA}
\nomenclature[A]{5FU}{5-fluorouracil}
\nomenclature[A]{HexCer}{Hexosylceramide}
\nomenclature[A]{PA}{Phosphatidate}
\nomenclature[A]{Cer}{Ceramide}
\nomenclature[A]{SM}{Sphingomyelin}
\nomenclature[A]{LPC}{Lysophosphatidylcholine}
\nomenclature[A]{PS}{Phosphatidylserine}
\nomenclature[A]{PE}{Phosphatidylethanolamine}
\nomenclature[A]{PI}{Phosphatidylinositol}
\nomenclature[A]{PG}{Phosphatidylglycerol}
\nomenclature[A]{DAG}{Diacylglycerol}
\nomenclature[A]{PC}{Phosphatidylcholine}
\nomenclature[A]{Hex2Cer}{Dihexosylceramide}
\nomenclature[A]{CE}{Cholesterol Ester}
\nomenclature[A]{Chol}{Cholesterol}
\nomenclature[A]{TAG}{Triacylglycerol}
\nomenclature[A]{ARD}{Automatic Relevance Determination}
\nomenclature[A]{FOLFOX}{Folinic Acid, 5-Fluorouracil, and Oxaliplatin}
\nomenclature[A]{FOLFIRI}{Folinic Acid, 5-Fluorouracil, and Irinotecan}
\nomenclature[A]{FOLFOXIRI}{Folinic Acid, 5-Fluorouracil, Oxaliplatin, and Irinotecan}
\nomenclature[A]{EKG}{Electrokardiogram}
\nomenclature[A]{BMP}{Bone Morphogenic Protein}
\nomenclature[A]{HRP}{Horse Radish Peroxidase}
\nomenclature[A]{RIPA}{Radioimmunoprecipitation Assay Buffer}
\nomenclature[A]{DTT}{Dithiothreitol}
\nomenclature[A]{PDO}{Patient Derived Organoid}
\nomenclature[A]{TRITC}{Tetramethylrhodamine}
\nomenclature[A]{DAPI}{4′,6-Diamidino-2-Phenylindole}
\nomenclature[A]{HDF5}{Hierarchical Data Format Version 5}
\nomenclature[A]{PCA}{Principal Component Analysis}
\nomenclature[A]{ICA}{Independent Component Analysis}
\nomenclature[A]{NMF}{Non-negative Matrix Factorization}
\nomenclature[A]{ANOVA}{Analysis of Variance}
\nomenclature[A]{RMA}{Robust Multiarray Average}
\nomenclature[A]{MOFA}{Multi-omics Factor Analysis}
\nomenclature[A]{CMS}{Consensus Molecular Subtype}
\nomenclature[A]{GSEA}{Gene Set Enrichment Analysis}
\nomenclature[A]{BCA}{Bicinchoninic Acid Assay}
\nomenclature[A]{WENRAS}{Organoid Medium containing Wnt3a, Egf, Noggin, R-Spondin, A83-01, and SB202190}
\nomenclature[A]{ENA}{Organoid Medium containing Egf, Noggin, and A83-01}
\nomenclature[A]{ENAS}{Organoid Medium containing Egf, Noggin, A83-01, and SB202190}
\nomenclature[A]{UMAP}{Uniform Manifold Approximation and Projection}
\nomenclature[A]{CRIS}{CRC Intrinsic Subtypes}
\nomenclature[A]{BWA}{Burrows-Wheeler Aligner}
\nomenclature[A]{GATK}{Genome Analysis Toolkit}
\nomenclature[A]{GRC38}{Genome Reference Consortium Human Build 38}
\nomenclature[A]{MCR}{Mutation Cluster Region}
\nomenclature[A]{WT}{Wildtype}
\nomenclature[A]{HEPES}{4-(2-Hydroxyethyl)-1-Piperazineethanesulfonic Acid}
\nomenclature[A]{DPBS}{Dulbecco's Phosphate Buffered Saline}
\nomenclature[A]{CTG}{Cell Titer Glo Viability Assay}
\nomenclature[A]{ATP}{Adenosine triphosphate}
\nomenclature[A]{PBS}{Phosphate Buffered Saline}
\nomenclature[A]{PBS-T}{Phosphate Buffered Saline containing 0.1\% (v/v) Triton X-100}
\nomenclature[A]{ACN}{Acetonitrile}
\nomenclature[A]{PAINS}{Pan Assay INterference compoundS}
\nomenclature[A]{siRNA}{small interfering RNA}

% notation section
\nomenclature[N]{$\mathbf{X}$}{Matrix}
\nomenclature[N]{$\mathbf{X}^T$}{Matrix Transpose}
\nomenclature[N]{$\mathbf{X}^+$}{Pseudoinverse of a Matrix}
\nomenclature[N]{$\mathbf{Y_m}$}{Measured data matrix of modality $m$}
\nomenclature[N]{$\mathbf{Z}$}{Factor score matrix}
\nomenclature[N]{$\mathbf{W_m}$}{Factor weight matrix (also called loading matrix) of modality $m$ }
\nomenclature[N]{$\boldsymbol{\epsilon_m}$}{Error terms of modality $m$}




% continuing with frontmatter
\tableofcontents
\listoffigures
\addcontentsline{toc}{chapter}{List of Figures}
\listoftables
\addcontentsline{toc}{chapter}{List of Tables}
\printnomenclature
\addcontentsline{toc}{chapter}{List of Abbreviations and Nomenclature}
\doublespacing



% include each chapter...
\setcounter{chapter}{0}  % start chapter numbering at 1 (set to -1 to start at 0)
\begin{savequote}[75mm]
Nothing in Biology Makes Sense Except in the Light of Evolution
\qauthor{Theodosius Dobzhansky}
\end{savequote}

\chapter{Introduction}
\label{introduction}
\begin{flushleft}
\setlength{\parindent}{7ex}
\section{Disclosure}
Parts of this introduction, especially the section on Wnt signaling, have been adapted from own publications, including \textit{Wnt signaling in cancer} \cite{Zhan2017}

\section{The Colon}
\subsection{Colon Function and Value as Model Organ}


evolutionary fundamental value
representative model for complex organ
well understood stem cell biology enabled modeling in vitro 
devastating diseases emerge from colon

\subsection{The Colon Stem Cell Niche}
In order to understand the underlying principles of colorectal cancer development, or any cancer in general, it is advisable to turn to the stem cell biology governing the tissue's anatomy and function. The colon stem cell niche, or crypt, is the source of all epithelial cells lining the colon. Similar to the small intestine, Lgr5+ instestinal stemcells are located at the bottom of the crypt and continuously renew the epithelium by proliferating and pushing out new cells towards the colon's lumen.

This architecture serves multiple purposes, including protection of stem cells and the control of cell fate decisions across the epithelium. Multiple developmental pathways, especially Wnt signaling, Notch, BMP and ERK MAPK signaling, govern cell identity in the intestinal niche \cite{Gehart2019}. The concentration of signaling cues for most of these pathways are organized in gradients along the crypt-lumen axis. For example, the concentration of stem cell property maintaining Wnt and EGF ligands, secreted by mesenchymal crypt cells and REG4+ deep secretory cells, decreases as cells are pushed outside of the crypt  \cite{Sasaki2016}. In contrast, the effect of cell differentiating BMP ligands increases as the effect basal mesenchmymal cell derived BMP inhibitors, such as Noggin, is reduced. In summary, as a cell is pushed outside the crypt by a continuous stream of fresh proliferating cells, developmental signaling cues vanish and subsequent gene expression changes lead to differentiation. Similarly, if cells were to move back into the crypt, the ambient signaling would lead to a reprogramming towards an intestinal stem cell fate. 

Given the spacial confinement of proliferate signals, the crypt architecture leads to a protection against malignant transformation, too. At the bottom of the crypt, a neutral competition of proliferating intestinal stem cells leads to the rapid removal of cells that show reductions in their proliferation rate relative to wildtype stem cells, which is often the case in malignant neoplasms. Given this neutral competition and the dependence on external signaling, every dysfunctional or transformed cell is likely removed from the niche and differentiates unless it acquires a set of molecular alterations that render it independent from niche signals. As mentioned above, the key signaling pathways that maintain stem cell properties in the crypt are canonical Wnt signaling and Notch signaling, while BMP signaling inhibits stemness and EGF dependent ERK MAPK signaling triggers cell proliferation. Given their exerted evolutionary pressure on malignant cells, it comes at no suprise that the majority of early driver mutations found in colorectal cancer, such as loss of APC (Wnt signaling), activation of \textbeta-catenin (Wnt signaling), activation of KRAS (ERK MAPK), BRAF (ERK MAPK) and loss of SMAD4 (TGFb/BMP signaling) are found in these exact signaling pathways. Of these mutations, especially mutations of APC and KRAS are frequently observed early in colorectal cancer development and are highly correlated with eachother (50\% of APC mutant tumors harbor mutations of KRAS), leading to both induced proliferative capacity and growth. In summary, the architecture of the colon stem cell niche and the signaling pathways required to regulate stemness and cell proliferation influence the evolutionary landscape of colorectal cancer development and thus account for the majority of early drive mutations, especially loss of APC and activation of KRAS, in this disease.

\subsection{Signaling Pathways controlling the Colon Stem Cell Niche}
\subsubsection{Canonical Wnt Signaling}
In 1973, the wingless gene was discovered in a screen for visual phenotypes, affecting patterning processes in Drosophila melanogaster, the fruit fly \cite{Sharma1973WinglessMelanogaster.}. Subsequently, further genetic screens identified components of the Wnt family as key regulators during embryonic development and later, cancer initiation as well as stem cell maintanance  \cite{Nusslein-Volhard1980MutationsDrosophila}. \par 

In canonical Wnt signaling, absence of Wnt ligands leads to phosphorylation of \textbeta-catenin by the destruction complex, which contains the scaffold protein Axin, the large protein APC (Adenomatous polyposis coli) and the kinases GSK3\textbeta as well as casein kinase (CK1\textalpha) (reviewed in Zhan, Rindtorff et al.\cite{Zhan2017}). 
In this state, \textbeta-catenin is phosphorylated by GSK3\textbeta, ubiquitinated by \textbeta-TrCP and subsequently targeted for proteasomal degradation. 
In the absence of nuclear \textbeta-catenin, the trasncritpional repressive complex containing TCF/LEF and transducing-like enhancer protein (TLE/Groucho) recruits Histone deacetylases to repress target genes. \par 

The canonical pathway is activated upon binding of secreted Wnt ligands (for example, Wnt1 and Wnt 3a) to Fzd receptors and LRP co-receptors. 
Subsequently, LRP receptors are  phosphorylated by CK1\textalpha and GSK3\textbeta, which then recruits Dishevelled (Dvl) proteins to the plasma membrane where they polymerize and are activated \cite{Metcalfe2011}. Next, the Dvl polymers inactivate the destruction complex by sequestration in multivesicular bodies. This results in stabilization and accumulation of \textbeta-catenin which then translocates into the nucleus. There, \textbeta-catenin forms an active complex with LEF (lymphoid enhancer factor) and TCF (T-cell factor) proteins by displacing TLE/Groucho complexes which leads to the recruitment of histone modifying co-activators such as CBP/p300, BRG1, BCL9 and Pygo \cite{Lien2014WntSignaling}. \par 

Next to Wnt ligands, members of the R-spondin ligand family are positive effectors of Wnt signaling \cite{Kazanskaya2004, Glinka2011, Hao2012}. R-spondins bind to leucine-rich repeat containing G-protein-coupled receptors (Lgr) 4-6 \cite{Koo2012a}. In the absence of R-spondin, the two E3 ubiquitin ligases Znrf3 and Rnf43 target the Frizzled (Fzd) receptor for lysosomal degradation \cite{DeLau2011}. The interaction of ubiquitin ligases and receptrs is dependent on Dishevelled (Dsh).\cite{Jiang2015}. In the presence of external R-Spondins, binding of ligands to Lgr4-6 inhibits the activity of Znrf3/ Rnf43 and leads to the accumulation of Fzd receptors on the cell surface \cite{Hao2012, Koo2012a}. Being transcriptional targets of Wnt signaling, Znrf3 and Rnf43 function as negative feedback regulators in Lgr5- positive cells \cite{DeLau2012}. \par 


\subsubsection{ERK MAPK Signaling}
The extracellular-signal-regulated (ERK) mitogen-activated protein kinase family (MAPK) is one of three major MAPK families, together with the JNK (c-jun N-terminal kinase or stress-activated protein kinases) and MAPK14 group of protein kinases. These signaling cascades play a major role in (I) integrating external proliferative signals, (II) reacting to stress or ambient cytokines and (III) protecting cells from apoptosis, respectively \cite{Oncol2005}. 


\subsubsection{IGF and mTOR Signaling}

\subsubsection{TGF beta Signaling}

\subsubsection{TP53 Signaling}

\subsection{Colon Organoid models}
Intestinal organoids are three-dimensional cell culture models from primary adult tissue. Organoids develop from Lgr5+ adult stem cells and were first isolated from the small intestine of mice \cite{Sato2011}. Subsequently, following the initial methodology, further organoid models across tissue-types and species have been developed. These include organoids from intact and cancerous human large intestine \cite{Sato2011}, pancreas \cite{Sachs2017}, mammary epithelium \cite{Zhang2016EstablishingCells, Sachs2017AHeterogeneity}and the hepatobiliary system \cite{Huch2013NIHAccess, Broutier2016CultureManipulation.}.
Culturing these cells requires the addition of specific tissue-dependent growth factors and the embedding of cell in 3D hydrogels \cite{Merker2016GastrointestinalOut}. In the case of colon organoids, the necessary growth factors are inspired by signaling cues available in the intestinal stem cell niche: Wnt and R-spondin ligands secreted by PDGFR+ myofibroblasts activate and maintain canonical Wnt signaling; EGF ligands stimulate ERK MAPK signaling and Noggin ligands inhibit the differentiating effects of the BMP signaling cascade \cite{Sato2013}. When combined with inhibitors of TGF\textbeta and p38 mediated signaling, these growth factors can stimulate the formation and continous proliferatation of organoids ex-vivo. 

Not only because of their high isolation efficancy of up to 80\%, organoids are an increasingly popular model as they mimic their respective tissue of origin, including colorectal cancer (Pauli et al. 2017). Due to a high isolation efficiency and preserved tumor biology, patient derived colorectal cancer organoids have been used as personalized cancer models for precision oncology \cite{VanDeWetering2015, Vlachogiannis2018}. Moreover, organoids from healthy tissue can be cultured in-vitro as well and are amenable to genetic editing (Matano et al. 2015 Drost et al. (2015)). Therefore, these models also enable studying tumor development at a single mutation resolution.

\section{Colorectal Cancer}
\subsection{Colorectal Cancer Epidemiology}
Colorectal cancer is the third most common cancer worldwide and is associated with a Western lifestyle. Similar to other solid tumors, colorectal adenocarcinoma progression is classified into four stages by the UICC (Union for International Cancer Control). These range from a \textit{carcinoma-in-situ}, a malignant patch of cells that has not yet breached the basal lamina of the intestinal mucosa (stage 0), to metastatic disease (stage 4).\par


\begin{figure}[h]
\centering
\includegraphics[scale=.35]{figures/colon_cancer_progression.png}
\caption{Colon Cancer Progression}
\label{colon_cancer_progression}
\end{figure}

\subsection{Colorectal Cancer Emergence}
According to the adenoma-carcinoma sequence model, the majority of all colorectal adenocarcinomas arise from previously formed adenomas, benign neoplasms of the intestinal epithelium \cite{Cho1992}. Thus, today, one of the most effective medical interventions to reduce death from colorectal cancer is the preventative removal of visible adenomas during lower endoscopy, such as colonoscopy \cite{Nishihara2013Long-TermEndoscopy}.\par

On a molecular level, colorectal adenocarcinoma can be organized into tumors arising through (I) a chromosomal instability or (II) a DNA-mismatch repair deficiency associated route \cite{Markowitz2009} (Figure \ref{colon_cancer_progression}). These two forms of tumor development have been associated with characteristic clinical, pathological and molecular findings. For example, tumors of the DNA-mismatch repair phenotype are more frequently located in the right colon, have a higher proportion of microsatellite instability, frequent BRAF mutations and a higher immune-cell infiltration \cite{Markowitz2009}. In contrast, tumors of the common chromosomal instability (CIN) phenotype are mostly microsatellite stable and have frequent APC and KRAS mutations. \par 

The cascade of genetic events leading to the more frequent chromosomal instability associated form of colorectal cancer cause hyperactivation of a range of signaling pathways. Briefly, this cascade, known as the Vogelstein sequence \cite{Cho1992}, starts with the loss of the tumor suppressor APC in the intestinal epithelium. Loss of APC, which leads to adenoma formation, is followed by the activation of KRAS, PIK3CA, loss of SMAD4 and TP53. Other forms of the disease, especially microsatellite-instable forms of colorectal cancer also harbor mutations of APC but show strong prevalence of BRAF mutations instead of KRAS mutations \cite{Guinney2015TheCancer.}. \par 

Prior studies trying to further define colorectal cancer beyond these two developmental routes have proposed a set of different molecular subtypes \cite{Menter}. In an attempt to unify these models, four consensus molecular subtypes (CMS) have been proposed \cite{Guinney2015TheCancer.}. Briefly, these subtypes organize colorectal cancer into classes defined by (I) a high fraction of microsatellite instable tumors, (II) APC mutations, (III) KRAS mutations and (IV) stromal infiltration, respectively. However, recent evidence has questioned the interpretability of these subtypes in multiple ways. First, the existence of non-malignant cells within the analyzed samples does not allow a direct interpretation of cancer-exclusive molecular processes which might govern treatment response and prognosis \cite{Dunne}. Second, the sampled intra-tumor region and its cellular composition influence the subtype classification result \cite{Dunne2016ChallengingCancer.}. Third, validating studies of the consensus molecular subtype have shown that a large fraction of cancer samples can not be confidently assigned to a single subtype and, more importantly, that subtypes, instead of being well separated, are rather on a high-dimensional continuum. This continuum can be defined by (I) markers of inflammation and T-cell activity and (II) markers of CDK-regulated DNA replication, leading to four quadrants in a continuous space \cite{Ma}. For example, microsatellite instable tumors, which are most frequently found in the first CMS subgroup, were associated with a strong T-cell activity signature and a low CDK-regulated DNA replication signature. Along these lines, immune-cell infiltration has been established as an independent prognostic factor of overall survival and recurrence risk of colorectal cancer and is exploited in immunotherapy, which is especially active in MSI-high tumors \cite{galon, pages}. In contrast, CDK-dependent signaling has recently been recognized as a potential driver of immune-escape in multiple solid cancers \cite{Chaikovsky1}. \par 

In summary, the molecular landscape of colorectal cancer is organized by two distinct forms of tumor development, chromosomal instability and microsatellite instability, that are linked to characteristic genetic changes resulting in a continuum of gene expression states which present themselves with varying degrees of cell proliferation and inflammation. \par

\subsection{Signaling Pathways controlling Colorectal Cancer}
\subsubsection{Wnt signaling in colorectal cancer}
The role of Wnt signaling during colorectal cancer development is well established \cite{Polakis2007}. While activating mutations of \textbeta-catenin do exist, loss of APC is the most frequent driver of Wnt signaling in colorectal cancer and can be found in about 80\% of colorectal cancer patients \cite{Fearon1989}. In line with its role as a tumor suppressor, truncating of APC using the CRISPR/Cas9 technology, leads to colorectal cancer development, which can be modeled ex vivo in human intestinal organoids \cite{Matano2015, Drost2015SequentialCells}. Furthermore, by using a mouse model allowing the reversible knockdown of Apc via shRNA, it was demonstrated that adenomas could regress to normal tissue once APC function is restored, underlining the importance of continuous Wnt signaling for tumor maintenance \cite{Dow2015}. \par
Although loss of APC in general is a driving event of colorectal cancer development and persistence, not every mutation of the APC gene leads to a similar phenotype. Studies of human colorectal cancer samples and tumors from mouse models revealed that different mutations of APC result in distinct levels of canonical Wnt pathway activity and, in addition, are associated with characteristic tumor locations within the large intestine \cite{Christie2013, Buchert2010}. \par

Besides APC, mutations in R-spondin and RNF43, regulators of Wnt receptor abundance at the cell surface level, 
were implicated as drivers of Wnt-dependent tumor growth. Deleterious RNF43 mutations have been described in ~20\% of colorectal cancer cases and are mutually exclusive to APC mutations. Also, amplified R-spondin3 fusion proteins have been described in 10\% of CRC cases. While APC and \textbeta-catenin are generally considered independent of Wnt ligand availability, RNF43 mutant cancers are strongly dependent on Wnt secretion, rendering them highly susceptible to Wnt secretion targeted therapy.

\subsubsection{ERK MAPK Signaling in Colorectal cancer}


In colorectal cancer, the ERK MAPK signaling cascade and its members Ras/Raf/MEK and ERK are key regulators of cell proliferation in malignant cells. The Ras kinase members are mutated in about 36\% of colorectal cancers \cite{Oncol2005}. According to the Vogelstein model of colorectal cancer initiation \cite{Fearon1989}, activating mutations of KRAS takes place early during cancer development, more specifically, after loss of APC.
Next to KRAS, BRAF mutations can also be found in around 10\% of colorectal cancers \cite{Oncol2005}. Of note, mutations of KRAS and BRAF occur mostly in a mutually exclusive pattern with BRAF mutations being enriched in Microsatellite instable colorectal cancers \cite{Oncol2005, Sahin2013}. 

\subsubsection{IGF and mTOR Signaling in Colorectal Cancer}

\subsubsection{TGF beta Signaling in Colorectal cancer}

\subsubsection{TP53 Signaling in Colorectal cancer}

\subsection{Colorectal Cancer Organoid models}

\section{Colorectal Cancer Therapy}
The treatment of colorectal adenocarcinoma depends on disease stage. While surgical removal of the tumor is at the center of the treatment strategy, neoadjuvant and adjuvant chemotherapy are part of the recommended therapy from UICC stage 2 and 4 on, respectively. Today, for metastatic colorectal cancer the first line treatment includes combination chemotherapy (FOLFOX or FOLFIRI) paired with Cetuximab (anti-EGFR) for KRAS wildtype disease, combination chemotherapy with Bevacizumab (anti-VEGFR) for KRAS mutant disease or triple chemotherapy (FOLFOXIRI) in combination with Bevacizumab for BRAF mutant disease \cite{Cutsem}. Following lines of therapy include different combinations of the aforementioned agents with the exception of Regorafenib and Triflouridin/Tipiracil as preferred third line agents for non-KRAS wildtype disease \cite{Cutsem}. Consequentially, the only genetic tests currently recommended during therapy are the determination of KRAS and BRAF status \cite{Cutsem}. Other genetic tests or targeted inhibitors have so far not found their way into clinical practice.\par

Both the important role of adenocarcinoma development and the limitations of personalized therapy in advanced stages of the disease motivated the research presented in this dissertation.

\section{Cancer Drug Discovery}
\subsection{Rational and Functional Drug Discovery}
\subsection{Image-based Profiling}
A challenge for organoid research is the need for information-rich drug-testing methods. In the past, automated microscopy of 2D-cells has been used to measure the biological activity of compounds (Breinig et al. 2015). Prompted by this, we build a platform for high-throughput drug activity profiling in organoids. The platform uses confocal microscopy to collect fluorescent images of treated organoids in 3D. We devel- oped SCOPE, a software package, to process these images and measure organoid phenotypes. We used this platform to test compounds on both patient derived organoids and genetically engineered organoid models.

\section{Aims of Thesis}
\subsection{Image-based profiling of colon organoid models}
\subsection{patient derived organoids identifies compound-induced phenotypes}
\subsection{Multi-omics profiling of intestinal organoids identifies an epistatic relationship of Apc loss and Kras activation during colorectal cancer development}
\end{flushleft}

% First, we isolated and characterized patient derived colorectal cancer organoids. Next, we performed high- throughput drug profiling of these organoids. Here, we observed a variety of recurring treatment-induced phenotypes. These were linked to specific cellular processes. Of note, the treatment response of organoids in-vitro matched the response of donating patients.

% Second, we isolated colon organoids from healthy mouse tissue. By using gene-editing, we generated models of colon adenoma - a precursor lesion of colorectal cancer. These models carried different combinations of mutations in the Apc and Kras gene. Both are frequent and co-occurring mutations in colorectal cancer (Schell et al. 2016). To better understand the mechanisms of tumor development, we performed a multi- omics characterization in these adenoma models. Moreover, we profiled genotype specific drug effects. Here we found a reorganization of organoid phenotype after loss of Apc, which masks the effects of an isolated Kras mutation. Loss of Apc leads to genotype specific drug-induced phenotypes and vulnerabilities. Oncogenic Kras buffers a subset of these vulnerabilities, offering a new perspective on the relationship of Apc and Kras during tumor development.
\begin{flushleft}

\chapter{Materials and Methods}

\section{Disclosure}
The materials and methods chapter, including text and figures, has been adapted from a joint first-author publication, \textit{The drug-induced phenotypic landscape of colorectal cancer organoids} \parencite{betgeDruginducedPhenotypicLandscape2022}. Descriptions of core facility services are quoted and indicated as such.

\section{Patients}
All patients were identified at the University Hospital Mannheim, Mannheim, Germany. Untreated patients with a new diagnosis of colorectal cancer were included in this study. Biopsies were obtained from their primary tumors and adjacent normal tissue via forceps-based endoscopy. Exclusion criteria were active HIV, HBV or HCV infections. Biopsies were transported in phosphate buffered saline (PBS) on ice. Clinical data, tumor characteristics and molecular tumor data were pseudonymized. The study was approved by the Medical Ethics Committee II of the Medical Faculty Mannheim, Heidelberg University (Reference no. 2014-633N-MA and 2016-607N-MA). All patients gave written informed consent before tumor biopsy was performed. The initial cohort consisted of organoids from 25 patients with colorectal cancer, 10 of them female, 15 male, with a mean age of 66 years.

\begin{table}[htbp]
\caption{Patient derived organoid Lines}
\label{tab:patient_organoid}
\begin{tabularx}{\textwidth}{lXXXXX}
\toprule
\textbf{Line} & \textbf{Optimal medium} & \textbf{Qualitative growth} & \textbf{Patient sex} & \textbf{Location} & \textbf{UICC stage} \\
\midrule
D004T & ENA & good & female & rectum & 3 \\
D007T & ENA & good & m & rectum & 3 \\
D010T & ENA & good & female & sigmoid & 3 \\
D013T & ENAS & good & m & rectum & 1 \\
D015T & ENA & good & m & descending & 2 \\
D018T & ENA & good & female & sigm & 1 \\
D019T & ENA & good & m & sigmoid & 2 \\
D020T & ENA & good & m & rectum & 1 \\
D021T & ENA & medium & m & rectum & 1 \\
D022T & ENA & good & m & ascending & 4 \\
D027T & ENA & good & female & rectum & 2 \\
D030T & ENA & good & female & ascending & 2 \\
D046T & ENA & good & female & rectum & 3 \\
\bottomrule
\end{tabularx}
\end{table}


\section{Organoid Culture}

\subsection{Patient Derived Organoid Culture}
Patient derived organiod (PDO) cultures were extracted from biopsies as reported by Sato et al. \parencite{satoLongtermExpansionEpithelial2011} with slight modifications. Tissue fragments were washed in DPBS (Life technologies) and digested with Liberase TH (Roche) before embedding into BME R1 (Trevigen). The medium, termed ENA, contained Advanced DMEM/F12 (Life technologies) medium with 1\% v/v penicillin/streptomycin (Life Technologies), Glutamax and HEPES (basal medium) supplemented with 100 ng/ml Noggin (Peprotech), B27 (Life technologies), 1,25 mM n-Acetyl Cysteine (Sigma), 10 mM Nicotinamide (Sigma), 50 ng/ml human EGF (Peprotech), 10 nM Gastrin (Peprotech), 500 nM A83-01 (Biocat), 10 nM Prostaglandin E2 (Santa Cruz Biotechnology), 10 μM Y-27632 (Selleck chemicals) and 100 mg/ml Primocin (Invivogen). After isolation, cells were kept in 2 conditions including medium as described (ENA), or supplemented with additional 3 uM SB202190 (Biomol) (ENAS) as described by Fujii et al. \parencite{Fujii2016-ax}. 
The tumor niche was determined after 14 days and organoids were subsequently cultured in the condition with best growth. 
Organoids were passaged every 7 days and medium was changed every 2-3 days.

\begin{table}[htbp]
\caption{Basal Medium Components}
\label{tab:basal_medium_components}
\begin{tabularx}{\textwidth}{Xll}
\toprule
\textbf{Component} &  \textbf{Concentration} & \textbf{Manufacturer} \\
\midrule
Advanced DMEM/F12 & 97\% v/v & Life technologies; Carlsbad, California, USA \\
GlutaMAX™ (100x) & 1\% v/v & Life technologies; Carlsbad, California, USA \\
Pen/Strep (100x) & 1\% v/v & Life technologies; Carlsbad, California, USA \\
1 M HEPES (100x) & 1\% v/v & Sigma-Aldrich Life Science/Merck, St. Louis, Missouri, USA \\
\bottomrule
\end{tabularx}
\end{table}

\begin{table}[htbp]
\caption{ENA Medium Components}
\label{tab:ena_medium_supplements}
\begin{tabularx}{\textwidth}{Xll}
\toprule
\textbf{Supplement} & \textbf{Concentration} & \textbf{Manufacturer} \\
\midrule
Basal Medium (described above) & 98\% v/v & Produced In-House \\
B27 (50x) & 2\% v/v & Life technologies; Carlsbad, California, USA \\
N-Acetylcysteine & 1.25 mM & Sigma/Merck; St. Louis, Missouri, USA \\
Human EGF & 50 ng/ml & PeproTech; Hamburg, Germany \\
Noggin & 100 ng/ml & PeproTech; Hamburg, Germany \\
Y-27632 & 10 µM & Selleck chemicals; Houston, Texas, USA \\
A83-01 & 500 nM & BioCat; Heidelberg, Germany \\
Prostaglandin E2 & 10 nM & Santa Cruz Biotechnology \\
Gastrin & 10 nM & PeproTech; Hamburg, Germany \\
Nicotinamide (Vit. B3) & 10 mM & Sigma/Merck; St. Louis, Missouri, USA \\
Primocin & 100 µg/ml & Invivogen; San Diego, California, USA \\
\bottomrule
\end{tabularx}
\end{table}


\begin{figure}[!h]
\centering
\includegraphics[width=250pt,
                height=\textheight,
                keepaspectratio]{figures/promise/pdf/fig_1_0.pdf}
\caption[Patient derived organoid cohort overview]{\textbf{Patient derived organoid cohort overview a} Tumor location (right/left/rectum) and AJCC/UICC stage of colorectal cancers that patient derived organoids were derived from. \textbf{b}  Consensus molecular subtypes of organoids determined by RNA expression analysis. \textbf{c} Mutation status in patient derived organiods, as analyzed by amplicon sequencing. Figure created by Johannes Betge (graphical presentation),  Erica Valentini (sequencing data analysis) and Benedikt Rauscher (CMS type inference). Figure adapted from \textit{The drug-induced phenotypic landscape of colorectal cancer organoids} \parencite{betgeDruginducedPhenotypicLandscape2022}}
\label{fig_120}
\end{figure}


\subsection{Mouse Organoid Culture}
A heterozygous \textit{LSL-Kras G12D (B6.129S4-Krastm4Tyj/J)} female mouse \parencite{jacksonAnalysisLungTumor2001} was crossed with a homozygous \textit{Rosa26-CreERT2 (B6.129-Gt(ROSA)26Sortm1(cre/ERT2)Tyj/J)} male to generate offspring with a Tamoxifen activatable \textit{Kras}\textsuperscript{G12D/+} allele. A single healthy \textit{LSL-Kras\textsuperscript{G12D} CreERT2} mouse (male, 8 weeks) was sacrificed for organoid generation. \par 

Mouse colon organoids were isolated based on work by Sato et al. \parencite{satoSingleLgr5Stem2009}. After cervical dislocation of the sacrificed mouse the colon was prepared and excised between caecum and rectum. The tissue was stored on ice in cooled DPBS (Life technologies), cut open lengthwise and washed three times with DPBS. After thorough washing, colon fragments were cut into 2mm pieces and incubated in a 5mM EDTA/DPBS (Sigma) solution for 60 minutes on a rocking table at 4C. Digested fragments were allowed to settle and resuspended in DMEM/F12 (Life technologies) by repeated up- and down-pipetting with a serological pipette. Here, care was taken to pre-wet the serological pipette to avoid loss of isolated cells during liquid handling. The resulting crypt suspension was filtered with a 70ul filter (Falcon), crypts were counted and centrifuged at 150g, 10min, 4C. The resulting pellet was resuspended in 10mg/ml Matrigel (Corning) and plated on prewarmed 6-well suspension plates (Greiner). After 30-60 minutes of solidification, droplets were overlaid with complete organoid growth medium and incubated at 37C, 5\% CO2 in atmospheric air.

Complete colon organoid medium, termed WENRAS, contained 30\% advanced DMEM/F12 (Life Technologies) supplemented with 1\% v/v penicillin/streptomycin solution (Life Technologies), 1\% v/v HEPES buffer (Life Technologies) and 1\% v/v Glutamax (Life Technologies), 50\% Wnt3A conditioned medium, and 20\% R-spondin1-FC conditioned medium. 
The medium was further supplemented with recombinant Noggin (100 ng/ml), 1x B27 (1x), n-Acetyl-cysteine (1.25 mM), Nicotinamide (10 mM), EGF (50 ng/ml), 500 nM A83-01 (Tocris), SB202190 (3 μM), Y-27632 (10 µM) and Primocin (100 µg/ml). All small molecule inhibitors were dissolved in DMSO. 


\begin{table}[htbp]
\caption{WENRAS Medium Components}
\label{tab:wenras_medium_components}
\begin{tabularx}{\textwidth}{Xll}
\toprule
\textbf{Component} & \textbf{Concentration} & \textbf{Manufacturer} \\
\midrule
Wnt3A Conditioned Medium & 50\% v/v & Produced In-House \\
Advanced DMEM/F12 & 25\% v/v & Life Technologies; Carlsbad, California, USA \\
R-spondin1-FC Conditioned Medium & 20\% v/v & Produced In-House \\
B27 Supplement (50x) & 2\% v/v & Life Technologies; Carlsbad, California, USA \\
Penicillin/Streptomycin Solution & 1\% v/v & Life Technologies; Carlsbad, California, USA \\
HEPES Buffer & 1\% v/v & Life Technologies; Carlsbad, California, USA \\
GlutaMAX & 1\% v/v & Life Technologies; Carlsbad, California, USA \\
Noggin & 100 ng/ml & PeproTech; Hamburg, Germany \\
N-Acetyl-cysteine & 1.25 mM & Sigma-Aldrich; St. Louis, Missouri, USA \\
Nicotinamide (Vit. B3) & 10 mM & Sigma-Aldrich; St. Louis, Missouri, USA \\
EGF & 50 ng/ml & PeproTech; Hamburg, Germany \\
A83-01 & 500 nM & Tocris; Bristol, UK \\
SB202190 & 3 µM & Selleck Chemicals; Houston, Texas, USA \\
Y-27632 & 10 µM & Selleck Chemicals; Houston, Texas, USA \\ 
Primocin & 100 µg/ml & InvivoGen; San Diego, California, USA \\
\bottomrule
\end{tabularx}
\end{table}

\begin{table}[htb]
\caption{Cell Lines for Conditioned Media Production}
\label{tab:cellline} % Label for referencing
\begin{tabularx}{\textwidth}{Xll}
\toprule
\textbf{Reagent or Resource} & \textbf{Source}\\
\midrule
R-Spondin1 expressing 293T Cell Line & Sigma/Merck; St. Louis, Missouri, USA  \\
L Wnt-3a Cell Line & Clevers laboratory, Utrecht, Netherlands \\
HEK-293 Noggin-Fc Cell Line & Clevers laboratory, Utrecht, Netherlands \\
\bottomrule
\end{tabularx}
\end{table}

\begin{table}[htb]
\caption{Hydrogels}
\label{tab:hydrogels} % Label for referencing
\begin{tabularx}{\textwidth}{XlX}
\toprule
\textbf{Reagent or Resource} & \textbf{Manufacturer} \\
\midrule
Matrigel & Corning, Corning, NY 14831 USA \\
BME R1 & Trevigen, Gaithersburg, Maryland, USA \\
BME V2 & Trevigen, Gaithersburg, Maryland, USA \\
\bottomrule
\end{tabularx}
\end{table}

After isolation, colon organoids were cultured in solidified BME R1 (Trevigen) droplets for normal passaging and overlaid with genotype and experiment dependent growth medium. The medium was exchanged every 48-72 h. 
APC mutant colon organoid lines were cultured without Wnt and R-spondin conditioned medium, which was replaced by basal medium instead.
Organoids were passaged weekly by digestion with TrypLE (Gibco) and resuspension in BME R1 (10mg/ml). 
Organoids were regularly tested for Mycoplasma contaminations.  

\subsection{Genetic editing of organoids}
An sgRNA targeting the murine ortholog of the \textit{APC} mutation cluster region (MCR) was designed using E-CRISP \parencite{heigwerECRISPFastCRISPR2014}. The \textit{Apc} targeting sgRNA was cloned into the one-vector plasmid pSpCas9(BB)-2A-Puro (PX459) V2.0 according to a protocol by Ran et al. \parencite{ranGenomeEngineeringUsing2013}. Briefly, the vector was digested with Bbs1-HF (Thermo Fischer Scientific) and the phosphorylated and annealed oligonucleotides for sgApc1 (sgApc1 F and -R) was ligated using T4-Ligase (Thermo Fischer Scientific). The construct was transformed into chemically competent bacterial cells (Stellar, Clontech) and plated on Carbenicillin agar. Individual colonies were isolated and sequencing of plasmid DNA from cultured colonies confirmed successful molecular cloning.   
Extracted organoids (abbreviated, "wildtype" or "WT") were cultured for multiple passages before transfection of the plasmid with Lipofectamine 2000 (Thermo Fischer Scientific). For this step, grown organoids were digested with TrypLE (Gibco) and treated with Lipofectamine and plasmid DNA according to the manufacturer’s protocol. Transfected organoids were seeded in BME R1 (10mg/ml) and Wnt3A/R-Spondin1-Fc withdrawal was started 7 days after transfection. Surviving organoids were cultured continuously without Wnt3A and R-Spondin1-Fc conditioned medium.
To activate oncogenic Kras, Wildtype and $Apc$ mutant organoid lines were treated for 7 days with 0.5uM 4-Hydroxytamoxifen (Sigma) without EGF in the medium. 4-Hydroxytamoxifen was dissolved in Ethanol. After treatment, organoids were cultured with EGF containing media thereafter.

\begin{table}[htb]
\caption{List of Recombinant DNA}
\label{tab:recombinant_dna} % Label for referencing
\begin{tabularx}{\textwidth}{Xll}
\toprule
\textbf{Reagent or Resource} & \textbf{Source} & \textbf{Identifier} \\
\midrule
pSpCas9(BB)-2A-Puro (PX459) V2.0 & F. Zhang via Addgene & \# 62988 \\
\bottomrule
\end{tabularx}
\end{table}

\begin{table}[htb]
\caption{List of sgRNAs}
\label{tab:sgrna} % Label for referencing
\begin{tabularx}{\textwidth}{XlX}
\toprule
\textbf{Reagent or Resource} & \textbf{Sequence} & \textbf{Source} \\
\midrule
sgAPC1 & GGCACTCAAAACGCTTTTGA & GATC Biotech \\
\bottomrule
\end{tabularx}
\end{table}


\section{Image-based profiling}

\subsection{Patient derived organoid seeding during compound testing}
Patient dervied organoid drug profiling followed a standardized protocol with comprehensive documentation of all procedures. Organoids were collected and digested in TrypLE Express (Life technologies). Fragments were collected in basal medium with 300 U/ml DNAse and strained through a 40μm filter to achieve a homogeneous cell suspension with single cells and small clusters of cells, but without large organoid fragments. 384 well microclear assay plates (Greiner) were coated with 10μL BME V2 (Trevigen) at a concentration of 6.3 mg/ml in basal medium, centrifuged and incubated for >20 min at 300G and 37° C to allow solidification of the gel. Organoid cell clusters together with culture medium (ENA) and 0,8 mg/ml BME V2 were added in a volume of 50μl per well using a Multidrop dispenser (Thermo Fisher Scientific). Plates were sealed with a plate-loc (Agilent) and centrifuged for additional 20 min allowing cells to settle on the pre-dispensed gel. Cell number was normalized before seeding by measuring ATP levels in a 1:2 dilution series of digested organoids with CellTiter-Glo (Promega). The number of cells matching 10,000 units were seeded in each well. After seeding of organoid fragments, plates were incubated for three days at 37°C to allow organoid formation before addition of compounds. Two biological replicates of each organoid line from different passages were profiled at different time points.

\begin{figure}[h!]
\centering
\includegraphics[width=\textwidth,
                height=\textheight,
                keepaspectratio]{figures/promise/pdf/fig_0_1.pdf}
\caption[Core organoid liquid handling methods]{\textbf{Core organoid liquid handling methods a} Organoid isolation procedure. Colorectal cancer tissue biopsies were collected via endoscopy, enzymatically removed from extracellular matrix proteins, washed and resuspended in basal membrane extract hydrogel. After solidification of hydrogel domes, organoids were overlayed with growth factor rich culture medium. \textbf{b} Organoid high-throughput experimentation. Colorectal cancer organoids were harvested, partially digested, seeded in hydrogel-coated 384-well plates. Figure produced with Biorender}
\label{fig_110}
\end{figure}

\subsection{Mouse Organoid seeding during compound testing}
Mouse organoid screening was performed as described above with slight modifications. Clotting of organoid fragments was avoided by adding 10 U/ml of bovine DNAse1 to the medium during filtration. The cell viability of digested fragment suspensions was estimated using Cell-Titer-Glo (Promega). 40μl of cell suspension was mixed with 40μl of undiluted reagent and measured after 30 minutes on a Mithras plate reader (Berthold). Cell fragments with a viability corresponding to 5000 units were seeded per well on pre-coated 384 well microclear assay plates (Greiner) using a Multidrop peristaltic pump robot (Thermo Fischer Scientific) analogous to the human organoid seeding protocol outlined above. 

\subsection{Compound Libraries and Treatment for Patient derived Organoids}
Two compound libraries were used for profiling: A library containing 63 clinically relevant drugs (clinical cancer library) and a large library of 464 compounds targeting kinases and stem cell or developmental pathways associated genes (KiStem library). The clinical cancer library was manually curated by relevance for current (colorectal) cancer therapy, mechanism of action and potential clinical applicability. Compounds of this library are in clinical use or at least in phase I/II clinical trials. Five concentrations per compound were screened (five-fold dilutions). The concentrations were determined by analysis of literature data from previous 3D and 2D drug screens and own experiments. The KiStem library includes 464 compounds targeting a diverse set of kinases and stem cell relevant pathways. All compounds in this library were used in a concentration of 7.5μM. All compounds were obtained from Selleck chemicals. Compounds of both libraries were arranged in an optimized random layout. Compound libraries were stored in DMSO at -80 C.
\bigbreak
After 72 hours of expansion at 37C, medium was aspirated from all organoid screening plates and replaced with fresh ENA medium devoid of Y-27632, resulting in 45μl volume per well. Drug libraries were diluted in basal medium and subsequently 5μl of each compound was distributed to screening plates. 
All liquid handling steps were performed using a Biomek FX robotic system (Beckmann Coulter). Plates were sealed and incubated with the compounds for four days. All PDO lines underwent profiling with the clinical cancer library, while the KiStem library was used with 13 PDO lines.

\subsection{Compound Libraries and Treatment for Mouse Organoids}
Two compound libraries were used for profiling: The KiStem library (see above) as well as a library of FDA-approved small molecules. For both the KiStem library and the FDA-approved library, one single concentration was used. All compounds in this library were used in a concentration of 7.5μM. All compounds were obtained from Selleck chemicals. Compounds of both libraries were arranged in an optimized random layout and complete compound libraries were stored in DMSO at -80 C.
\bigbreak
Mouse Organoids were treated similar to Patient derived Organoids with slight modifications: After 72 hours of organoid expansion in WENRAS, the medium was changed to ENR and compound libraries were added using a BiomekFX (Beckmann Coulter). All Mouse Organoids were profiled with the KiStem library and FDA-approved compound library.

\subsection{Automated Microscopy}
After 96h of small molecule treatment, Image-IT DeadGreen (Thermo Fisher) was added to the cultures with a Multidrop dispenser (Thermo Fisher) in 100nM final concentration and incubated for four hours. Afterwards, medium was removed and organoid cultures were fixed with 3\% PFA in PBS with 1\% BSA. Fixed plates were stored at 4° C for up to three days before permeabilization and staining. On the day of imaging, organoids were permeabilized with 0.3\% Triton-X-100 and 0.05\% Tween in PBS with 1\% BSA and stained with 0.1μg/ml TRITC-Phalloidin (Sigma) and 2μg/ml DAPI (Sigma). All liquid handling steps were performed with a BiomekFX. Screening plates were imaged with an Incell Analyzer 6000 (GE Healthcare) line-scanning confocal fluorescent microscope. We acquired 4 fields per well with z-stacks of 16 slices at 10x magnification. The z-steps between the 16 slices had a distance of 5μm, the depth of field of each slice was 3.9μm.

\subsection{Luminescence Viability Read Out of Patient Derived Organoids}
Organoid screening plates undergoing ATP-based viability testing were cultured in solid, white plates (Greiner) treated with 30μl CellTiter-Glo reagent after medium aspiration with a Biomek FX. After incubation for 30 minutes, luminescence levels were measured with a Mithras reader (Berthold technologies).

Raw luminescence data of each plate were first normalized using the Loess-fit method in order to correct for edge effects where increased luminescence intensity was observed along the edges of each plate. Subsequently, each plate was normalized by division with the median luminescence intensity of the DMSO controls. Drug response Hill curves (DRC) were fitted and area under the curve values were calculated for each DRC using the ‘PharmacoGx’ \parencite{smirnovPharmacoGxPackageAnalysis2016} R/Bioconductor package.

\subsection{Luminescence Viability Read Out of Patient Derived Organoids of Mouse Organoids}
For selected compounds, before compound addition, organoid viability was measured using Cell-Titer-Glo (Promega). Cell viability was measured after compound exposure as described above. The pre-treatment viability of organoids was used to estimate growth-rate controlled dose-response curves according to \parencite{hafnerGrowthRateInhibition2016} Measurements of GR metrics for were not robust for slow proliferating WT lines. Therefore, these lines were omitted in the analysis.

\section{Image analysis}

\subsection{Image Processing}
Microscopic image z-stacks were compressed to HDF5 format for archival and underwent maximum contrast projection using the R/Bioconductor package MaxContrastProjection developed by Jan Sauer for further processing of the images. Standard image features, including shape, moment, intensity, and Haralick texture features on multiple scales, were extracted using the R/Bioconductor package EBImage \parencite{pauEBImagePackageImage2010}. Of note, the strong diversity of unperturbed organoid phenotypes between organoid lines did not allow the definition of a core set of individual reproducible descriptive features across all screened organoids. Therefore, no correlation-based filtering of features was done, allowing comparisons between different lines. Out-of-focus objects were manually and programmatically removed from the dataset using a custom classifier developed by Jan Sauer.

\subsection{Feature Processing and Treatment-Induced Phenotypes}
The dimensionality of calculated single-organoid morphological features was reduced using a principal component analysis (PCA) that was performed on the entire datasets (patient derived and mouse organoids separately) to reduce the dimensionality. 25 principal components were selected, explaining ca. 80\% of the total variance within both datasets. These 25 principal components were used for further analysis of single-organoid morphology and were averaged on a per-well level for the MOFA analysis.
\bigbreak
For treatment activity estimation, a logistic regression classifier was trained per line and treatment (and per concentration where applicable) to differentiate individual treated organoids from negative controls based on the PCA-transformed features. Drugs were categorized as either active or inactive based on the accuracy of the model. The direction of the decision plane's normal vector was used as the treatment effect vector. Drugs were clustered based on the cosine similarity of their normal vectors.


\section{Multi-omics factor analysis}
\subsection{Model training of Patient Derived Organoids}
A multi-omics factor analysis model \parencite{argelaguetMultiOmicsFactorAnalysis2018b} was trained based on a set of five modalities describing unperturbed organoid lines:

\begin{itemize}
    \item organoid size estimated based on log-normal model fit of all DMSO treated organoids [22 replicates, 1 feature]
    \item organoid somatic mutations as determined by amplicon sequencing [20 replicates, 12 features]
    \item organoid transcript expression of the top 10\% genes with the highest coefficient of variance after robust multi-array average normalization [22 replicates, 3222 features]
    \item organoid morphology as determined by averaging DMSO treated morphological profiles [22 replicates, 25 features]
    \item organoid drug activity as determined by AUROC score of logistic regression models for drugs that were defined as active in at least one observation [22 replicates, 252 features]
\end{itemize}


Input data was scaled and the MOFA model was trained with default MOFA2 training parameters and a number of 3 factors. The number of factors was chosen given the limited number of observations in the training data. The further analysis focused on the first two factors, which correlated with prominent visible organoid phenotypes. Gene set enrichment analysis and Reactome pathway enrichment of factor loadings was performed using the clusterprofiler R package \parencite{yuClusterProfilerPackageComparing2012}. Enrichment of drug targets within factor loadings was tested using ANOVA by fitting a linear model, \textit{lm(factor loading vs. target)}. Drug targets that were represented with at least three small molecule inhibitors were included in this analysis. The analysis was run using the MOFA docker container available from https://hub.docker.com/r/gtca/mofa2.

\subsection{Model training of Mouse Organoids}
A multi-omics factor analysis model with k=4 factors was trained and results were analysed as above, with a different set of input data: 
\begin{itemize}
    \item organoid size all DMSO treated organoids (one replicate of Apc-/- organoids was removed from the analysis) [7 replicates, 1 feature]
    \item organoid transcript expression including the top 10\% genes with the highest coefficient of variance after robust multi-array average normalization [8 replicates, 2727 features]
    \item organoid protein expression [12 replicates, 3906 features]
    \item organoid lipid abundance [12 replicates, 397 features]
    \item organoid genotype for the $Apc$ and KrasG12 allele [12 replicates, 2 features]
    \item organoid morphology as determined by averaging DMSO treated morphological profiles (one replicate of Apc\textsuperscript{-/-} organoids was removed from the analysis) [7 replicates, 25 features]
    \item organoid drug activity as determined by AUROC score of logistic regression models for drugs [7 replicates, 1699 features]
\end{itemize}


\subsection{Model projection}
To estimate the factor scores for treatment-induced organoid morphologies, the morphology feature matrix was multiplied with the pseudoinverse of the previously learnt model's factor weight matrix for the organoid morphology modality. 

\smallbreak
The resulting projected factor score matrix was used to estimate the drug-induced biological changes in both patient derived and mouse organoids. Associations between drug targets and projected factor scores of drug treated organoids were identified via ANOVA by fitting a linear model, \textit{lm(projected factor score vs. target)}. For mouse organoids, the model was extended to account for line-wise effects in the modeling, \textit{lm(projected factor score vs. target + organoid-line)}. Drug targets that were represented with at least three small molecule inhibitors were included in the analysis, except in the modeling of individual small molecule effects on mouse organoids.


\section{Biochemical assays}

\subsection{Amplicon Sequencing of Mouse Organoids}
Amplicon sequencing was performed to validate the CRISPR perturbation of \textit{Apc}. DNA from \textit{Apc} targeted and untargeted organoid lines was prepared using the DNA Blood and Tissue Kit (Qiagen), according to the manufacturer’s tissue protocol including an RNAse digestion. The targeted region was PCR amplified using primers F1 and R1. Libraries were sequenced on a MySeq (Illumina) using 100bp single end reads. 

\begin{table}[htb]
\caption{List of Genomic PCR Primers for Amplicon Sequencing}
\label{tab:oligonucleotides} % Label for referencing
\begin{tabularx}{\textwidth}{XlX}
\toprule
\textbf{Reagent or Resource} & \textbf{Sequence} & \textbf{Source} \\
\midrule
Primer F1 & TCCCTACACGACGCTCTTCCGATCTGGAATGTCAGAAGGGAGACC & GATC Biotech \\
%Primer F2 & TCCCTACACGACGCTCTTCCGATCTGAGGAATGTCAGAAGGGAGA & GATC Biotech \\
Primer R1 & AGTTCAGACGTGTGCTCTTCCGATCTCCAACCAGAAATGCCAGTG & GATC Biotech \\
%Primer R2 & AGTTCAGACGTGTGCTCTTCCGATCTGCCAACCAGAAATGCCAGT & GATC Biotech \\
\bottomrule
\end{tabularx}
\end{table}

\subsection{Genomic PCR of the KRAS G12D allele in Mouse Organoids}
To confirm activation of oncogenic $Kras$ in 4-Hydroxytamoxifen treated lines, genomic DNA was isolated from all 4 organoid lines as described above. Presence or absence of the Lox-STOP-Lox cassette was evaluated by PCR according to the $Kras^{G12D/+}$ conditional PCR protocol by Tyler Jacks’ group \parencite{jacksonAnalysisLungTumor2001}. Briefly, primers \#2 and \#3 were used for genotyping on genomic DNA using the Q5 PCR protocol (NEB).

\begin{table}[htb]
\caption{List of Primers for validating KRAS conditional PCR}
\label{tab:kras} % Label for referencing
\begin{tabularx}{\textwidth}{XlX}
\toprule
\textbf{Reagent or Resource} & \textbf{Sequence} & \textbf{Source} \\
\midrule
Primer \#2 & CTCTTGCCTACGCCACCAGCT & GATC Biotech \\
Primer \#3 & AGCTAGCCACCATGGCTTGAGTAAGTCTGCA & GATC Biotech \\
\bottomrule
\end{tabularx}
\end{table}

\subsection{Western Blot of Patient derived Organoids}
Organoids seeded in 6-well plates were cultured in Matrigel (Corning). After 3-days incubation with WYE-132, organoids were collected, and cells were isolated using Matrisperse (Corning) for 40 minutes on a rocking table. Cells were subsequently lysed in RIPA buffer (Thermo Fisher Scientific) supplemented with Protease inhibitors (Complete Mini, Roche) and Phosphatase inhibitors 1 and 2 (Sigma), followed by sonication (Branson Sonifier, Heinemann). Protein concentrations of supernatants were measured using the Pierce BCA kit (Thermo Fisher Scientific) according to the manufacturers protocol. Lysates were mixed with an SDS-loading buffer and heated to 99 C for 5 minutes. Proteins were separated by SDS–PAGE in MOPS running buffer and transferred to a nitrocellulose membrane. Membranes were blocked with 5\% (w/v) skim milk in PBS containing 0.1\% (v/v) Triton X-100 (PBS-T). Western Blotting was performed with following antibodies (all in 5\% (w/v) skim milk, PBS-T): anti-IRS1 (1:1000, 06-248, Sigma–Aldrich), anti-HSP-90 (1:1000, sc-13119, Santa Cruz), anti-Mouse-IgG-HRP (1:10000, Sigma–Aldrich). ECL Western Blotting W1001 (Promega) was used for visualization of bands.

\begin{table}[htb]
\caption{List of Antibodies used in experiments}
\label{tab:antibodies} % Label for referencing
\begin{tabularx}{\textwidth}{Xll}
\toprule
\textbf{Reagent or Resource} & \textbf{Source} & \textbf{Identifier} \\
\midrule
Anti-IRS1 (rabbit) & Sigma–Aldrich & 06-248 \\
Anti-Erk (p44/42 MAPK) (rabbit) & Cell Signaling & 4695 \\
Anti-phospho-Erk (phospho-p44/42 MAPK) (rabbit) & Cell Signaling & 4370 \\
Anti HSP-90 (mouse) & Santa Cruz & 13119 \\
Anti-Mouse-IgG-HRP (goat) & Sigma–Aldrich & 12-349 \\ 
Anti-beta-actin HRP (mouse) & Cell Signaling & 12262 \\
\bottomrule
\end{tabularx}
\end{table}

\subsection{Western Blot of Mouse Organoids}
Organoids were cultured in Matrigel (Corning). Organoids were collected, and cells were isolated using Matrisperse (Corning) for 40 minutes on a rocking table. Isolated organoids were lysed in RIPA buffer (Sigma) with Protease inhibitor (Sigma) and Phosphatase inhibitor 3 (Sigma). Protein concentration was measured using the Pierce BCA kit (Thermo Fischer Scientific) according to the manufacturers protocol. Lysates were mixed with an SDS-loading buffer and heated to 99 C for 5 minutes. Samples were loaded onto NuPage gels (Thermo Fischer Scientific), separated in MOPS running buffer and transferred to a nitrocellulose membrane. Membranes were blocked with 5\% (w/v) skim milk in PBS containing 0.1\% (v/v) Triton X-100 (PBS-T). Western Blotting was performed with following antibodies (all in 5\% (w/v) skim milk, PBS-T): anti-p(hospho)-Erk (1:2000, Cell Signaling, ID 4370), anti-Erk (1:1000, Cell Signaling, ID 4695) and anti-beta-actin-HRP secondary antibody (1:150,000).

\subsection{Mouse Organoid Growth Patterns}
Organoids were passaged and seeded in 4 different growth media with medium changes every 48h. Images were taken 120h after seeding with 4x magnification on a Zeis Axiovert (Zeis). 

\subsection{RT-qPCR of Patient derived Organoids}
Organoids seeding and treatment timing was performed in accordance with the image-based profiling protocol. After 120h of growth, total RNA was isolated using the RNAEasy Mini kit (Qiagen) without additives. cDNA synthesis was done with Verso cDNA kit (Thermo Fisher Scientific), and RT-qPCR was performed using the SYBR Green Mix (Roche) on a LightCycler480 system (Roche). UBC expression levels were used as controls.

\begin{table}[htb]
\caption{List of RT-qPCR Primers}
\label{tab:qpcr} % Label for referencing
\begin{tabularx}{\textwidth}{lXll}
\toprule
\textbf{Reagent or Resource} & \textbf{Organism} & \textbf{Sequence} & \textbf{Source} \\
\midrule
LGR5 F & Human & TTCCCAGGGAGTGGATTCTAT & GATC Biotech \\
LGR5 F & Human & ACCAGACTATGCCTTTGGAAAC & GATC Biotech \\
UBC F & Human & CTGATCAGCAGAGGTTGATCTTT & GATC Biotech \\
UBC R & Human & TCTGGATGTTGTAGTCAGACAGG & GATC Biotech \\
Axin2 F & Mouse & GAGAGTGAGCGGCAGAGC & GATC Biotech \\
Axin2 R & Mouse & CGGCTGACTCGTTCTCCT & GATC Biotech \\
Ccnd F & Mouse & TTTCTTTCCAGAGTCATCAAGTGT & GATC Biotech \\
Ccnd R & Mouse & TGACTCCAGAAGGGCTTCAA & GATC Biotech \\
Sdha F & Mouse & TGTTCAGTTCCACCCCACA & GATC Biotech \\
Sdha R & Mouse & TCTCCACGACACCCTTCTGT & GATC Biotech \\
Hprt F & Mouse & CCTCCTCAGACCGCTTTTT & GATC Biotech \\
Hprt R & Mouse & CCTCCTCAGACCGCTTTTT & GATC Biotech \\
\bottomrule
\end{tabularx}
\end{table}

\subsection{RT-qPCR of Mouse Organoids}
Organoids were passaged and seeded in 4 different growth media with medium changes every 48h. After 120h, organoid RNA was isolated using the RNAEasy Kit (Qiagen) with beta-Mercaptoethanol (Invitrogen) and a DNAse digestion step. cDNA was synthesized using Oligo-dT primers (Thermo Fisher Scientific), RiboLock Ribonuclease inhibitor (Thermo Fisher Scientific) and Revert Aid H Minus reverse transcriptase (Thermo Fisher Scientific). RT-qPCR was performed using the ROCHE UPL kit (Roche) on a LightCycler480 system (Roche). Sdha and Hprt expression levels were used as controls and averaged. Relative transcript abundance was measured using the ddCT method.
 
\subsection{Proteomics Profiling of Mouse Organoids}
Organoids were cultured according to the image-based profiling protocol. Samples were isolated with Matrisperse (Corning) as described above. Isolated organoids were lysed in Ammonium Bicarbonate lysis buffer (50mM, pH 8.2) with 2.5\% w/v SDC and 25U/ml Benzonase. 
\bigbreak
Samples were handed over to the Proteomics core facility at the German Cancer Research Center. There, a "1D-SDS-PAGE of the lysate was run followed by fractionation. Gel pieces were extracted, cysteins residues reduced by DTT and carbamidomethylated using iodoacetamide. The samples were digested with Trypsin overnight.
Resulting peptides were loaded on a cartridge trap column, packed with Acclaim PepMap300 C18, 5µm, 300Å wide pore (Thermo Fischer Scientific) and segregated in a 60 min gradient from 3\% to 40\% ACN on a nanoEase MZ Peptide analytical column (300Å, 1.7 µm, 75 µm x 200 mm, Waters). Eluted peptides were analyzed by an online coupled Q-Exactive-HF-X mass spectrometer" (\textit{quoted description taken from direct correspondence with the core facility}).
\bigbreak
Protein abundance data were processed using the DEP package \parencite{zhangProteomewideIdentificationUbiquitin2018} accessed through the R/Bioconductor ecosystem \parencite{huberOrchestratingHighthroughputGenomic2015}. After removing missing LFQ values, ca. 3000 protein fragments were observed across all 12 conditions. Data were normalised using VSN variance stabilisation \parencite{huberVarianceStabilizationApplied2002} and missing values imputed. Batch effects were removed using Combat \parencite{leekCapturingHeterogeneityGene2007}. 
Pearson correlation between Protein and Transcript abundance ranged between 0.3-0.36 across four organoid lines. 

\subsection{Lipidomics Profiling of Mouse Organoids}
Organoids were cultured according to the the image-based profiling protocol and prepared for analysis according to the method developed for proteomics profiling (above). 
Isolated organoids were processed by the Lipidomics and Metabolomics core facility. 
"After lipid extraction, isolates were analysed using a Qtrap 6500 mass spectrometer (SCIEX) coupled to a NanoMate (NanoMate) electrospray source" (\textit{quoted description taken from direct correspondence with the core facility}).
\bigbreak
Lipid abundance data was processed using the DEP package \parencite{zhangProteomewideIdentificationUbiquitin2018} accessed through the R/Bioconductor ecosystem \parencite{huberOrchestratingHighthroughputGenomic2015}. After removing missing values, ca. 350 lipid species were observed across all 12 conditions. Data were normalised using VSN variance stabilisation \parencite{huberVarianceStabilizationApplied2002} and missing values imputed. Batch effects were removed using "Combat" \parencite{leekCapturingHeterogeneityGene2007}. 

\subsection{Transcript Expression Profiling of Patient derived Organoids}
Organoid RNA was isolated from 19 patient derived organoid lines with the RNeasy mini kit after snap freezing organoids on dry ice. Samples were hybridized on Affymetrix U133 plus 2.0 arrays. 
\bigbreak
Raw microarray data were normalized using the robust multi-array average (RMA) method \parencite{irizarryExplorationNormalizationSummaries2003} followed by quantile normalization as implemented in the "affy" \parencite{gautierAffyAnalysisAffymetrix2004} R/Bioconductor \parencite{huberOrchestratingHighthroughputGenomic2015} package. Differential gene expression analyses were performed using a moderated t-test as implemented in the R/Bioconductor package "limma" \parencite{ritchieLimmaPowersDifferential2015}. Gene set enrichment analyses were performed using ConsensusPathDB \parencite{kamburovConsensusPathDBMoreComplete2011} for discrete gene sets or GSEA as implemented in the "fgsea" R/Bioconductor package for ranked gene lists. 

\subsection{Transcript Expression Profiling of Mouse Organoids}
Mouse organoids were cultured according to the image-based profiling protocol. Briefly, organoid models were seeded and cultured for 72h in WENRAS and additional 96h in ENR. Samples were harvested after 7 days and RNA was isolated using the RNAEasy Kit (Qiagen) as described above. Transcript expression levels were measured using MoGene-2\_0-st chips (Affymetrix).
\bigbreak
Raw microarray data processing was generally performed analogous to the analysis of Patient derived Organoids. With a series of modifications, including the use of "Combat" \parencite{leekCapturingHeterogeneityGene2007} to remove batch after robust multi-array averaging (RMA) \parencite{irizarryExplorationNormalizationSummaries2003}, removing the lowest 5\% of observed probes, and mapping of mouse genes to human orthologs. Gene set enrichment analyses were performed using methods described above as well as using the "clusterProfiler" package \parencite{yuClusterProfilerPackageComparing2012}. 


\end{flushleft}
\begin{savequote}[75mm]
What I cannot create, I do not understand.
\qauthor{Richard Feynman}
\end{savequote}

% pending plagiarism check
\begin{flushleft}
\chapter{Image-based profiling to identify revertant therapeutics in pre-malignant models of colon cancer}


\end{flushleft}
\begin{savequote}[75mm]
What I cannot create, I do not understand.
\qauthor{Richard Feynman}
\end{savequote}

% pending plagiarism check
\begin{flushleft}
\chapter{Profiling to identify revertant therapeutics in pre-malignant models of colon cancer}

\section{Motivation}

In the previous chapter I demonstrated that interpretable multi-view representations from organoid morphology and molecular measurements can be learned. Furthermore, previously unobserved small molecule induced morphological changes can be projected into the learnt representation space to aid the interpretation of these small molecule effects and generate hypotheses for further validating experiments. In this chapter I built upon these observations and explored to what extent the projection of small molecule induced morphological changes could be applied for therapeutic discovery scenarios. 

I generated a set of four genetically engineered mouse colon organoid lines that model the initiation of colorectal cancer by harboring tumorigenic Apc and Kras mutations in isolation and combination. The organoid models were then characterised using transcript expression, proteomics and lipidomics measurements. After their characterisation, a high-throughput image-based profiling experiment covering ca 1700 FDA-approved, natural and targeted small molecules was performed.

The goal of the image-based profiling experiment was to identify treatments that shift organoids between pre-malignant and wildtype states \ref{fig_a02}. As mentioned above, the chosen Ansatz to identify such phenotype-shifting treatments was to project the morphology of treated organoids into a previously learnt multi-view representation where movement along well-annotated biological axes of variation can readily be identified.  

\begin{figure}[h]
\centering
\includegraphics[width=250,
                height=\textheight,
                keepaspectratio]{figures/adenomaprofiling/pdf/fig_0_2.pdf}
\caption[Visual abstract of adenoma model profiling project]{\textbf{Visual abstract of adenoma model profiling project.} Mouse colon organoid models were developed, characterized and subjected to image-based small molecule profiling. The resulting data was used to learn a representation space of organoid states, identify state dependent changes in organoid lipid composition and generate hypotheses about small molcules that potentially have the ability to move organoid states.}
\label{fig_a02}
\end{figure}
\bigbreak



\section{Generation and characterization of organoid models}
\subsection{Generation of organoid colon adenoma models}
The emergence of colorectal cancer via the chromosomal instability process is a well understood sequence of genetic events that start with hyperactivation of canonical Wnt signaling, i.e. through truncating mutations of APC, followed by the hyperactivation of RAS-MAPK signaling, i.e. via oncogenic mutations of KRAS. I genetically engineered mouse colon organoid models carrying Apc truncating mutations and/or a Kras G12D allele, thereby modelling the first set of genetic events within the chromosomal instability process leading to colorectal cancer. 

\begin{figure}[H]
\centering
\includegraphics[width=\textwidth,
                height=\textheight,
                keepaspectratio]{figures/adenomaprofiling/pdf/fig_1_0.pdf}
\caption[Establishing organoid models of colon adenoma]{\textbf{Establishing organoid models of colon adenoma, a} Overview of organoid model establishment. Mouse colon organoids were isolated from a transgenic donor animal carrying an inactive conditional oncogenic KrasG12D allele. Homozygous truncation of Apc via CRISPR and activation of the heterozygous KrasG12D allele lead to four different genetically defined organoid models.
\textbf{b} In vitro growth factor dependency of adenoma models. Organoids were cultured in complete or modified medium containing combinations of Wnt3A, R-Spondin1-Fc and EGF for 120h and subsequently imaged. Scalebar = 200µm.
\textbf{c}	Oncogenic KrasG12D increases resistance to Egfr inhibition. Organoid ATP levels were measured 4 days after Gefitinib treatment and adjusted for organoid growth rate. Points represent mean of n=2 independent experiments. Error bars represent standard error of mean.
\textbf{d} Erk phosphorylation is increased by oncogenic KrasG12D. Organoid models were cultured with or without Wnt3A and R-Spondin1-Fc for 72h and analyzed for protein levels. p, phospho.   
\textbf{e}	Loss of Apc induces transcription of canonical Wnt-signaling target genes. qRT–PCR for Axin2 and Ccnd in the presence or absence of Wnt 3a and R-spondin1-Fc after 120h of culture. Expression levels are normalized to Sdha and Hprt transcript abundance. Bar graphs represent the mean of n=4 independent experiments. Wilcoxon rank sum test 
}
\label{fig_a10}
\end{figure}
\bigbreak

To model the formation of colon adenomas in vitro, I used a transgenic mouse to derive organoid cultures. The transgenic animal carried a conditional tamoxifen inducible KrasG12D/+ allele \citep{Jackson2001-wv} (Figure \ref{fig_a10} a). After isolation, I confirmed that extracted colon organoids did not express an activated form of KrasG12D (Figure \ref{fig_a11}a) and defined these organoids as wildtype (WT). To model loss-of-function mutations of the tumor suppressor Apc, the ortholog of the frequently mutated mutation-cluster-region on the APC gene was targeted by CRISPR (Figure \ref{fig_a10} a). Generated organoids harbored biallelic loss-of-function mutations in Apc (Figure \ref{fig_a11}a). Subsequent activation of oncogenic KrasG12D by treatment with 4-Hydroxytamoxifen led to four distinct organoid adenoma models (Figure \ref{fig_a10} a and Figure \ref{fig_a11}a-b); wildtype (WT), Apc-/- (A), KrasG12D/+ (K), and Apc-/- / KrasG12D/+ (AK).


\begin{figure}[h]
\centering
\includegraphics[width=\textwidth,
                height=\textheight,
                keepaspectratio]{figures/adenomaprofiling/pdf/fig_1_1.pdf}
\caption[Structural validation of organoid colon adenoma models]{\textbf{Structural validation of organoid colon adenoma models, a} Allele-specific PCR products of colon organoid models isolated from a transgenic mouse with a conditional tamoxifen inducible KrasG12D/+ allele.
\textbf{b} Amplicon sequencing result of the murine mutation cluster region ortholog for organoids transfected with an Apc targeting sgRNA and Cas9 carrying plasmid. The sequencing results show the presence of 3 different insertion/deletions within the pool of sgRNA treated organoid models. Wildtype sequences are absent within the CRISPR targeted pool, while mutant sequences are absent in the untreated organoid pool.}
\label{fig_a11}
\end{figure}
\bigbreak

Similar to genetically modified human colon organoids (\citet{Drost2015-ph}; \citet{Matano2015-zw}), murine colon organoids showed characteristic niche requirements. Both Apc mutant organoid lines grew independent of the Wnt-signaling activating factors Wnt 3a and R-Spondin1 (Figure \ref{fig_a10} b) and at an accelerated proliferation rate. In fact, Apc mutant lines showed an increased growth in a Wnt3a and R-Spondin1 free environment when compared to the complete medium. Organoid models with an activated KrasG12D allele were less sensitive to removal of EGF from the media. However, as observed before \citep{Drost2015-ph}, the mutant KrasG12D allele was insufficient to compensate completely for the loss of EGF from the medium. Nevertheless, KrasG12D mutant organoid lines were more resistant to pharmacological inhibition of Egfr signaling (Figure \ref{fig_a10} c). In conclusion, organoid model genotypes were reflected in characteristic growth factor dependencies in experimental conditions.

\bigbreak
Next, I investigated the effects of mutations in Apc and Kras on both canonical Wnt- and Erk dependent signaling. While the presence of the KrasG12D/+ allele led to an increase in Erk-phosphorylation across models, Apc-/- / KrasG12D/+ organoids showed no marked additional increase in Erk-phosphorylation when compared to KrasG12D/+ organoids (Figure \ref{fig_a10} d). Moreover, Apc-/- / KrasG12D/+ adenoma models showed no significant differences in expression of the Wnt target genes Axin2 and Ccnd when compared to Apc-/- single-mutant models (A) (p > 0.34 for all conditions, Wilcoxon rank sum test) (Figure \ref{fig_a10} e). These results indicate that organoid adenoma models show genotype-dependent activity of characteristic signaling pathways, while there is no extensive crosstalk between the Apc-/-  and KrasG12D/+ allele in mouse colon organoids that that is directly reflected in canonical Wnt- and Erk dependent signaling.  

\bigbreak
\subsection{Molecular characterisation of organoid models}
To comprehensively characterise the four organoid models, transcriptome, proteome and lipidome profiling were performed using mRNA microarrays and mass spectrometry, respectively (Figure \ref{fig_161}). For these measurements organoids were kept in the same culture condition and duration that were used during the subsequent image-based profiling experiment: After passaging, all organoids were kept in Wnt 3a, R-Spondin1 rich medium (WENRAS) to model conditions within the niche and stimulate outgrowth before the medium was changed to a Wnt 3a-free medium (ENR) to model conditions outside the niche. The medium was supplemented with EGF both before and after the medium change. Transcriptome profiling of organoid models showed an increased expression of the stem-cell marker Lgr5 and negative Wnt-signaling regulators such as Nkd1, Notum, Wif1 and Znrf3 in Apc mutant organoid lines (Figure \ref{fig_160} b). In contrast, Apc wildtype organoid lines showed an increased expression of epithelial differentiation markers, such as Krt20, Alpp and Abcb1 (P-glycoprotein). Overall, the number of genes with significant expression changes after Apc loss was 2.5 times greater compared to isolated KrasG12D activation (FDR = 0.1, Apc-/-: 44.5\%, KrasG12D/+: 18.3\% of assessed genes). A related observation was made during the analysis of protein abundance. Again, Wnt signaling regulators (Axin2, Notum) were enriched in Apc mutant organoid lines and the number of significantly regulated proteins after Apc loss was 2.5 times greater compared to an isolated KrasG12D activation (FDR = 0.1, Apc-/-: 260, KrasG12D/+: 105 assessed proteins). Principal component analysis of both transcriptome, proteome and lipidome data showed related axes of variation across measurements. In all observed modalities, the first principal component captured differences between Apc wildtype and Apc mutant organoid models, while the second (in case of proteomics measurements the third) principal component captured differences between wildtype and KrasG12D/+ single-mutant models (Figure \ref{fig_161}b, \ref{fig_161}c and \ref{fig_161}d). In every modality, a high degree of similarity was observed among Apc-/- and Apc-/- / KrasG12D/+ organoid lines. While activation of oncogenic KrasG12D in wildtype organoids led to global changes in transcript, protein and lipid expression, these changes were not as pronounced in organoids without functional Apc. In fact, only the mRNA expression of 91 genes was significantly altered between Apc-/- and Apc-/- / KrasG12D/+ organoids (FDR = 0.1). 

\begin{figure}[H]
\centering
\includegraphics[width=\textwidth,
                height=\textheight,
                keepaspectratio]{figures/adenomaprofiling/pdf/fig_1_6_1_2.pdf}
\caption[Molecular characterisation of organoid adenoma models]{\textbf{Molecular characterisation of organoid adenoma models. a} Differential gene expression of adenoma models. Shown are scaled expression values for the top 125 differentially expressed genes for every organoid line. Selected genes are highlighted. All organoids were cultured for 3 days in WENRAS before exposure to ENR for 4 days. Cell number was controlled between experiments. Whole organoid lysates were analyzed. 
\textbf{b} Transcript abundance data. Shown are the first two principal components of scaled gene expression data. The proportion of variance of each principal component is listed in parenthesis. 
\textbf{c} Protein abundance data. Shown are the first and third principal component of scaled protein expression data. The proportion of variance of each principal component is listed in parenthesis. 
\textbf{d} Lipid species abundance data. Shown are the first two principal components of scaled lipid abundance data. The proportion of variance of each principal component is listed in parenthesis. 
\textbf{e} Loss of Apc leads to increased expression of proliferation and intestinal stem cell associated genes. Shown is a gene set enrichment analysis of differentially expressed genes between Apc mutant and WT organoids. Intestinal gene expression signatures were used according to Merlos-Suarez et al. NES, normalized enrichment score. 
\textbf{f} Overview of cellular processes in organoid adenoma models. Shown are selected enriched differential gene expression signatures from Reactome and Merlos-Suarez et al. NES, normalized enrichment score. NES > 0 suggests an enriched/ activated biological process. FDR < 0.1.}
\label{fig_161}
\end{figure}
\bigbreak

To explore active biological processes, gene set enrichment analysis on organoid transcript expression data was performed. The strongest changes in expression after loss of Apc were linked to an increased proliferation rate (Figure \ref{fig_160} e). Gene set enrichment analysis of published intestinal cell-proliferation and stem cell signatures showed an enrichment of both signatures in Apc-/- organoids (Figure \ref{fig_160} e) \citep{Merlos-Suarez2011-gd}. In contrast, a signature for differentiating, transit-amplifying cells was depleted. Gene set enrichment analysis of Apc-/- / KrasG12D/+ double-mutant organoids showed the same results. Next to these published signatures, I explored the enrichment of curated gene sets from the Reactome database \citep{Griss2020-qi}. Here, both Apc-/- and Apc-/- / KrasG12D/+ double-mutant lines showed a positive enrichment of cell cycle and DNA repair related genes when compared to wildtype organoids (Figure \ref{fig_162}a). Unique to the KrasG12D/+ organoid line was a decreased expression of citric acid cycle and respiratory chain related genes (Figure \ref{fig_162}b). This effect, was not observed in Apc-/- / KrasG12D/+ double mutant organoids (Figure \ref{fig_161}f). In addition, organoid models with an KrasG12D/+ genotype showed a downregulation of the EGFR receptor, in line with a potential negative feedback response to hyperactivated RAS-MAPK signaling (Figure \ref{fig_162}b). Both Apc-/- and KrasG12D/+ organoid models showed a strong reduction of lipid metabolism and beta-oxidation (Figure \ref{fig_162}a,b). In summary, (1) loss of Apc leads to a global shift in transcript, protein, and lipid composition in colon organoids, including a strong increase in cell proliferation associated transcripts; (2) Activation of isolated oncogenic KrasG12D leads to pronounced reduction in citric acid cycle related transcripts while this phenotype was not seen in organoid models with an additional loss of Apc; (3) Both Apc loss and activation of oncogenic KrasG12D lead to a reduction of lipid beta-oxidation related transcripts.

\begin{figure}[h]
\centering
\includegraphics[width=\textwidth,
                height=\textheight,
                keepaspectratio]{figures/adenomaprofiling/pdf/fig_1_6_2.pdf}
\caption[Representative up and down-regulated transcriptional processes after loss of Apc and activation of oncogenic Kras G12D]{
\textbf{a} Representative up and down-regulated transcriptional processes after loss of Apc. Expression signatures were sourced from Reactome and average log2 fold changes for included transcripts are illustrated. FDR < 0.1.
\textbf{b} Representative up and down-regulated transcriptional processes after activation of oncogenic Kras G12D. Expression signatures were sourced from Reactome and average log2 fold changes for included transcripts are shown. FDR < 0.1.
}
\label{fig_162}
\end{figure}
\bigbreak

To further understand the pronounced changes in fatty acid metabolism observed in both Apc-/- and KrasG12D/+ organoid models, differences in lipid composition were measured using untargeted lipid extraction and Mass Spectrometry. In total, more than 350 lipids from 15 species were identified across all samples (Figure \ref{fig_168} a). The majority of identified lipids had fatty acid chain lengths of 20 to 40 carbon atoms, while Triglycerides (TAG) had an increased length of 40 to 60 carbon atoms (Figure \ref{fig_168} b). Major differences between organoid lines were seen especially for storage lipids - Triglycerides (TAG) and Cholesterol Esters (CE) (Figure \ref{fig_168} c and d, respectively). Both lipid species were more abundant in the two Apc-/- organoid lines (p < 0.05, TAG: t > 4.8, CE: t > 3.7, ANOVA). In single-mutant KrasG12D/+ organoids, Triglycerides were also more abundant compared to wildtype organoids (p < 0.05, t = 5.9, ANOVA), while Cholesterol esters were depleted (p < 0.05, t = -3.7, ANOVA). The increase in the abundance of storage lipids as a result of Apc loss of function and oncogenic KrasG12D mutation were aligned with the transcriptional changes that indicated a reduced rate of beta-oxidation in these models (Figure \ref{fig_162}a,b). 

\begin{figure}[h]
\centering
\includegraphics[width=350,
                height=\textheight,
                keepaspectratio]{figures/adenomaprofiling/pdf/fig_1_6_8.pdf}
\caption[Lipid composition changes across organoid adenoma models]{\textbf{Lipid composition changes across organoid adenoma models. a} Lipid species abundance of adenoma models. Shown are scaled abundance values for major lipid species for every organoid line. Selected lipid species are highlighted (Lipid Maps Abbreviations: HexCer - Hexosylceramide; PA - Phosphatidate; Cer - Ceramide; SM - Sphingomyelin; LPC - Lysophosphatidylcholine; PS - Phosphatidylserine; PE - Phosphatidylethanolamine; PI - Phosphatidylinositol; PG - Phosphatidylglycerol; DAG - Diacylglycerol; PC - Phosphatidylcholine; TAG - Triacylglycerol; Hex2Cer - Dihexosylceramide; CE - Cholersterolester, Chol - Cholesterol). All organoids were cultured for 3 days in WENRAS before exposure to ENR for 4 days. Cell number was controlled between experiments. Whole organoid lysates were analyzed. 
\textbf{b} Fatty acid chain lengths by lipid species. Shown are the distribution of fatty acid chain lengths.
\textbf{c} Distribution of Triacylglycerol (TAG) abundance across organoid adenoma models. ANOVA was performed to model average lipid abundance as a function of organoid line across replicates.
\textbf{d} Distribution of Cholesterolester (CE) abundance across organoid adenoma models. ANOVA was performed to model average lipid abundance as a function of organoid line across replicates.
}
\label{fig_168}
\end{figure}
\bigbreak


\newpage
\section{Image-based profiling of organoid models}

\begin{figure}[h!]
\centering
\includegraphics[width=\textwidth,
                height=\textheight,
                keepaspectratio]{figures/adenomaprofiling/pdf/fig_1_2.pdf}
\caption[Image-based profiling of organoid adenoma models]{\textbf{Image-based profiling of organoid adenoma models. a} Overview of experiments. Organoids were isolated from a transgenic mouse model and genetically edited. Organoids were dissociated and evenly seeded in 384-well plates before perturbation with an experimental small molecule library. After treatment, high-throughput fluorescence microscopy was used to capture the morphology of organoids in 16 selected z-layers and 3 channels. 3D imaging data were projected on a 2D plane using a maximum contrast projection. Here, only pixel areas with the largest contrast among the z-axis were retained. Morphological features were computed based on the projection. Untreated organoid morphology, organoid size and treatment activity scores were integrated with transcript expression, protein abundance, lipid abundance and genogtype data in a Multi-Omics Factor Analysis (MOFA) model. Figure created with support from Johannes Betge (graphical presentation). 
\textbf{b} Uniform Manifold Approximation and Projection (UMAP) of organoid-level features for a random 5\% sample out of imaged organoids. The identical sample is used for visualizations throughout the figure. Organoid genotype is colorcoded and representative images are displayed (magenta = DNA, cyan = actin, cell permeability = yellow, scale-bar: 200µm). \textbf{c} Graph-based clustering of organoids by morphology with 8 resulting clusters. \textbf{d} Organoid size distribution. Color corresponds to the log-scaled organoid area (dark blue: minimum size, yellow: maximum size).}
\label{fig_120}
\end{figure}
\bigbreak

\subsection{Single-Organoid level phenotypes are organised by model genotypes}
Once models were characterised on a molecular level, the previously developed image-based profiling approach was applied. Organoid models of the four different genotypes were perturbed with a library of ca. 1700 compounds and morphological profiles were systematically observed (Figure \ref{fig_120} a). A UMAP projection of the first 25 principal components representing single-organoid morphology showed distinct genotype-dependent morphological states for identified organoids (Figure \ref{fig_120} b). Graph based clustering of organoid morphology profiles resulted in 8 clusters (Figure \ref{fig_120} c). Organoids within cluster 4 and 3 were enriched for Apc+/+ organoid models, cluster 2 and 1 were populated by Apc-/- models. Analogous to gene expression, lipidomics and proteomics representation space, the two Apc mutant organoid models were less distinct from each other than organoids with a WT and isolated KrasG12D/+ genotype (Figure \ref{fig_140} b). While developed organoids that present with a larger organoid area showed distinct genotype-specific morphologies, small organoids and non-viable organoid fragments clustered together across genotypes within cluster 5 (Figure \ref{fig_120} c and d). The distribution of DMSO-treated organoids and small molecule perturbed organoids in morphological space overlapped considerably (Figure \ref{fig_140} a), most likely because many treatments were inactive and did not alter organoid morphology. 

\bigbreak
\begin{figure}[h!]
\centering
\includegraphics[width=\textwidth,
                height=\textheight,
                keepaspectratio]{figures/adenomaprofiling/pdf/fig_1_4.pdf}
\caption[Treatment and genotype dependent effects within the organoid morphology distribution]{\textbf{Treatment and genotype dependent effects within the organoid morphology distribution. a} UMAP representation of DMSO treated (vehicle) and small molecule treated organoids. \textbf{b}, UMAP embeddings of four organoid genotypes (baseline state = 0.1\% DMSO control-treated organoids), grey background consists of randomly sampled organoids.}
\label{fig_140}
\end{figure}

When comparing the morphologies of different organoid models in detail, characteristic differences were identifiable (Figure \ref{fig_130} a). DMSO-treated Apc+/+ organoids showed a strong, regular apical actin cytoskeleton (high average actin intensity) that organized the multicellular formation into a regular-patterned spherical morphology (low average eccentricity). In contrast, Apc-/- organoids showed a relative lack of a regular actin cytoskeleton (low average actin intensity) and a irregular, non-spherical morphology (high average eccentricity). In summary, organoid models showed genotype-dependent differences in morphology. Analogous to differences in molecular state, a primary source of variation was caused by loss of the tumor suppressor gene Apc. Organoids with truncated Apc presented with a higher proliferation rate, increased overall DNA staining intensity and loss of the regular spherical apical actin cytoskeleton that was observed in Apc +/+ organoid models.

\begin{figure}[h!]
\centering
\includegraphics[width=200,
                height=\textheight,
                keepaspectratio]{figures/adenomaprofiling/pdf/fig_1_3.pdf}
\caption[Genotype dependent effects on organoid morphology]{\textbf{Genotype dependent effects on organoid morphology. a}  Morphological organoid profiles from vehicle-treated adenoma models were aggregated. Shown are representative individual organoids with selected features. Points show the mean phenotype for each independent biological replicate. Representative, interpretable features and their z-scores relative to all single organoid profiles are shown (magenta = DNA, cyan = actin, cell permeability = yellow, scale-bar: 25µm)}
\label{fig_130}
\end{figure}
\bigbreak



\subsection{Scoring small molecule induced phenotypes across organoid models}

\begin{figure}[h!]
\centering
\includegraphics[width=\textwidth,
                height=\textheight,
                keepaspectratio]{figures/adenomaprofiling/pdf/fig_1_5_2.pdf}
\caption[Treatment activity scoring]{\textbf{Treatment activity scoring. a} A logistic regression classifier is trained to distinguish morphology profiles of individual treated and untreated organoids across all available replicates. Afterwards, the classifier is applied to a validation set of organoids and the classification performance is estimated using the area under the receiver operating characteristic curve (AUROC) metric. Method implemented by Jan Sauer.
\textbf{b} Distribution of treatment activity scores for all organoid lines, replicates and perturbations. 
\textbf{c} Identifying related treatment induced phenotypes. Normal vectors of treatment specific classifiers were compared by calculating the angular distance (related to cosine similarity, ranging from 0-180 degrees). Small angular distance between vectors correspond to a high similarity between the treatment-induced organoid phenotypes. Method implemented by Jan Sauer.
\textbf{d} A map of compound induced phenotypes for Apc mutant and Apc wildtype organoids. Highlighted are clusters of compound induced phenotypes with related targets. Normal vectors for Apc mutant and Apc wildtype organoids were concatenated before angular distance calculation. Method implemented by Jan Sauer.
}
\label{fig_150}
\end{figure}
\bigbreak

After identifying genotype-dependent morphological differences, the next step was to explore effects of small molecule treatment on different organoid models. To describe the activity of a treatment, the classification-based approach developed during the study of human cancer organoid phenotypes in the previous chapter was used. Briefly, for every treatment and genotype, a logistic regression classifier was trained to distinguish DMSO-treated organoids from treated organoids. The classification performance, expressed as the AUROC, was used to determine the activity of a treatment. A high AUROC score (approaching 1) is observed for compounds that lead to a treatment-induced organoid morphology that is very distinct from DMSO treated organoids. In contrast a low AUROC (centered around 0.5) is observed for compounds where the model's classification performance approaches random guessing (Figure \ref{fig_150} a). The distribution of activity scores across organoid lines showed that most compounds did not lead to an identifyable morphological change (Figure \ref{fig_150} b) and showed AUROC values centered around 0.5. Next to identifying differences around the number of active treatments, I was interested what small molecules were active in a given organoid genotype and their treatment-induced morphology change. Based on the observation that the primary source of variation for treatment activity was the state of the Apc allele, I aggregated organoid lines by their Apc allele for further analysis. In line with the approach taken in the previous chapter, normal vectors of the logistic regression classifiers were compared using the cosine distance (Figure \ref{fig_150} d). The resulting clustering of treatments showed an enrichment for small molecules with related mechanism of action (Figure \ref{fig_150} e and f). For example, EGFR inhibitors were significantly enriched in Apc-mutant organoid lines, while GSK3B-inhibitors, which lead to a stimulation of canonical Wnt signaling, were only enriched in Apc-wildtype organoid models. To summarise the findings above, organoid models showed genotype-specific treatment-induced phenotypes. For example, GSK3B-inhibitors were active in Apc +/+ organoids and showed a characteristic treatment-induced phenotype in these models.

\section{Multi-omics factor analysis identifies shared factors linking functional and structural biological views}

\subsection{Learning a multi-view representation with MOFA}

\begin{figure}[h!]
\centering
\includegraphics[width=350,
                height=\textheight,
                keepaspectratio]{figures/adenomaprofiling/pdf/fig_1_7.pdf}
\caption[Multi-omics factor analysis (MOFA) to identify shared factors linking morphology, size, gene expression, lipidomics, proteomics, genotype and treatment activity]{\textbf{Multi-omics factor analysis (MOFA) to identify shared factors linking morphology, size, gene expression, lipidomics, proteomics, genotype and treatment activity. a} Percent variance explained by the MOFA model for each factor. Untreated organoid morphology, organoid size and treatment activity scores were integrated with genotype, proteomics, lipidomics and mRNA expression data. \textbf{b} Cumulative proportion of total variance explained by each experimental data modality within the MOFA model. \textbf{c}, Visualization of samples in factor space showing factors 1 and 2 as well as factor 1 and 3. Shown are independent replicates for each organoid line. 
}
\label{fig_170}
\end{figure}
\bigbreak

To jointly model the biological and morphological state of organoid models, I performed multi-omics factor analysis (MOFA). Analogous to the process described in the previous chapter, molecular and morphological features from untreated organoids were factorized using k=4 factors (Figure \ref{fig_170} a and \ref{fig_180} a). The learned model was based on both morphological (e.g. morphology, size, small molecule activity) and molecular (e.g. genotype, proteomics, lipidomics and transcript expression) information. To reduce the dimensionality of data modalities with a high number of features, only high variance features from gene expression and proteomics analysis were used. The resulting factorization explained the data ranging from ca. 90\% (gene expression) to 50\% (morphology) of explained variance ($R^{2}$) across the analyzed views. The first three factors captured the majority of variance, >40\%, ca. 10\%, and <10\%, respectively (Figure \ref{fig_170} a). The learned model explained most variance within the mRNA expression and genotype data, while measurements within the organoid morphology data had the lowest explained variance (Figure \ref{fig_170} b). Visual inspection of factors as well as exploration of factor loadings within the genotype view showed that factor 1 explained differences caused by Apc loss of function, while factor 2 explained differences caused by the activation of KrasG12D in an Apc+/+ genotype (Figure \ref{fig_170} c and \ref{fig_180} b). In contrast to factor 2, factor 3 captured differences between Kras+/+ and KrasG12D/+ organoids with Apc loss of function. While the number of factors is a user-defined hyperparameter within MOFA, the method automatically drops excess factors if they are not considered effective based on an applied automatic relevance determination (ARD) prior \citep{Argelaguet2018-yi}. Increasing the number of factors above k=4 in this analysis, did not lead to an increased number of interpretable factors. In fact, factor 4 already did not capture differences between organoid genoypes and was not interpretable by the author from a biological point of view (Figure \ref{fig_180} b). The determined number of explanatory factors corresponded to the hypothesized intrinsic dimensionality of the data: effects attributed to the Apc-/- allele, the Kras G12D allele, and their interaction.

\begin{figure}[h!]
\centering
\includegraphics[width=300,
                height=\textheight,
                keepaspectratio]{figures/adenomaprofiling/pdf/fig_1_8.pdf}
\caption[Multi-omics factor analysis input data and loadings]{\textbf{Multi-omics factor analysis input data and loadings. a} measurement modalities, dimensionality and number of measurements. A third replicate of measurements were available for proteomics and lipidomics only. \textbf{b} Factor loadings for genotype information.} 
\label{fig_180}
\end{figure}
\bigbreak

\subsection{A canonical Wnt signaling associated program caused by Apc loss}

To understand the molecular changes associated with factor 1, factor loadings for mRNA expression data were analyzed using Reactome gene-set enrichment analysis (Figure \ref{fig_190} a). Three clusters of biological processes were significantly associated with a negative factor loading, caused by Apc loss-of-function: 1) Mitotic Anaphase related processes, including spindle checkpoints; 2) Mitotic S-phase, including DNA replication and 3) DNA repair mechanisms, including homology directed repair. In line with the enrichment of processes associated with cell proliferation, factor 1 loadings showed an enrichment of the previously described intestinal proliferation signature (Figure \ref{fig_190} c) and an LGR5+ instestinal stem cell identity signature (Figure \ref{fig_190} b). These findings are in line with the long-standing evidence that loss of Apc leads to a hyperactivation of canonical Wnt signaling, which in turn leads to increased intestinal cell proliferation and Myc-dependent changes towards a stem-like cell state \citep{Sansom2007-wm, Satoh2017-nd}. When focusing on compound activity, a low factor 1 score was significantly linked to increased activity of microtubuli and focal adhesion kinase (FAK) targeting small molecules (Figure \ref{fig_190} d). This morphological sensitivity presented itself primarily as reduced organoid size and number relative to the DMSO vehicle control (Figure \ref{fig_190} e). In contrast, the average treatment activity scores of small molecules targeting Wnt signaling were associated with high factor 1 scores (Figure \ref{fig_190} d).  

%\begin{figure}[h]
%\centering
%\includegraphics[width=\textwidth,
%                height=\textheight,
%                keepaspectratio]{figures/adenomaprofiling/pdf/fig_1_9.pdf}
%\caption{\textbf{Factor loadings for treatment activity. a} Factor 1 and 2 loadings, and \textbf{b} Factor 1 and 3 loadings. Average treatment activity score (AUROC) is color coded.}
%\label{fig_180}
%\end{figure}
%\bigbreak

\begin{figure}[h!]
\centering
\includegraphics[width=\textwidth,
                height=\textheight,
                keepaspectratio]{figures/adenomaprofiling/pdf/fig_2_1.pdf}
\caption[Factor 1, canonical Wnt signaling]{\textbf{Factor 1, canonical Wnt signaling. a} Gene-set enrichment network of factor 1 gene expression loadings. An edge connects Reactome pathways with more than 20\% overlap. Central enriched processes include mitosis, DNA replication and DNA damage repair. \textbf{b and c} Gene set enrichment results of the "Lgr5 intestinal stem cell" and "proliferation" signature by Merlos-Suarez et al \citep{Merlos-Suarez2011-gd}. over ranked factor 1 gene expression loadings (ranking from high factor 1 loading to low factor 1 loading, NES = normalized enrichment score). \textbf{d} Distributions of treatment activity loadings grouped by drug target for factor 1. \textbf{e} Example images of compound treated organoids with WT or Apc-/- genotype. Representaধve images are displayed (magenta = DNA, cyan = actin, yellow = cell permeability, scale-bar: 200μm).}
\label{fig_190}
\end{figure}
\bigbreak

Further exploration of the association between the treatment activity score and Apc genotype showed that small molecule inhibitors of the canonical Wnt secretion pathway protein Porcupine (Porcn), IWP-L6 and LGK-974, were more active in Apc+/+ organoids relative to their Apc-/- counterparts \citep{Liu2013-dh} (Figure \ref{fig_199}a). In contrast, this effect was not observable for PRI-724, a small molecule inhibitor targeting the interaction of beta-catenin and CREB-binding-protein in the canonical Wnt signaling pathway \citep{Okazaki2019-gy} (Figure \ref{fig_199}a). The observed differences in treatment activity scores among small molecule inhibitors are most likely related to their targets' relative location to Apc in the canonical Wnt signaling cascade. While Porcn-dependent Wnt secretion is generally upstream of the Apc-scaffolded destruction complex, the interaction of beta-catenin and the transcriptional coactivator CREB-binding-protein is located downstream of it. As a consequence, inhibition of destruction complex function by loss of Apc is expected to render cells less sensitive to perturbations of the Wnt secretion cascade than direct perturbations of transcription factor binding properties (Figure \ref{fig_199}b).

\begin{figure}[h!]
\centering
\includegraphics[scale=0.75,keepaspectratio]{figures/adenomaprofiling/pdf/fig_2_2_1.pdf}
\caption[Activity of small molecule Wnt signaling inhibitors]{\textbf{Activity of small molecule Wnt signaling inhibitors. a} AUROC activity score for three small molecule inhibitors of canonical Wnt signaling. \textbf{b} Target proteins for small molecules within the canonical Wnt signaling cascade with their relative position to the destruction complex (highlighted in blue).}
\label{fig_199}
\end{figure}
\bigbreak

\newpage

\subsection{A program caused by isolated KrasG12D activation shares signs of oncogene-induced senescence}
While the Apc-/- genotype contributed primarily to factor 1 (Figure \ref{fig_180} b), the KrasG12D allele showed a strong loading for both factor 2 and factor 3. This observation corresponded with the fact that only organoid models without Apc loss of function were separated by factor 2 (Figure \ref{fig_170} c). Factor 2 described a KrasG12D dependent change in cell state in the presence of intact Apc function.


\begin{figure}[h!]
\centering
\includegraphics[scale=0.75,
                keepaspectratio]{figures/adenomaprofiling/pdf/fig_3_1_2.pdf}
\caption[Factor 2, isolated KrasG12D activity and oncogene induced senescence]{\textbf{Factor 2, isolated KrasG12D activity and oncogene induced senescence. a} Distributions of treatment activity loadings grouped by drug target for factor 2. \textbf{b} Relationship of representative ERK inhibitor activity with factor 2 score. Shown are compounds from highlighted groups in panel (a). \textbf{c} Relationship of representative EGFR inhibitor activity with factor 2 score. Shown are compounds from highlighted groups in panel (a). \textbf{d} Gene set enrichment results of a senescence signature \citep{Fridman2008-ky} over ranked factor 2 gene expression loadings (ranking from high factor 2 loading to low factor 2 loading, NES = normalized enrichment score).}
\label{fig_200}
\end{figure}

As above, to understand the molecular mechanisms represented by factor 2, features with large absolute loadings were identified. ERK and MEK inhibitors were more active in factor 2 low models (KrasG12D+/-) while EGFR/HER2 inhibitors were more active in factor 2 high organoids (WT, figure \ref{fig_200} a and b). This juxtaposition in treatment activity against RAS-MAPK pathway members was reminiscent of the previous observations made for canonical Wnt signaling inhibitors (Figure \ref{fig_199}). With oncogenic Kras localized between the receptor-layer (including Egfr and Her2) and downstream mediating kinases (for example Erk), hyperactive Kras signaling likely leads to a cell state with relative resistance to EGFR inhibitors and increased dependency on Erk signaling. The previously observed transcriptional process of Egfr-downregulation as a response to KrasG12D+/- is in line with these observations (Figure \ref{fig_162}b). 

\bigbreak

Oncogene induced senescence is a cell state marked by an arrest of the cell cycle and expression of pro-inflammatory mediators as a response to an oncogenic perturbation. An activated KrasG12D+/- genotype leads to oncogene induced senescence of colon epithelial cells in vivo \citepp{Bennecke2010-zf}. Prompted by previous reports on the effect of an isolated oncogenic Kras allele, I identified an enrichment of a senescence related gene expression signature by Fridman et al. within the loadings of factor 2 \citep{Fridman2008-ky} (Figure \ref{fig_200} c). Transcripts linked to cell senescence, including Igfbp3 (factor 2 loading ca. -0.999) and Hmga2 (factor 2 loading ca. -1.050) ranked among the strongest contributors to the factor. 
\bigbreak


\newpage
\subsection{A program describing the effect of oncogenic Kras in the context of Apc loss of function is marked by increased mTOR signaling }

While factor 2 captured the effect of oncogenic Kras in an Apc wildtype state, factor 3 scores separated organoid models by Kras genotype in the context of an Apc loss of function (Figure \ref{fig_170} c). On average, factor 3 only accounted for less than 10\% of variance across modalities within the MOFA model, supporting the overall similarity of Apc-/- and Apc-/- KrasG12D+/- organoids previously observed separately on the transcriptome, proteome, lipidome and morphology level (Figure \ref{fig_161}a-d and \ref{fig_140} b).

\begin{figure}[h]
\centering
\includegraphics[scale=0.75,
                keepaspectratio]{figures/adenomaprofiling/pdf/fig_4_1.pdf}
\caption[Factor 3, KrasG12D effects in the context of Apc loss of function]{\textbf{Factor 3, KrasG12D effects in the context of Apc loss of function. a} Distributions of treatment activity loadings grouped by drug target for factor 3. \textbf{b} Relationship of representative mTOR inhibitor activity with factor 3 score. \textbf{c} Gene set enrichment results of a Reactome mTORC1 activation signature over ranked factor 3 gene expression loadings (ranking from high factor 3 loading to low factor 3 loading, NES = normalized enrichment score). \textbf{d} Visual summary of Myc gene set enrichment results for organoid state transitions. Myc signatures were significantly enriched in Apc-/- models and depleted in models with isolated KrasG12D+/- mutation.}
\label{fig_300}
\end{figure}
\bigbreak

Exploration of factor 3 loadings identified mTOR and FAK inhibitor activity to contribute to a negative factor score. Double-mutant organoid models (Apc-/- KrasG12D+/-), which had a low factor 3 score, showed a greater treatment activity for these small molecule inhibitors than their single-mutant Apc-/- counterparts (Figure \ref{fig_300} a). This difference in treatment activity was observerable for both ATP-competitive (e.g. INK-128, Sapanisertib) and non-ATP-competitive (e.g. Rapamycin) inhibitors (Figure \ref{fig_300} b). In line with the increased sensitvity to mTOR inhibitors, differential transcript expression analysis comparing single (Apc -/-, abbreviated A) and double mutant (Apc-/- KrasG12D+/-, abbreviated AK) organoids identified a significant increase in mTORC1 activation according to a Reactome signature (Figure \ref{fig_300} c, FDR=0.0095, NES=2.5019). In summary, factor 3 was aligned with the effect of oncogenic Kras in Apc mutant colon organoids and was characterized by increased transcriptional activity and sensitivity to mTOR signaling.


\newpage
\section{Projecting treatment-induced organoid morphologies to identify small molecules with factor-specific effects}
\subsection{Interpreting treatment-induced organoid morphologies using a learnt multi-omics factor model}

A central idea 
learn a multi view representation to
* identify dimensionality
* estimate magnitude
* simplify the modeling of multiple modalities 

even when the dimensionality of the data is a function of the genotypes that are compared, the estimation of the magnitude is useful, as well as the simplification 



Explain the basic algorithm one more time

Now to the projection> 
Explain the inversion 
another benefit is the 
learning of the distribution of states within a support set, it can help interpret samples in a query set. 

This enables us
In our case we could train a classifier to compare lines directly, but at the risk of interinterpreting compounds that moved between states

We invert a function to interpret treated organoid morphology

no clustering needed for detection 
direction of the change observable




\begin{figure}[h]
\centering
\includegraphics[scale=0.75,
                keepaspectratio]{figures/adenomaprofiling/pdf/fig_5_0.pdf}
\caption[]{\textbf{a} \textbf{b}}
\label{fig_500}
\end{figure}
\bigbreak

\begin{figure}[h]
\centering
\includegraphics[scale=0.75,
                keepaspectratio]{figures/adenomaprofiling/pdf/fig_5_3_1.pdf}
\caption[Projection of factor 1 scores for treatment-induced phenotypes and viability changes]{\textbf{Projection of factor 1 scores for treatment-induced phenotypes and viability changes.} Highlighted are compounds leading to a significant change in projected factor scores across all organoid lines (ANOVA). Organoid viability is predicted using a random-forest based classified (LDC) with scores from 0 (no toxicity) to 1 (complete toxicity)}
\label{fig_180}
\end{figure}
\bigbreak


a desired effect moves along one factor, but keeps other factors constant - specificity of the effect - factors are orthogonal 
I also treatments that affect both factors
enriched for highly active and toxic compounds
also enriched for natural compound fulfilling PAINS criteria



\subsection{Active treatments without factor-specificity are enriched for toxic and assay-interference compounds}

To conclude this chapter, I searched through small molecules within the query set that were predicted to move organoids towards a wildtype state along factor 1 and 2. Such small molecules are of particular medical relevance, as this would suggest a property for a given small molecule to counteract the effects of Apc loss or Kras activation, respectively. Analogous to the concept of revertant genetic mutations, I referred to small molecules with a predicted ability to move transformed organoids towards a wildtype state as potential revertant therapeutics. Unfortunately, the majority of compounds that led to a predicted shift towards a wildtype organoid state along both axes were also associated with a high predicted toxicity (Figure \ref{fig_310} a, color coding are predictions based on LDC viability classifier). Ellagic acid, however, a natural compound that was predicted to function as a revertant, showed a moderate predicted toxicity (Figure \ref{fig_310} a, top right). Inspection of organoid models treated with Ellagic acid showed reduced number and size of organoids compared to the vehicle control, while organoid viability was generally maintained (Figure \ref{fig_310} b). 

\begin{figure}[h]
\centering
\includegraphics[scale=0.75,
                keepaspectratio]{figures/adenomaprofiling/pdf/fig_5_1_2.pdf}
\caption[Projected changes in factor 1 and 2 scores after small molecule treatment]{\textbf{Projected changes in factor 1 and 2 scores after small molecule treatment a} Shows are treatments leading to change in projected factor scores across all organoid lines (ANOVA). Color coding represents predictions based on LDC viability classifier. The structural formula of Ellagic Acid is depicted in the top right. \textbf{b} Organoid treatment with Ellagic acid. Depicted are representative example images for each line (blue = DNA, red = actin, green = cell permeability, scale bar=200µm).}
\label{fig_310}
\end{figure}
\bigbreak


\subsection{GSK3 beta inhibitors move Apc wildtype organoids specifically along a canonical Wnt signaling program}

After linking factor 1 to Apc loss and identifying associated molecular changes, I was interested in identifying active treatments that shifted organoid morphologies along the factor 1 axis. As described in the previous chapter, small molecules that led to a predicted factor 1 change were identified using ANOVA. To identify treatments that led to a drop in organoid viability, a classifier developed by Jan Sauer and described in the previous chapter (LDC) trained on ground-truth lethal treatments within the stem-cell library was applied. Two groups of compounds were identified that induced a shift towards lower factor 1 scores (group A, figure \ref{fig_180}) and higher factor 1 scores (group B, \ref{fig_180}). Small molecules within group A induced a morphology associated with Apc loss while maintaining organoid viability. In contrast, members of group B primarily led to a loss of viability and a shift towards a morphological state associated with Apc wildtype organoids. Of note, small molecules within these groups had related target proteins. Compounds within group A (incl. CHIR-98014, CHIR-99021, LY2090314) targeted GSK3 beta - a kinase with central function within the canonical Wnt signaling destruction complex. Inhibition of GSK3 beta leads to hyperactivation of canonical Wnt signaling. Members of group B primarily targeted the Proteasome (incl. Bortezomib, Carfilzomib, MLN2238). Further validation of group A showed how treatment with the GSK3 beta inhibitor CHIR-98014 led to treatment-induced phenotypes in WT and KrasG12D+/- organoids that phenocopied the unperturbed morphology of Apc-/- and double mutant organoid models (Figure \ref{fig_185} a). On the feature level, treatment led to an increase in organoid size and DNA intensity (Figure \ref{fig_185} b) in Apc wildtype models. This change in morphology was likely due to an increased proliferation rate of mutant cells, leading to rising organoid size and a higher density of nuclei per analyzed object. Guided by the identification of a strong GSK3 beta inhibition phenotype in Apc wildtype organoids, which was already identified during the unbiased clustering of treatment-induced phenotypes previously (Figure \ref{fig_150} d), I analyzed small molecules that clustered with known inhibitors of this kinase based on the angular distance of their drug effect vectors within these models (Figure \ref{fig_185} c). In line with the observations in this experiment, all small molecules clustering with well-described inhibitors of GSK3 beta had previously described off-target binding activity against this kinase within the LINCS KINOMEScan database \citep{Duan2014-ku}. 





\begin{figure}[h]
\centering
\includegraphics[scale=0.75,
                keepaspectratio]{figures/adenomaprofiling/pdf/fig_2_4_1.pdf}
\caption[GSK3 beta inhibition dependent morphology in colon organoid models]{\textbf{GSK3 beta inhibition dependent morphology in colon organoid models. a} Small molecule inhibition of GSK3 beta (CHIR98014) leads to phenocopying of Apc-/- genotype organoid models. \textbf{b} Shift of morphological features of wildtype and KrasG12D +/- organoid models treated with CHIR98014. Shown is an increase in organoid size (Area) and DNA intensity. \textbf{c} Excerpt of clustering from figure \ref{fig_150} d, labeled with known binding activity of listed small molecules. Rucaparib is not member of the cluster and shown for comparison.}
\label{fig_185}
\end{figure}
\bigbreak




\end{flushleft}
\begin{savequote}[75mm]
I learned that if you work hard and creatively, you can have just about anything you want, but not everything you want. Maturity is the ability to reject good alternatives in order to pursue even better ones.
\qauthor{Ray Dalio}

\end{savequote}

\begin{flushleft}

\chapter{Discussion}

My contributions to the general field of biological profiling methods described in this thesis are (1) application of the image-based profiling method to organoid models, and (2) learning biologically meaningful low dimensional representations of organoid state by pairing imaging data with additional experimental modalities through multi-view representation learning. 
\bigbreak

This discussion is divided into three sections, a general methodological part as well as discussions on findings within the patient-derived organoid project, and the engineered mouse organoid profiling project.


\section{Image-based profiling of organoids}

\subsection{Organoids as high validity in-vitro models}

Organoids are representative in vitro models for diverse human tissues and can be used for image-based profiling. While the prospective use of cancer organoids as a diagnostic is currently limited by high sample dropout and tunraround time, a high overall predictive validity for multiple therapeutic regimens has been reported. The use of organoid models in early stage drug discovery, however, is not limited by the same constraints existing in a diagnostic context. For therapeutic discovery, previous studies have successfully used organoids to perform medium-scale small molecule treatment assays. The most commonly used method is screening with an ATP based cell viability readouts \cite{Van_De_Wetering2015-ko}. Additionally, imaging studies with organoids have been used to characterize developmental processes such as the self-organization of intestinal cells \cite{lukoninPhenotypicLandscapeIntestinal2020c, boehnkeAssayEstablishmentValidation2016a} or the morphological response to individual drugs \cite{badder3DImagingColorectal2020b, serraSelforganizationSymmetryBreaking2019}. While image-based profiling of in vitro models has become an important tool for the analysis of biological processes, particularly in drug discovery and functional genomics \cite{carpenterImagebasedChemicalScreening2007}, performing such high-content experiments in organoids has been a technological challenge. The primary challenges comprise the 3D growth pattern of organoid models as well as the high morphological heterogeneity. Sparse 3D imaging and projection as well as adjustment to the fragment seeding protocol presented in this thesis enabled small molecule image-based profiling of patient-derived and genetically engineered organoid disease models. 
\bigbreak

High validity in-vitro models, such as organoids, are an essential component of the therapeutic discovery process. Therapeutic discovery is a sequential decision making process in which, for example, a selected small moelcule can not, for ethical and practical reasons, be tested directly in a clinical context to observe its treatment effect. Instead, therapeutic candidates have to be evaluated using one or more models to approximate the treatment effect that could be observed in large clinical trials. In the context of optimisation theory, these models can be considered "surrogate models", while the large clinical trial being an "oracle" that is approximated. In practice, multiple in-silico, in-vitro and in-vivo surrogate models are combined and used sequentially to guide decision making during a discovery program. The higher the predictive validity and the lower the (ethical and financial) cost of using a surrogate model, the more value does it provide to the therapeutic discovery process. While determining the validity of an in-vitro model is an empirical process, a set of axioms to prioritise models have been formulated by Vincent et al. \cite{vincentDevelopingPredictiveAssays2015a} are presented below in a modified format: 

\begin{enumerate}
    \item The model must have a clear link to the disease of interest (i.e. matched tissue of origin, representative culture conditions) and, if engineered, model the disease state based on the best understanding of the disease pathophysiology
    \item The phenotype observed during the experiment should represent the desired clinical endpoint (i.e. overall tumor regression) and ideally capture a high degree of information
    \item The treatment schedule and overall experimental design should be closely aligned to assays that have been successfully used in a clinical diagnostic context
\end{enumerate}

Designing or "fitting" in-vitro biological models to a disease along these axioms is a cost intensive process, but already minor changes in predictive validity of an experiment can offset these cost in a drug discovery program. In fact, economic research into the overall cost of therapeutic discovery by Scannel et al. \cite{scannellWhenQualityBeats2016} concludes with the statement:

\begin{quote}
"The rate of creation of valid screening and disease models may be the major constraint on R\&D productivity."
\end{quote}

\subsection{Applying multi-view representation learning to image-based profiling: Towards multi-view profiling}

Increasing the interpretability of the representation space within image-based profiling experiments can aid in the discovery of new biological processes. While gene-level measurements from other profiling experiments, such as transcriptome profiling, can be intuitively interpreted and used for human or algorithmic causal discovery, image-based profiling measurements suffer from a lack of mechanistic biological interpretability. For example, while changes in protein abundance after treatment with a small molecule can be directly interpreted by biologists and put into context with prior knowledge and literature to guide subsequent decision making, the same is not true for image-based profiling data. Classic quantitative features of cellular morphology are rarely descriptive, and even when dimensionality reduction methods or self-supervised learning techniques are used, the learnt representations are not interpretable within the context of our current mechanistic understanding of cellular biology. One approach to increase the interpretability of representations in a given image-based profiling experiment, is to collect a multi-view support set comprising transcriptomic, proteomic or metabolomic data to "annotate" the morphological states. This learnt factor representation can then be used to interpret observations within the remaining image set, here referred to as the query set. By choosing such an approach I enabled a higher degree of interpretability of morphological information in image-based profiling experiments, while increasing the risk of mis- or over-interpreting state changes in factor space. The underlying motivation and trade-offs involved in using multi-view representation learning for image-based profiling experiments are discussed below. 
\bigbreak

The most direct way to increase the interpretability and robustness of profiling experiments in general and image-based profiling experiments in particular would be to collect complete multi-view information for all observed conditions (i.e. imaging, transcript abundance, protein abundance, and phosphorylation state). Such experiments would also directly address the "target deconvolution problem" of phenotype-based therapeutic discovery. A treatment could be interpreted both in terms of its effect on course-grained morphological and fine-grained molecular states. Both from a physical (some measurement methods destroy the sample) and financial perspective, however, performing such profiling experiments poses a challenge. A possible solution to approximate the interpretability of such a thorough "multi-view profiling" experiment, is to divide the conditions and the corresponding observations into a thoroughly annotated small, multi-view (i.e. images, transcript abundance, protein abundance) support set and a large, uncharacterised single-view (i.e. only images) query set. The "support set" and "query set" nomenclature is taken from few shot learning - a part of the machine learning literature focused on scenarios when only few well-annotated observations are available, a common situation in biomedical research and therapeutic discovery. Once the support set is available, a simple multi-view representation is learnt from the data. It is important to include conditions in the support set that are well annotated *and* representative of the states within the phenotype space that are supposed to be explored, as the representation learnt from this data will be used to interpret the remaining observations within the query set. For example, in the mouse adenoma project, observations of all four untreated organoid models were part of the support set which then enabled describing treatments within the query set, such as GSK3b inhibitors, in terms of their ability to shift Apc wildtype organoids towards an Apc-null-like state. With a support set defined and a representation learnt, the remaining observations within the query set can then be projected into the representation space. This way hard-to-interpret but easy-to-measure morphology changes can be interpreted based on the previously learnt relationships between morphology and other, more interpretable, modalities.
\bigbreak

There are, however, risks associated with applying multi-view representation learning in profiling experiments generally, and image based profiling experiments in particular. A central limitation of the described multi-view representation learning approach are out of distribution samples - the risk of misinterpreting the state of a sample within the query set that is far outside of the distribution of states learnt from the support set. For example, if observation from Apc mutant organoid lines had not been included in the support set that the multi-view representation was learnt from, the effect of GSK3b inhibitors on Apc wildtype organoids would have been not interpretable or, even more risky, misinterpreted. Put differently, a support set needs to be constructed with care, because all other observations will be interpreted using a model that was learnt from it.
\bigbreak

A second risk associated with multi-view representation learning is overinterpretation of treatment effects that go beyond the learnt factor-level. To use the example of GSK3b inhibitors again: While causality between the treatment and the shift in factor space can be established, it is not clear what underlying mechanisms led to the shift. Without further knowledge about the small molecules, the plausible hypotheses explaining the GSK3b inhibitor's effect could include (1) inhibition of Apc, (2) activation of beta-catenin, and (3) activation of Myc based transcription, to name a few. While describing the state of the biological model using multiple factors instead of a single factor, like cell viability, reduces the degeneracy (= the number of ways a given state can be explained) of the model, there are still a multitude of possible molecular mechanisms that can lead to a shift in factors.
\bigbreak

Another challenge this approach faces is at the same time an opportunity: increasing the resolution of factors and discovering causal relationships between them. For example, factor 1 in the mouse colon organoid project was driven by (1) transcript abundance changes associated with mitosis and (2) dna repair, as well as (3) changes of cholesterol ester abundance. Factor 2, in contrast, was driven by (1) transcript abundance changes associated with cellular senescence, and (2) reduced beta-oxidation, as well as (3) accumulation of triacylglycerol lipid species. At this point, it is challenging to assign any causal hierarchy to these coarse-grained factors or the diverse processes that they contain from the data alone. With more observations being available, breaking down coarse-grained factors and establishing a directional graph between them becomes a possibility. First directed factor graph representations are already being learnt on single-view perturb-seq profiling data and show promising results in their ability to predict the phenotype of unseen treatments. Overall, identification of directed factor graphs is challenging, but new opportunities in machine learning aided causal discovery are emerging lopezLargeScaleDifferentiableCausal2022.
\bigbreak

Given the risks and limitations of interpreting learnt multi-view representations, simple, linear methods should be preferred. In the presented projects, I used multi-omics matrix factorisation (MOFA) to learn a multi-view representation. MOFA can be conceptionalised as a sparse-PCA for multi-view data and is based on the bayesian group factor analysis framework. While other methods -such as iCluster, classical group factor analysis or canonical correlation analysis- can be used to learn a representation across modalities, MOFA has been developed with design choices that make it particularly useful for biological data from profiling experiments:
\bigbreak

\begin{enumerate}
    \item feature sparsity regularisation - the model uses a spike and slab prior to reduce the number of features assigned to a given factor
    \item factor sparsity regularisation - the model includes an automatic relevance determination prior to reduce the number of factors that are active in a given view
    \item modality-specific noise models -  the model uses modality-specific prior distributions for continous, binary and count-based observations noise terms 
\end{enumerate}

Additional simplifications of MOFA include its ability to handle missing data (not supported by iCluster) and high training speed. Generally, multi-view representation learning method should perform simple transformations using modality specific features, have a tolerance to data missingness, and regularised towards sparsity. 
\bigbreak

\subsection{Perspective on multi-view profiling of organoids}
This thesis discusses the establishment of image-based profiling for patient derived and genetically engineered organoid models. Progress in image analysis \cite{chandrasekaranImagebasedProfilingDrug2021a}, the emergence of standardised protocols \cite{Bray2016-ue}, as well as cross-organisational alliances \cite{chandrasekaranJUMPCellPainting2023} are currently leading to a widespread uptake of the image-based profiling method. In the future, I expect the combination of multi-view representation learning and image-based profiling of organoids introduced in this thesis to be further explored and improved - with possibly even more complex and representative organoid models, such as coculture systems. A direct next step towards more robust multi-view profiling could be the combination of RNA-Sequencing of formalin fixed cells after samples have completed their microscopy run. Directly enriching learnt image representations with such transcriptomic data holds the potential to accelerate the discovery of new biological mechanisms and therapeutic candidates. Advances in modality-specific general purpose representation learning models, such as for biological images, will likely further increase the interest in learning and sampling multi-view representations of cellular state \cite{pfaendlerSelfsupervisedVisionTransformers2023}.



\section{Multi-view profiling of colorectal cancer organoids identifies factors of cancer organoid architecture and plasticity}

\subsection{The importance of large in-vitro cancer organoid collections for translational medicine}

Patient derived cancer organoids are representative in-vitro models of their tissue of origin. In therapeutic discovery, it is desireable to test a candidate against as many such high validity in-vitro models as possible, to make confident estimates about its future clinical value. Despite the extremely high value of human in-vitro models, the number of available models is relatively low. Today only about 500 in-vitro models are easily accesible from large commercial-grade providers, about 2000 have been systematically characterised by international scientific consortia, and about 100,000 human cell lines have been tracked in literature-based repositories. More so, not all human cell lines models have the same degree of representativeness relative to their tissue of origin. Mapping of transcript abundance data from human cancer cell lines and bulk tissue samples shows systematic differences in the degree of representation across tissues. Central nervous system, liver and lung cancer derived cell lines have the lowest similarity with their tissue of origin, while colorectal cancer cell lines show an overall high degree of similarity. As a consequence, both the quantity and quality of new in-vitro models needs to be considered during the establishing process. At the time of writing, there are multiple efforts to develop new in-vitro disease models from primary samples. Here, the organoid method has emerged as a popoular approach to develop new models, given its establishment efficiency and availability of protocols for multiple tissues. 
\bigbreak

\subsection{Characterising colorectal cancer organoid collections with image-based profiling}

To increase their value for applied research, new in-vitro models should be thoroughly characterised - both in an untreated and perturbed state. Image-based profiling is a cost effective and informative profiling method (introduced and discussed above). In this project I combined the isolation of new patient derived organoid models with the direct collection of an image-based profiling reference dataset. I applied a multi-view representation learning approach to model the observed data (discussed above) and identified two primary axes of variation across organoid models. 
\bigbreak

\subsection{Two primary biological programs identified in colorectal cancer organoids}

An IGF1R signaling program is associated with increased organoid size, EGFR inhibitor resistance and can be induced by mTOR inhibition
\bigbreak

An LGR5+ program is associated with cystic organoid architecture, Wnt signaling inhibitor sensitivity, and can be induced by inhibition of MEK
\bigbreak

\subsection{Towards an organoid multi-view profiling atlas}


\section{Multi-view profiling of pre-malignant models of colon cancer}





\end{flushleft}
% the back matter
\abstractpage
\singlespacing
\clearpage
\printbibliography
\addcontentsline{toc}{chapter}{\numberline{9}References}
%\bibliographystyle{harvard}
\eigenanteil
\addcontentsline{toc}{chapter}{Eigenanteil}
\curriculumvitae
\addcontentsline{toc}{chapter}{Curriculum Vitae}
\acknowledgments
\addcontentsline{toc}{chapter}{Acknowledgments}
\eidesstaat
\addcontentsline{toc}{chapter}{Eidesstaatliche Versicherung}
\end{document}