Das kolorektale Karzinom gehört zu den häufigsten und tödlichsten Krebserkrankungen. Organoide sind multizelluläre, drei-dimensionale in-vitro Modelle, welche die effiziente Kultur von gesundem und malignen adulten Gewebe ermöglichen, unter anderem auch die Kultur von Kolonepithelium und kolorektalen Karzinomen. Aufgrund ihrer Eigenschaften, werden Organoide aktuell sowohl für diagnostische Zwecke als auch für den Einsatz in der Wirkstoffforschung evaluiert. Eine populäre Methode in der Wirkstoffforschung ist die bild-basierte Hochdurchsatztestung - hier werden in-vitro Modellen mit potentielle Wirkstoffe behandelt und die morphologischen Veränderungen aufgenommen und mittels Bildverarbeitung ausgewertet. Aufgrund des drei-dimensionalen Wachstumsmusters sowie komplexer Wachstumsbedingungen wurden Organoid Modelle des kolorektalen Karzinoms bisher noch nicht in der bild-basierten Hochdurchsatztestung eingesetzt. 
\bigbreak

In dieser Arbeit wurde die bild-basierte Hochdurchsatztestung von kolorektaken Karzinomorganoiden und Adenomorganoiden und insgesamt mehr als 2000 Medikamenten-Bedingungen getestet. Die gesammelten Informationen wurden daraufhin zusammen mit unterstützenden "multi-omics" Daten mittels einer Gruppen-Faktor-Analyse analysiert und Faktoren, welche molekulare Unterschiede als auch Unterschiede in der Medikamentensensitivität erklärten, identifiziert.
\bigbreak

Zuerst wurden Organoid Modelle angelegt welche das Karzinomstadium sowie die frühe Pathogenese im Adenomstadium repräsentieren. Für Modelle des Karzinomstadiums wurden endoskopisch isolierte Biopsien genutzt um patienten-stämmige kolorektale Karzinomorganoide zu isolieren. Für Modelle des Adenomstadiums wurden Kolonorganoide von einer \textit{LSL-Kras\textsuperscript{G12D} CreERT2} transgene Maus gewonnen und mittels CRISPR eine  \textit{Apc\textsuperscript{-/-}} insertion/deletions Mutation eingefügt.
\bigbreak

In patientent-stämmigen kolorektalen Karzinomorganoiden wurden zwei prominente Faktoren identifiziert: Der erste Faktor organisisert Organoide nach ihrer Groesse und der aktivitaet des IGF signalweges. Organoide mit einem groesseren Diameter und hoher Aktivitaet des IGF Signalweges waren sensitiv gegenueber der Inhibition des IGF1 Rezeptors und resistent gegueber Inhibitoren des EGF Rezeptors. 
\smallbreak
Der zweite Faktor organisierte Organoide nach dem Ausprägung der LGR5+ Stammzell Identität. Organoide mit einer ausgeprägten intestinalen Stammzellsignatur zeichnen sich durch eine sphärische und symmetrische multizelluläre Organisation aus. Diese Organoide zeigten eine hohe relative Sensitivität gegenueber Inhibitoren des Wnt Signalweges und eine relative Resistenz gegenueber Inhibitoren der ERK-MAPK Signalweges. 
\smallbreak
Mithilfe der gelernten Faktoren konnten auch bekannte nicht-lethale Medikamenteneffekte systematisch identifiziert werden: (1) Inhibitoren des mTOR Komplexes zeigten in den Experimenten eine behandlungs-induzierte Morphologie, welcher auf einen positiven Effect auf Faktor 1 hindeutete. Tatsächlich führte Behandlung von Organoiden mit einem mTOR Inhibitor in validierenden Experimenten zu einer reaktiven Expression des IGF1 Rezeptor Effektors Irs-1. (2) Ein ähnlicher Effekt zeigte sich bei MEK Inhibitoren in Bezug auf Faktor 2. In validierenden Experimenten führte die Behandlung zu einem erhöhten Expression von \textit{LGR5} Transkripten.
\bigbreak

In genetisch modifizierten Maus Organoiden wurde ein ähnlicher experimenteller Ansatz verfolgt. Hier wurden drei prominente Faktoren identifiziert: Der erste Faktor unterteilte Apc wildtyp von Apc muttierten Organoiden. Verlust der Apc Funktion führte zu erhöhter Transkript Expression von Proliferations- und DNA Reparatur-Markern. Es fuehrte ebenso zu erhoehter Sensibilitaet gegenueber Perturbatoin des Cytoskeletts, zum Beispiel durch Taxane und Fak Inhibitoren, sowie morphologisch zu einem Verlust der sphaerischen und symmetrischen multizellulaeren Organisation. Apc mutierten modelle zeigten eine unabhaengigkeit von üblicherweise essentiellen Wnt liganden im Kulturemedium. Mit Bezug auf den zellulaeren Metabolismus, fuehrte der Verlust von Apc zu einer Reduktion von Transkripten der beta-Oxidation, sowie einer Akkumulation von Tri-acylglyceriden und Cholesterolester Speicherlipiden.
\bigbreak

Der zweite Faktor unterteilte Kras G12D/+ Organoide modelle von wildtyp Organoiden. Organoide mit Kras G12D/+ genotyp zeigten eine Onkogen-induzierte Seneszenz signatur, reduzierte  Expression von Transkripten der beta-Oxidation, eine Akkumulation von Speicherlipiden, sowie eine deutliche Reduktion von Transkripten der oxidativen Phosphorylierung. Es bestand eine hohe relative Sensibilität gegenüber ERK- und MEK- Inhibitoren und eine geringe relative Sensibilität gegenüber Inhibitoren des EGF Rezeptors. 
\bigbreak

Der dritte Faktor trennte Apc -/- Organoide von Apc -/- Kras G12D/+ Organoid. Modelle mit  Kras G12D / -, Apc -/- Genotyp zeigten eine erhöhte Sensitivität gegenüber mTOR Inhibitoren sowie eine ehöhte Expression einer mTORC1-Aktivierungs Signatur.
Wie zuvor, konnten die gelernten Faktoren genutzt werden um nicht-lethale Medikamenteneffekte systematisch zu identifizieren. Hierbei zeigten Inhibitoren der GSK3-beta Kinase, bekannte Aktivatoren des Wnt Signalweges, einen deutlichen Effekt auf Faktor 1.
\bigbreak

Die Ergebnisse dieser Arbeit zeigen das Potenzial von Kolon Organoiden und derer integrierten bild-basierten Medikamententestung in der translationalen medizinischen Forschung. Hoffentlich tragen die im Rahmen dieser Arbeit generierten Ergebnisse zur Erforschung neuer, präziser Wirkstoffe für das Kolorektale Karzinom bei.