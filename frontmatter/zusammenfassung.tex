Das kolorektale Karzinom 
Organoide erlauben die kultur von gesundem und malignment transformierten epithelialium des Kolons
Gruppen Patienten Kohorte
Gruppe Organoid modelle etabliert mittels CRISPR - vom gesudentn Maus Organoide entwickelt
Gen Expressions sowie Lipidom und Proteom-weite Massen Spectrometrie ß intesnive charackterisirung 
Kombinierte hochdurchsatz Mikroskopie und Medikamenten Testung weiteren aufschluss über die Morphologie von Organoiden und deren Aenderung unter Medikemantenbehandlung
Gruppen Faktor Analyse, Multi-omics Factor Analysis, angewendet, um diese Daten zu integrieren und über die verschiedenen Modalitäten hinweg existierende biologische Faktoren zu identifizieren. 

Organoid Groes1e und einfache Intensitaet sind gute praediktoren von Zell viabilitaet 

In Patienten Organoiden wurden zwei prominente Faktoren identifiziert: (1) der erste Faktor organisisert Organoide nach ihrer Groesse und der aktivitaet des IGF signalweges. Organoide mit einem groesseren Diameter und hoher Aktivitaet des IGF Signalweges waren, im Vergleich zu anderen Organoiden innerhalb der Kohorte, sensitiv gegenueber der Inhibition des IGF1 Rezeptors und resistent gegueber Inhibitoren des EGF Rezeptors. 
(2) der zweite Faktor organisierte Organoide nach dem Ausmass der LGR5+ Stammzell identitaet. Organoide mit einer hohen Gen Expression von LGR5 und der damit assoziierten intestinalen Stammzellsignatur zeichnen sich durch eine ausgeprägte sphaerische und symmetrische multizellulaere Organisation aus. Diese Organoide zeigen eine hohe relative Sensitivitaet gegenueber Inhibitoren des Wnt Signalweges und eine relative Resistenz gegenueber Inhibitoren der RAS-MAPK Signalweges, wie MEK- und ERK- inhibitoren. 
Mithilfe der gelernten Faktoren konnten auch bekannte nicht-lethale Medikamenteneffekte systematisch identifiziert werden: (1) ATP kompetitive und nicht-kompetitive inhibitoren des mTOR Komplexes zeigten in den hoch-durchsatz Experimenten eine Behandlungs-induzierte Morphologie, welcher auf einen positiven Effect auf Faktor 1 hindeutete. Tatsächlich führte Behandlung von Organoiden mit einem mTOR Inhibitor in validierenden Experimenten zu einer reaktiven Expression des IGF1 Rezeptor Effektors IRS-1. (2) Ein ähnlicher Effekt zeigte sich bei Inhibitoren der Mek Kinase in Bezug auf Faktor 2. Behandlung von Organoiden mit Mek Kinase Inhbitoren in validierenden Experimenten führte zu einem erhöhten Expression von LGR5 Transkripten.

In genetisch modifizierten Maus Organoiden wurde ein âhnlicher experimenteller Ansaty verfolgt. Hier wurden drei prominente Faktoren identifizier: der erste Faktor unterteilte Apc wildtyp von Apc muttierten Organoid modellen. Verlust der Apc Funktion fu2hrte zu erheohter Transkript Expression von Proliferations- und DNA Reparatur-Markern. Es fuehrte ebenso zu erhoehter Sensibilitaet gegenueber Perturbatoin des Cytoskeletts, zum Beispiel durch Taxan-Spindelgiften und Fak Inhibitoren, sowie morphologisch zu einem Verlust der sphaerischen und symmetrischen multizellulaeren Organisation. Apc mutierten modelle zeigten eine unabhaengigkeit von üblicherweise essentiellen Wnt liganden im Kulturemedium. Mit Bezug auf den zellulaeren Metabolismus, fuehrte der Verlust von Apc zu einer Reduktion von Transkripten der beta-Oxidation, sowie einer Akkumulation von Tri-acylglyceriden und Cholesterolester Speicherlipiden.
Der zweite Faktor unterteilte Kras G12D / - mutierte organoid modelle von Kras WT organoiden in der Anwesenheit von intaktem Apc. Organoide mit einer isolierten Kras G12D mutation zeigten eine Onkogen-induzierte Seneszenz signatur, reduzierte  Expression von Transkripten der beta-Oxidation, eine Akkumulation von Speicherlipiden, sowie eine deutliche Reduktion von Transkripten der oxidativen phosphorylierung. Es bestand eine hohe relative Sensibilität gegenüber Inhibitoren der Erk und Mek Kinase und eine geringe relative Sensibilität gegenüber Inhibitoren des EGF Rezeptors. 
Der dritte Faktor unterteile Kras G12D / - mutierte organoid modelle von Kras WT organoiden bei Verlust von Apc. Organoid Modelle mit  Kras G12D / -, Apc -/- Genotyp unterschieden sich von  Apc -/- organoiden durch eine erhöhte relative Sensitivität gegenüber mTOR Inhibitoren sowie einer ehöhten mTORC1-Aktivierungs Transkript Signatur.
Wie zuvor, konnten die gelernten Faktoren genutzt werden um nicht-lethale Medikamenteneffekte systematisch zu identifizieren. Hierbei zeigten zum Beispiel Inhibitoren der Gsk3-beta Kinase einen deutlichen erwarteten Effekt entlang Faktor 1 - ein Effekt in dem Gsk3-beta Inhibition eine Phänokopie des Apc Verlustes im Apc +/+ Organoid Modell erzeugt.

Die Ergebnisse dieser Arbeit zeigen das Potenzial von Kolon Organoiden und derer integrierten bild-basierten Medikamententestung in der translationalen medizinischen Forschung. Zum einem, erlaubt diese Methode die Identifikation von molekularen klinischen Subtypen und deren characteristischen Medikamentensensitivitaeten auf der Basis von Patienten-Organoid modellen. Zum anderen, erlaubt diese Methode die genetische, schrittweise Dekonstruktion der Kolon Karzinom Entstehung und zeigt hier ebenfalls charackteristische Medikamentensenitivitaeten auf. Die im Rahmen dieser Arbeit etablierte Methode sowie die hiermit durchgeführten Experimente tragen somit hoffentlich sowohl zur personalisierten Behandlung von Patienten als auch zur Erforschung neuer, präziser Wirkstoffe für das Kolorektale Karzinom bei.