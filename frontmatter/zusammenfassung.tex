Das kolorektale Karzinom gehört zu den häufigsten und tödlichsten Krebserkrankungen. Organoide sind multizelluläre, drei-dimensionale \textit{in-vitro} Modelle, welche die effiziente Kultur von gesundem und malignen adulten Gewebe ermöglichen, unter anderem auch die Kultur von Kolonepithelium und kolorektalen Karzinomen. \textit{In-vitro} zeigen Organoide eine ausgeprägte morphologische Heterogeneität. Die molekularen Faktoren, welche diese Unterschiede in der Organoid Architektur bedingen, sind jedoch nicht gut verstanden.
\bigbreak

Das Ziel dieser Promotion war es (1) molekulare Faktoren, welche die Organoid Architektur determinieren, zu identifizieren, (2) Faktor-abhängige Unterschiede in der Medikamentensensibilität zu beschreiben, und (3) Medikamente zu charakterisieren, welche die Organoid Architektur entlang der identifizierten Faktoren verändern können. 
\bigbreak

Zuerst wurden Organoid Modelle angelegt, welche das kolorektale Karzinomstadium sowie die frühe Pathogenese im Adenomstadium repräsentieren. Für Modelle des Karzinomstadiums wurden endoskopisch isolierte Biopsien genutzt um patienten-stämmige kolorektale Karzinomorganoide zu isolieren. Für Modelle des Adenomstadiums wurden Kolonorganoide von einer \textit{LSL-Kras\textsuperscript{G12D} CreERT2} transgene Maus gewonnen und mittels CRISPR eine  \textit{Apc} insertion/deletions Mutation eingefügt.
\bigbreak

Daraufhin wurden Organoide mittels bild-basierter Hochdurchsatz-Medikamententestung untersucht und insgesamt 2000 Behandlungsbedingungen getestet. Die gesammelten morphologischen Informationen wurden daraufhin zusammen mit unterstützenden "multi-omics" Daten mittels einer Gruppen-Faktor-Analyse analysiert und Faktoren, welche molekulare und morphologische Unterschiede als auch Unterschiede in der Medikamentensensitivität erklärten, identifiziert.
\bigbreak

In patienten-stämmigen kolorektalen Karzinomorganoiden wurden zwei prominente Faktoren identifiziert: Der erste Faktor organisierte Organoide nach ihrer Größe und der Aktivität des IGF-Signalweges. Organoide mit einem größeren Diameter und hoher Aktivität des IGF-Signalweges waren sensitiv gegenüber der Hemmung des IGF1 Rezeptors und resistent gegenüber EGFR-Inhibitoren. 
\smallbreak
Der zweite Faktor organisierte Organoide nach dem Ausprägung der LGR5+ Stammzell Identität. Organoide mit einer ausgeprägten intestinalen Stammzellsignatur zeichneten sich durch eine sphärische und symmetrische multizelluläre Organisation aus. Diese Organoide zeigten eine hohe relative Sensitivität gegenueber Inhibitoren des Wnt Signalweges und eine relative Resistenz gegenüber Inhibitoren der ERK-MAPK Signalweges. 
\smallbreak
Mithilfe der gelernten Faktoren konnten auch bekannte nicht-letale Medikamenteneffekte systematisch identifiziert werden: Zum Beispiel zeigten Inhibitoren des mTOR Komplexes in den Experimenten eine behandlungs-induzierte Morphologie, welcher auf einen positiven Effekt auf Faktor 1 hindeutete. Tatsächlich führte Behandlung von Organoiden mit einem mTOR Inhibitor in validierenden Experimenten zu einer reaktiven Expression des IGF-Signalweg Komponenten Irs-1. Ein ähnlicher Effekt zeigte sich bei MEK Inhibitoren in Bezug auf Faktor 2. In validierenden Experimenten führte die Behandlung zu einer erhöhten Expression von \textit{LGR5} Transkripten.
\bigbreak

In genetisch modifizierten Maus Organoiden wurde ein ähnlicher experimenteller Ansatz verfolgt. Hier wurden drei prominente Faktoren identifiziert: Der erste Faktor unterteilte \textit{Apc} wildtyp von \textit{Apc} mutierten Organoiden. Verlust der \textit{Apc} Funktion führte zu erhöhter Expression von Proliferations- und DNA Reparatur-Markern. Es fuehrte ebenso zu erhöhter Sensibilität gegenüber Perturbationen des Cytoskeletts, zum Beispiel durch Taxane und FAK-Inhibitoren, sowie morphologisch zu einem Verlust der sphärischen und symmetrischen multizellulären Organisation. Organoide mit \textit{Apc}\textsuperscript{-/-} Genotyp zeigten eine Unabhängigkeit von üblicherweise essentiellen Wnt liganden im Kulturmedium. Mit Bezug auf den zellularen Metabolismus, führte der Verlust von \textit{Apc} zu einer Reduktion von Transkripten der beta-Oxidation, sowie einer Akkumulation von Triacylglyceriden und Cholesterolester Speicherlipiden.
\bigbreak

Der zweite Faktor unterteilte \textit{Kras}\textsuperscript{G12D/+} Organoide von wildtyp Organoiden. Organoide mit \textit{Kras}\textsuperscript{G12D/+} Genotyp zeigten Zeichen einr Onkogen-induzierte Seneszenz, reduzierte  Expression von Transkripten der beta-Oxidation, eine Akkumulation von Speicherlipiden, sowie eine deutliche Reduktion von Transkripten der oxidativen Phosphorylierung. Es bestand eine hohe relative Sensibilität gegenüber ERK- und MEK- Inhibitoren und eine geringe relative Sensibilität gegenüber Inhibitoren des EGF Rezeptors. 
\bigbreak

Der dritte Faktor trennte \textit{Apc}\textsuperscript{-/-} Organoide von \textit{Apc}\textsuperscript{-/-} / \textit{Kras}\textsuperscript{G12D/+} Organoiden. Modelle mit  \textit{Apc}\textsuperscript{-/-} / \textit{Kras}\textsuperscript{G12D/+} Genotyp zeigten eine erhöhte Sensitivität gegenüber mTOR Inhibitoren sowie eine erhöhte Expression einer mTORC1-Aktivierungs Signatur.
Wie zuvor, konnten die gelernten Faktoren genutzt werden um nicht-letale Medikamenteneffekte systematisch zu identifizieren. Hierbei zeigten Inhibitoren der GSK3-beta Kinase, bekannte Aktivatoren des Wnt Signalweges, einen deutlichen Effekt auf Faktor 1.
\bigbreak

Die Ergebnisse dieser Arbeit beleuchten primäre molekulare Faktoren der Organoidarchitektur und ihre molekulare Bedeutung für das kolorektale Karzinom. Die Resultate demonstrieren ebenfalls das Potenzial von Organoiden in der integrierten bild-basierten Medikamententestung für die translationale medizinischen Forschung. Hoffentlich tragen die im Rahmen dieser Arbeit generierten Ergebnisse zur Erforschung neuer, präziser Wirkstoffe für das kolorektale Karzinom bei.