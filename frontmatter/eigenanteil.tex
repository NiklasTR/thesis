\begin{flushleft}
This thesis comprises two parts focused on patient derived organoids and engineered mouse colon organoids. 
\bigbreak

The parts of this thesis focused on patient derived organoids were conducted in collaboration with Johannes Betge (Physician Scientist, 2nd medical clinic, Mannheim) and Jan Sauer (PhD Student, Computational Genome Biology Group, DKFZ, Heidelberg) with occasional support by Benedikt Rauscher (PhD Student, Signaling and functional Genomics Group, DKFZ, Heidelberg). 
\smallbreak
Johannes Betge led the multi-center procurement and oversaw the isolation, amplicon sequencing and transcript expression measurements of patient derived colorectal cancer organoids. He also oversaw the RT-qPCR (Figure \ref{fig_253} d) and Western Blot (Figure \ref{fig_261} g) based validation of MEK and mTOR inhibitors in patient dervied organoids, respectively. High-throughput experiments were done with technical support from Johannes Betge. 
\smallbreak
Jan Sauer implemented the image analysis process, comprising the compression of raw images, the maximum contrast projection method, the segmentation of organoids, the extraction of features as well as the estimation of treatment effect vectors through logistic regression. 
\smallbreak
Benedikt Rauscher advised and implemented a version of the transcript expression analysis. 
\smallbreak
Erica Valentini performed the analysis of amplicon sequencing data. \par
I conducted all experiments outlined in the main part of this dissertation unless otherwise specified. Data analysis and visualization was performed by me unless otherwise specified. 
\bigbreak

The parts of this thesis focused on engineered mouse colon organoids were conducted independently from the collaboration outlined above.
\bigbreak

\textbf{Parts of this dissertation have already been published in the following article:}
\bigbreak

Johannes Betge*, \textbf{Niklas Rindtorff*}, Jan Sauer*, Benedikt Rauscher*, Clara Dingert, Haristi Gaitantzi, Frank Herweck, Kauthar Srour-Mhanna, Thilo Miersch, Erica Valentini, Veronika Hauber, Tobias Gutting, Larissa Frank, Sebastian Belle, Timo Gaiser, Inga Buchholz, Ralf Jesenofsky, Nicolai Härtel, Tianzuo Zhan, Bernd Fischer, Katja Breitkopf-Heinlein, Elke Burgermeister, Matthias P. Ebert, Michael Boutros, The drug-induced phenotypic landscape of colorectal cancer organoids. Nature Communications; 3135 (2022). https://doi.org/10.1038/s41467-022-30722-9 *\textbf{shared first authorship}

\bigbreak
The publication corresponds to all experimental results presented in the first chapter of the thesis results with small modifications including the addition of (Figure \ref{fig_216}). The publication covers also smaller parts of the introduction as well as the discussion related to patient derived organoids. My personal contribution to the publication consisted in the performance of all experiments except the amplicon sequencing of organoid samples (conducted by the laboratory engineer T. Miersch), as well as the validating RT-qPCRs, and Western Blots (conducted by the laboratory engineer K. Kaiser). Laboratory engineers A. Falzone, and A. Kerner supported with the establishment and processing of patient derived organoids. All data analysis, visualisations and figures with the exception of Figure 3, Figure 5i, and Figure 6g were created by me. I contributed to the complete writing process and the majority of the results and introduction section were written by me, as well as the discussion on the limits of using multi-omics factor analysis.


\newpage

Throughout the period in which this dissertation was created a set of additional publications was contributed to: 
\begin{itemize} 
    \item \textbf{Niklas Rindtorff}, MingYu Lu, Nisarg Patel, Huahua Zheng, Alexander D'Amour, A Biologically Plausible Benchmark for Contextual Bandit Algorithms in Precision Oncology Using in vitro Data. Arxiv (2019) https://doi.org/10.48550/arXiv.1911.04389
    \item Tianzuo Zhan*, \textbf{Niklas Rindtorff*}, Michael Boutros, Wnt signaling in cancer. Oncogene; 1461–1473 (2017). https://doi.org/10.1038/onc.2016.304
    \item Tianzuo Zhan*, \textbf{Niklas Rindtorff*}, Johannes Betge, Matthias P. Ebert, Michael Boutros, CRISPR/Cas9 for cancer research and therapy. Seminars in Cancer Biology; 106-119 (2019). https://doi.org/10.1016/j.semcancer.2018.04.001
    \item Leonhard Valentin Bamberg, Florian Heigwer, Anna Maxi Wandmacher, Ambika Singh, Johannes Betge, \textbf{Niklas Rindtorff}, Johannes Werner, Julia Josten, Olga Valerievna Skabkina, Isabel Hinsenkamp, Gerrit Erdmann, Christoph Röcken, Matthias P Ebert, Elke Burgermeister, Tianzuo Zhan, Michael Boutros, Targeting euchromatic histone lysine methyltransferases
sensitizes colorectal cancer to histone deacetylase inhibitors. Int. J. Cancer.; 151:1586–1601 (2022). https://doi.org/10.1002/ijc.34155
    \item Tianzuo Zhan, Giulia Ambrosi, Anna Maxi Wandmacher, Benedikt Rauscher, Johannes Betge, \textbf{Niklas Rindtorff}, Ragna S. Häussler, Isabel Hinsenkamp, Leonhard Bamberg, Bernd Hessling, Karin Müller-Decker, Gerrit Erdmann, Elke Burgermeister, Matthias P. Ebert, Michael Boutros, MEK inhibitors activate Wnt signalling and induce stem cell plasticity in colorectal cancer. Nature Communications; 2197 (2019). https://doi.org/10.1038/s41467-019-09898-0
\end{itemize}
*\textbf{shared first authorship}
\bigbreak

A manuscript presenting the results generated during the image-based profiling of engineered mouse colon organoid is in preparation at the time of this thesis submission.
\end{flushleft}