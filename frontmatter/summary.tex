Colorectal cancer belong to the most common and lethal forms of cancer. Organoids are multicellular, three-dimensional in-vitro models which enable the culture of both health and transformed adult tissue, including the culture of healthy and cancerous colon epithelium. Given their properties, organoids are currently being evaluated for diagnostic purposes, as well as for use in therapeutic discovery. In culture, organoids show highly heterogeneous morphologies - both within and between organoid lines. The factors underlying these differences in organoid architecture are, however, not well understood. 


A popular method in the therapeutic discovery context is image-based profiling. Here, in-vitro models are treated with therapeutic candidates and their treatemend-induced morphologies are captured through microscopy and analysed. Given their three-dimensional architecture and more complex growth conditions colorectal organoids have not, until recently, been used in image-based profiling experiments.
\bigbreak



In this thesis, image-based profiling of colorectal carcinoma- and adenoma-models was established and in total more than 2000 treatment conditions were evaluated. Morphological information was analysed jointly with "multi-omics" data using group-factor-analysis. Factors, which captured molecular variation among organiod models, as well as differences in small molecule sensitivity were identified.
\bigbreak

First, organoid models were generated that represented the ccolorectal arcinoma- and adenomastage. For the former, patient derived organoids were derived from endoscopic biopsies of colorectal carcinoma. For the latter, organoids were isolated from a genetic \textit{LSL-Kras\textsuperscript{G12D} CreERT2} mouse model and, via CRISPR, \textit{Apc\textsuperscript{-/-}} insertion/deletions were introduced which yielded four organoid models with distinct genotypes.
\bigbreak

Two primary factors of variation were identified in patient-derived colorectal cancer organoids: The first factor organised organoids according to their size and the activity of the IGF1R signaling pathway. Organoids with a large diameter and high activity of IGF1R signaling were sensitive towards small molecule inhibitors of the IGF1 receptor and resistant towards EGFR inhibition. 
\smallbreak
The second factor organised organoids according to their Lgr5+ stemness. Organoids with a high Lgr5+ stemness signature showed a spherical and symmetric multicellular organisation. These organoids showed a sensitivity towards inhibitors of Wnt signaling and a relative resistance towards inhibitors of ERK-MAPK signaling.
\smallbreak
Using these identified factors, known, non-lethal, small molecule treatment effects were identifiable: For example, small molecule inhibitors of mTOR showed a treatment-induced morphology that corresponded to a positive shift along factor 1. In fact, treatment of organoids with mTOR inhibitors during validating experiments led to a reactive overexpression of the IGF1R signaling effector Irs-1. A related effect was observed in MEK inhibitors with regards to factor 2. Treatment of organoids with MEK inhibitors led to an increased expression of \textit{LGR5} transcripts.
\bigbreak

A similar approach was used in genetically engineered mouse organoids. Here, three prominent factors were identified: The first factor separated \textit{Apc} wildtype from \textit{Apc} mutated organoid models. Loss of \textit{Apc} led to an increase transcript expression of proliferation- and DNA-repair markers. These organoids also showed an increased sensitivity towards perturbation of the cytoskeleton, for example through taxanes or FAK inhibitors, as well as a loss of the spherical and symmetrical multicellular structure. Organoids with an \textit{Apc}\textsuperscript{-/-} genotype grew independently of the otherwise essential Wnt ligands in the culture medium. With regards to changes in the cellular metabolism, loss of \textit{Apc} led to a reduced expression of beta-oxidation related transcripts and to an accumulation of triglycerides and cholesterol-ester storage lipids.
\bigbreak

The second factor separated \textit{Kras}\textsuperscript{G12D/+} organoid models from wildtype organoids. Organoids with \textit{Kras}\textsuperscript{G12D/+} genotype showed signs of oncogene-induced senescence, reduced expression of beta-oxidation related transcripts, an accumulation of triglycerides as well as a marked reduction in the expression of oxidative phosphorylation related transcripts. Organoids showed an increased sensitivity towards ERK- and MEK-inhibitors and a low relative sensitivity towards EGFR inhibitors. 
\bigbreak

The third factor separated \textit{Apc}\textsuperscript{-/-} from \textit{Apc}\textsuperscript{-/-} / \textit{Kras}\textsuperscript{G12D/+} organoids. Double mutant organoid models showed an increased sensitivity towards mTOR inhibitors and an increased expression of mTORC1-activation related transcripts. As demonstrated previously, learned factors were useful to identify non-lethal small molecule treatment effects that moved organoids along the identified factors. For example, inhibitors of GSK3-beta -which are known activators of canonical Wnt signaling- had a marked effect on factor 1. 
\bigbreak

The results presented within this thesis elucidate primary factors of morphological and molecular heterogeneity in organoid models of colorectal cancer and its precursor states. The results also demonstrate the potential of colon organoids and their use in image-based profiling within translational medical research. Hopefully, results collected throughout this work can contribute to the discovery of new treatment options against colorectal cancer. 
