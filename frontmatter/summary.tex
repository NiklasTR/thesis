Colorectal cancer belongs to the most common and lethal forms of cancer. Organoids are multicellular, three-dimensional in-vitro models which enable the culture of both healthy and transformed adult tissue, including the culture of healthy and cancerous colon epithelium. In culture, organoids show highly heterogeneous morphologies. The molecular factors underlying these differences in organoid architecture are, however, not well understood. 
\bigbreak

The aims of this thesis were (1) to understand the factors determining the diversity of organoid morphologies, (2) to identify differences in small molecule treatment sensitivity linked to these factors, and (3) to explore whether non-toxic small molecules existed that could move organoids from one morphology to another morphology along the previously identified factors.
\bigbreak

First, organoid models were generated that represented the colorectal carcinoma and adenoma stages. For the former, patient-derived organoids were derived from endoscopic biopsies of colorectal carcinoma. For the latter, organoids were isolated from a genetic \textit{LSL-Kras\textsuperscript{G12D} CreERT2} mouse model and, via CRISPR/Cas9, \textit{Apc} insertion/deletions were introduced which yielded four organoid models with distinct genotypes.
\bigbreak

Organoid models were then subjected to image-based profiling, a microscopy-based high-throughput assay, and more than 2,000 small molecule treatment conditions were evaluated in total. The resulting morphological information was analyzed jointly with multi-omics data using group factor analysis. The identified factors captured variation of molecular, morphological, and small molecule treatment sensitivity differences.
\bigbreak

Two primary factors of variation were identified in patient-derived colorectal cancer organoids: The first factor organized organoids according to their size and the activity of the IGF1R signaling pathway. Organoids with a large diameter and high activity of IGF1R signaling were sensitive towards small molecule inhibitors of the IGF1 receptor and resistant towards EGFR inhibition. 
\smallbreak
The second factor organized organoids according to their Lgr5+ stemness. Organoids with a high Lgr5+ stemness signature showed a spherical and symmetric multicellular organization. These organoids showed a sensitivity towards inhibitors of Wnt signaling and a relative resistance towards inhibitors of ERK-MAPK signaling.
\smallbreak
Using these identified factors, known, non-lethal, small molecule treatment effects were identifiable: For example, small molecule inhibitors of mTOR showed a treatment-induced morphology that corresponded to a positive shift along factor 1. In fact, treatment of organoids with mTOR inhibitors during validating experiments led to a reactive overexpression of the IGF1R signaling effector Irs-1. A related effect was observed in MEK inhibitors regarding factor 2. Treatment of organoids with MEK inhibitors led to an increased expression of \textit{LGR5} transcripts.
\bigbreak

A similar approach was used in genetically engineered mouse organoids. Here, three prominent factors were identified: The first factor separated \textit{Apc} wild type from \textit{Apc} mutated organoid models. Loss of \textit{Apc} led to an increase in transcript expression of proliferation- and DNA-repair markers. These organoids also showed an increased sensitivity towards perturbation of the cytoskeleton, for example through taxanes or FAK inhibitors, as well as a loss of the spherical and symmetrical multicellular structure. Organoids with an \textit{Apc}\textsuperscript{-/-} genotype grew independently of the otherwise essential Wnt ligands in the culture medium. Regarding changes in the cellular metabolism, loss of \textit{Apc} led to a reduced expression of beta-oxidation-related transcripts and to an accumulation of triglycerides and cholesterol-ester storage lipids.
\bigbreak

The second factor separated \textit{Kras}\textsuperscript{G12D/+} organoid models from wild type organoids. Organoids with \textit{Kras}\textsuperscript{G12D/+} genotype showed signs of oncogene-induced senescence, reduced expression of beta-oxidation related transcripts, an accumulation of triglycerides as well as a marked reduction in the expression of oxidative phosphorylation related transcripts. Organoids showed an increased sensitivity towards ERK- and MEK-inhibitors and a low relative sensitivity towards EGFR inhibitors. 
\bigbreak

The third factor separated \textit{Apc}\textsuperscript{-/-} from \textit{Apc}\textsuperscript{-/-} / \textit{Kras}\textsuperscript{G12D/+} organoids. Double mutant organoid models showed an increased sensitivity towards mTOR inhibitors and an increased expression of mTORC1-activation related transcripts. As demonstrated previously, learned factors were useful to identify non-lethal small molecule treatment effects that moved organoids along the identified factors. For example, inhibitors of GSK3-beta, which are known activators of canonical Wnt signaling, had a marked effect on factor 1. 
\bigbreak

The results presented within this thesis elucidate primary factors of morphological and molecular heterogeneity in organoid models of colorectal cancer and its precursor states. The results also demonstrate the potential of image-based profiling of organoids within translational medical research. 
